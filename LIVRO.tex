\documentclass[semcabeco,showtrims,10pt,conselho,spreadimages]{memoir}

\usepackage[post]{hedraoptions} 
\usepackage[baruch]{hedrastyles}
\usepackage[xetex,chicagofootnotes]{tipografia} %,swiftneue

\usepackage[standart]{toc}

\usepackage{hedraextra}
\usepackage{penalidades}
\usepackage{graficos}
\usepackage{hedralogo}
\usepackage{hifensextras}
\usepackage{fichatecnica}
\usepackage[standart]{aparatos}
\usepackage{tabelas}
\usepackage{versos}

\usepackage{imakeidx} 
\makeindex[program=xindy, options=-C utf8 -L portuguese]
\newcommand\gobbleone[1]{}
\newcommand*{\seeonly}[2]{\ (\emph{\seename} #1)}
\newcommand*{\also}[2]{\emph{cf.} #1}
\newcommand{\Also}[2]{\emph{See also} #1}
\renewcommand\indexname{Índice onomástico}
\makeindex[intoc]


\addtocontents{toc}{\vskip-5pt}
%------------------------------------------ INCLUIR EPIGRAFE EM PART
\let\oldafterpartskip\afterpartskip 

\newcommand\partepigraph[3][60pt]{
\renewcommand{\afterpartskip}{
\vskip#1
\epigraph{#2}{#3}
\vfil}
}
\newcommand\removeepigraph{
  \let\afterpartskip\oldafterpartskip}
%------------------------------------------
\linespread{1.15}

\setcounter{tocdepth}{1}     % amplitude da presença das partes no índice
\setcounter{secnumdepth}{-2} % amplitude da numeração das partes

\makeatletter

\def\footnoterule{\kern-8\p@
    \hrule \@width 2in \kern 7.6\p@} % the \hrule is .4pt high

%\renewcommand{\foottextfont}{\small}
%\footmarkstyle{\normalfont{\raisebox{3pt}{\tiny#1}}}
%\def\@makefnmark#1{{\tiny \raisebox{3pt}{\@thefnmark}}}
\skip\footins=10mm\@plus3mm\@minus2mm
\makeatother

%\fontsize{10.5pt}{14.2pt}\selectfont

\begin{document}

%LUGAR DE NEGRO, LUGAR DE BRANCO?

%Esboço para uma crítica à metafisica racial

%Douglas Rodrigues Barros


%SUMÁRIO%

%Pequena nota introdutória%

%Parte I%

%Fanon contra o misticismo%

%Linguagem e identidade%

%Os significados da dialética%

%\emph{Weltanschauung} do negro%

%Sartre e a dialética espanada%

%A radicalidade do pensamento de Fanon%

%Parte II%

%O movimento negro e o misticismo%

%Os limites do problema%

%Contra o misticismo do trabalho libertador%

%A origem do mito e a construção de um epígono%

%Em busca da África perdida?%

%Uma ilusão necessária contra um mito perigoso%

%Uma crítica necessária%

%Ao pé do muro

\chapter*{}

\emph{Para Douglas Belchior e Adervaldo José dos Santos pelo compromisso
com a luta antirracista.}

\chapter{\emph{Pequena nota introdutória}}

Este curto ensaio foi atropelado pela ordem. Quando escrevia a segunda
parte, o seu autor foi surpreendido pela terrível notícia do assassinato
de Marielle Franco e Anderson Gomes ocorrido em 14 de março de 2017,
data do nascimento de Castro Alves, de Carolina Maria de Jesus, de
Abdias do Nascimento e por contingência do próprio autor.

Eu não conhecia Marielle Franco, só a vi uma vez num debate em que tive
a oportunidade de acompanhar, discordar e conversar, porém, seu
assassinato foi sentido em minha pele. Choramos sua morte. O seu sangue
era o nosso.

Desde então tudo mudou.

Qualquer militante e crítico de esquerda não será mais o mesmo depois
dessa morte. As poucas ilusões com a política brasileira deveriam ser
desfeitas pelos tiros dados em Marielle e no seu motorista. É uma
política de morte que funciona sob pressupostos rentáveis e significante
colonizado.

Nós sabíamos disso? Talvez, mas até então duvidávamos. Hoje não se pode
duvidar mais. Essas mortes, sem dúvida, tem o peso do nosso fracasso.
Devemos nos responsabilizar por elas e por outras tantas que ocorrem
longe da segunda maior cidade do país. Só assim poderemos dar um basta.

Urge imaginarmos outro mundo, outras formas de sociabilidade, outra
dinâmica de vida em que o componente racial não seja decisivo na escolha
de quem deve morrer.

Este curto ensaio foi atropelado pela ordem, mas não esmoreceu diante
daquilo que era sua tarefa: desiludir-nos com os misticismos para
enfrentar a barbárie imposta.

É o que tenho a oferecer como trabalho de luto. Um trabalho que espero
possibilite o amadurecimento de nossa luta.

O autor

\part{FANON CONTRA O MISTICISMO}

\begin{quote}
Eu lhe direi: é o meio, é a sociedade que é responsável pela sua
mistificação. Isso dito, o resto virá por si só. E sabemos do que se
trata. Do fim do mundo...(\emph{Fanon})
\end{quote}

\chapter{Linguagem e identidade}

*

"Nós desconfiamos do entusiasmo", assim se expressa Fanon na introdução
de sua obra como quem cita uma verdade lúcida desperta pelos sinais
daqueles que não tiveram irmandade com as coisas e foram esmagados por
fileiras de carruagens atadas às costas. Entusiasmar-se é tornar-se
impotente.

Com alguns farelos recentemente caídos da mesa da elite econômica
durante um curto intervalo de tempo\footnote{Trata-se do tempo das vacas
  gordas em que a crise mundial propiciou um forte investimento e
  crescimento nas importações de \emph{commodities} e o superávit
  primário passou de 3,7\% para 4,5\%. Com isso, houve a captura de
  grande parte da esquerda tanto material quanto espiritualmente.},
entretanto, a esquerda e grandes setores do \emph{movimento negro}
pareciam se entusiasmar e abandonaram qualquer princípio que não o de se
incluir no jogo.

Tornar-se \emph{colaborador}\footnote{Colaborador como colaboracionistas
  (remetendo aos anos hitleristas) implica executar um trabalho
  independente de suas sequelas. A esse respeito, mas não no sentido
  aqui exposto Cf. ARANTES, P. E. \emph{O novo tempo do mundo: e outros
  estudos sobre a era da emergência}. São Paulo: Boitempo, 2014 p.
  101-140} tinha um preço a ser pago: a elevação do mito em verdade, o
abandono e a acusação contra qualquer posição que pensasse para além das
miudezas e misérias cotidianas sob a égide do mundo da mercadoria.

Competiu, desse modo, a esquerda realizar o trabalho sujo com
zelo\footnote{Trabalho muito bem demonstrado por Francisco de Oliveira:
  ``Sindicatos de trabalhadores do setor privado também já estão
  organizando seus próprios fundos de previdência complementar, na
  esteira daqueles das estatais. Ironicamente, foi assim que a Força
  Sindical conquistou o sindicato da então Siderúrgica Nacional, que era
  ligado à CUT, formando um ``clube de investimento'' para financiar a
  privatização da empresa; ninguém perguntou depois o que aconteceu com
  as ações dos trabalhadores, que ou viraram pó ou foram açambarcadas
  pelo grupo Vicunha, que controla a Siderúrgica. É isso que explica
  recentes convergências pragmáticas entre o PT e o PSDB, o aparente
  paradoxo de que o governo de Lula realiza o programa de FHC,
  radicalizando-o: não se trata de equívoco, mas de uma verdadeira nova
  classe social, que se estrutura sobre, de um lado, técnicos e
  intelectuais \emph{doublés} de banqueiros, núcleo duro do PSDB , e
  operários transformados em operadores de fundos de previdência, núcleo
  duro do PT. A identidade dos dois casos reside no controle do acesso
  aos fundos públicos, no conhecimento do ``mapa da mina''. (in:
  OLIVEIRA, F. Crítica à razão dualista, o Ornitorrinco. São Paulo:
  Boitempo, 2003 p.146)}: se, se matam milhares de jovens anualmente, a
maioria negros, que importa? Pensemos em nosso próximo candidato.

Na rede social, agora convertida em Areópago, desfilam mil gênios
consagrados as verdades provisórias em busca de likes e comentários. O
Facebook, separando cada um no seu nicho próprio, construiu a
\emph{máquina do mundo} que procura não só a \emph{implicação} como a
\emph{mobilização total} de seus usuários. A subordinação e organização
``do e para o'' trabalho agora passam por essa ``ferramenta'' que altera
radicalmente as relações de sociabilidade de seus usuários.

Por outro lado, ainda será necessário refletir sobre a insensibilidade
social e a invisibilidade do massacre cotidiano que se perpetua aqui
desde que o colonizador chegou\footnote{O que em todo caso não faremos
  aqui.}. Se antes a carne negra era a mais barata e rentável do
mercado, agora, é necessário dizimar o seu excesso. Entre passado e
presente, a infâmia que atende pelo nome de \emph{racismo}.

O passado, porém, é lição para se meditar, não para reproduzir, dizia
Mario de Andrade\footnote{ANDRADE, M. \emph{Pauliceia desvairada}.
  Barueri: Ciranda Cultural, 2016 p.18}, e conquanto não estejamos
dispostos a ruir sob os maus auspícios de um romantismo estéril, devemos
perguntar o que é ser negro atualmente sem cair na cilada de uma
identificação remota com um passado inexistente.

A \emph{identificação} é a forma de ligação emocional mais profunda e,
com ela, dificilmente se ultrapassa as limitações que forjam a
experiência concreta na formação do \emph{Eu} com o mundo\footnote{FREUD.
  S. \emph{Psicologia das massas e analise do eu}. Porto Alegre: LPM,
  2017.}. São nessas lições que Freud deixou um grande aprendizado sobre
o funcionamento da \emph{psicologia das massas.} Psicologia que hoje em
dia fora capturada pelo gozo escópico do narcisismo autorreferenciado
das redes sociais\footnote{Essa noção foi me passada pelo grandioso
  artigo de Patrícia do Prado Ferreira Lemos intitulado: \emph{Entre o
  olho e o olhar: o gozo escópico no Facebook.}}.

Sendo assim, o que governa a identificação é a simpatia que impõe não
somente a imitação de características em comum, como também sua defesa
acrítica. É na identificação que a ligação mútua entre indivíduos da
massa é produzida pela característica afetiva. Essa se encontra por
vezes numa qualidade particular em comum, numa cor em comum, num
fenótipo em comum e numa história mítica em comum.

É daí que a multiplicidade que constitui o \emph{Eu} particular deixa de
importar: o que importa é aquilo no seu Eu exterior que se parece
comigo, um cabelo em comum, por fim, uma \emph{raça}.

Tanto a identificação com o branco quanto a identificação com o
\emph{negro} -- a despeito do fato de que esta última palavra consegue
ser ao mesmo tempo um substantivo e um adjetivo -- eliminam de si
qualquer capacidade reflexiva mais profunda. Ser igual no infortúnio ou
no privilégio significa que alguma coisa sustenta essa condição. Sabemos
o que é.

Nesse sentido, Fanon segue sendo o arsenal crítico contra a leviandade e
preguiça daqueles que falam em seu nome e que se tornaram reles
colaboracionistas\footnote{Há diferenças entre os colaboracionistas e os
  francamente fascistas que podem ser resumidas grosseiramente da
  seguinte maneira: os fascistas faziam e sabiam o que faziam, tinham
  clareza das ordens que seguiam e onde queriam chegar, ao passo que os
  colaboracionistas só estavam tentando trabalhar e seguir a vida
  normalmente sem se envolver com algo para além do que o limite
  impunha, ou seja, ``eles faziam, mas não sabiam o que faziam''. Esse
  resumo grosseiro que faço pode ser melhor estudado no ensaio de Paulo
  Arantes (2014, p. 101-141) intitulado \emph{Sale Boulot} no livro já
  citado, ou ainda, em Hannah Arendt (Cf. ARENDT, H. \emph{Eichmann em
  Jerusalém. Um Relato Sobre a Banalidade do Mal}. São Paulo: Cia das
  Letras, 1999).~}. Isto impõe uma reflexão crítica sobre a
hegemonização de um determinado setor do movimento negro que impôs uma
pauta na qual alguns temas são francamente reacionários. Atualmente,
esta hegemonização impossibilita qualquer horizonte para além das formas
impostas socialmente pelo modo de sociabilidade capitalista e, portanto,
nosso arsenal se voltará contra esta mesma hegemonia.

**

Fica evidente que investigar Fanon é colocar-se um problema importante:
o que o presente significa diante de Fanon?\footnote{Utilizamos aqui a
  mesma ironia de Adorno na abertura de seu Skoteinos que inverte a
  questão: o que Hegel significa para o presente para o que presente
  significa diante de Hegel. in: ADORNO, T. W. \emph{Três estudos sobre
  Hegel}. São Paulo: Editora Unesp, 2013 p.71} A linguagem é o cerne de
suas descobertas. Atribuir importância central à linguagem implica se
pôr sob a perspectiva socialmente construída e o critério de refletir
sobre as limitações que as formas da linguagem impõe.

Assentar esse problema implica tomar posição. Não nos importando a
originalidade dos novos termos, conceitos de última hora fundados na
conjuntura dos últimos dias, ``novas epistemologias'' que não criticam
profundamente aquilo que possibilita a manutenção do racismo, é preciso
dizer que: a elevação da identidade hipostasiada implica uma
subordinação colonizada. Sonhar com uma cultura limpa e desinfectada do
Outro é, assim, mais que uma patologia cínica, é uma postura ideológica
fetichista que, encontrando corpo político, se torna altamente
reacionária.\footnote{Como disse Mbembe: ``ao reduzir o corpo e o ser
  vivo a uma questão de aparência, de pele ou de cor, outorgando à pele
  e à cor o estatuto de uma ficção de cariz biológico, os mundos
  euro-americanos em particular fizeram do Negro e da raça duas versões
  de uma única e mesma figura, a da loucura codificada'', em que valha
  os pesos dessa construção simbólica a construção de ``raça''
  possibilitou inúmeras catástrofes durante a modernidade (in. MBEMBE,
  A. Crítica à razão negra. Lisboa: Antígona, 2014 p. 11)}

Isto porque, a condição própria da existência se efetiva enquanto um
olhar do Outro e uma compreensão entregue à nós por esse Outro. A
construção da identidade é um processo em devir, algo que jamais pode
atingir um ponto de estabilidade. A identidade é uma tarefa instável que
possibilitando questionar os hábitos e tradições, pode também chafurdar
no lamaçal da identificação estanque e narcísica. Nascida como ficção,
como dizia Bauman\footnote{BAUMAN, Z. \emph{Identidade: Entrevista a
  Benedetto Vecchi}. Rio de Janeiro, Zahar, 2005.}, suas contradições
inerentes ganharam grande relevância quando a sensação de
\emph{pertencimento}, seja o de uma classe ou o de um Estado-nação,
entrou em declínio.

Por isso, a afirmação da diferença de tratamento da linguagem entre um
\emph{negro} e um \emph{negro}, ou entre um \emph{negro} e um
\emph{branco}\footnote{FANON, F. \emph{Pele negra, máscaras brancas}.
  Salvador: EDUFBA, 2008.}, feita por Fanon, serve, assim, para dar uma
sacudidela aos ossos da estrutura de uma diferença de raças
simbolicamente criada e ainda hoje naturalizada. E aqui é importante
dizer que Fanon, em seu famoso livro, teve um \emph{insight} genuíno que
possibilita entendermos não apenas esses processos contraditórios, como
avançar para além deles.

A invenção da diferença sob os pressupostos da exploração colonial impõe
ao Negro uma realização impossível. \emph{Um Real} impossível,
traumático, em que a rede simbólica de reconhecimento mútuo está
fechada. O modo simbólico da linguagem como resultado de uma contingente
luta complexa pelo poder sociossimbólico é abordado aí no sentido da
exclusão que esse processo efetivou para o negro.

O negro, nesse sentido, não é um Outro do branco em sua universalidade
colonizadora, mas um inexistente numa universalidade que elide ao negro
qualquer possibilidade de reconhecimento. Inversamente, porém, o negro
só existe em relação a essa exclusão do domínio branco. Ultrapassar
essas limitações é o fim previsto na radicalidade fanoniana.

Talvez, seja por isso que Fanon advirta desde o início que, em se
tratando de uma análise psicológica do negro, não se deve esquecer dos
elementos que fundamentaram essa ordem sociossimbólica, quer dizer: não
podemos perder de vista a história socioeconômica que engendrou essa
noção de diferenciação.

Em termos simples, enquanto o branco alçou-se a condição de Sujeito,
para o negro essa condição está vedada pelos processos de colonização.
Entretanto, ao contrário do que parece sugerir, a condição buscada na
análise de Fanon não é a resolução pura e simples dessa condição de um
não-sujeito para a condição de um sujeito como ponto de síntese e
resolução dos conflitos.

Fanon de nenhuma forma poderia incidir nessa ingenuidade pois sabia que,
independente dos espaços de simbolizações, a lacuna e a castração se
mantém inalterada. Por isso, esmiuçar as patologias sociais criadas pelo
sistema de linguagem dominante requer um desnudamento da relação de
sujeição. Essa práxis-teórica aponta o limite a ser ultrapassado. É a
neutralidade do marco simbólico da linguagem que está em disputa e com
ela a própria noção do que é \emph{ser Negro}.

Por isso, desde o início, o destino da identidade em si mesma está
selado. O paradoxo em questão é que o próprio fato de não haver uma
identidade hipostasiada, na qual se possa fundamentar ontologicamente o
\emph{ser Negro}, é o que torna possível a efetiva resistência negra a
partir da implosão da estruturação sociossimbólica.

A questão da linguagem, desse modo, determina uma forma de ser no mundo,
de estar aí em \emph{relação a} fundamentando-se a partir dos processos
sociais implicados no mundo concreto. Assim, se, por um lado, adotar a
linguagem do colonizador implica uma desestruturação da identidade, por
outro, é a partir dela que se toma posição contrária e acerca-se dos
seus limites.

Há duas posições antinômicas que Fanon faz questão de evidenciar:
\emph{a}) aquela de super-identificação com os mecanismos colonialistas
-- adotando e privilegiando os aspectos dominantes da colônia, a
branquitude, a europeização, etc., -- e; \emph{b}) aquela que, \emph{a
negando}, busca um retorno a si e se redobra em defender suas origens.
Ambas são patologias mistificadoras.

São os processos implicados na aproximação com a linguagem do
colonizador que instauram a negação de si para acatar acriticamente as
formas da universalidade imposta. Há uma questão de fundo que ressoa; a
identificação pura e simples com a alteridade imposta do colonizador
leva a manutenção das relações de subordinação.

Assim, ao demonstrar a situação de assimilado do martinicano, Fanon
deixa claro que o que regula seu processo psíquico é um desavim consigo
mesmo, uma negação de sua humanidade, por ver no \emph{Outro}
colonizador a capacidade de sua realização. O que está implicado nisso
são as condições de possibilidade no qual processo de colonização, e
subordinação as suas imposições, torna-se um processo autorreferente de
realização. O processo aí é tão totalizador que do mesmo modo o
comportamento da mulher negra, analisado por Fanon, em relação ao branco
europeu evidencia os mecanismos de captura da subjetividade e de sua
castração egóica que vê no colonizador, com suas características
fenotípicas, a possibilidade de realização do próprio ego. O resultado
disso é que o processo de embraquecimento já está todo articulado por
uma posição cuja antinomia negro/branco está naturalizada e é aceita no
registro simbólico.

É essa abstração real da raça\footnote{Tomamos este termo emprestado de
  Marx para quem o capital é uma abstração real e empregamos aqui no
  mesmo sentido.} -- que ao mesmo tempo que fundamenta a relação social
funda sua forma categorial -- um processo no qual a justificação
excludente se dá no plano sociossimbólico. Isso passará a governar os
destinos individuais guiando-os para uma submissão frente aquilo que
aparece como o "bom". É essa estruturação da subjetividade colonizada
que importa à Fanon.

Tendo isso em vista chega-se à conclusão de que o processo entre acatar
essa condição ou fugir dela em busca de um retorno originário se coloca
como algo imediatamente interno ao processo, quer dizer: as duas
posições são coniventes com os termos erguidos pelo colonizador.

Assim, a análise empreendida sobre os romances \emph{Je suis
Martiniquaise}, \emph{Nini} e o de René Maran demonstra como no nível
simbólico das personagens os resultados da colonização já estão postos.
É como se os romances em sua possibilidade de desnudamento de um
\emph{etos} fossem mais impregnados de verdade do que a empiria da vida
do \emph{aqui} e \emph{agora}.

A incapacidade de Jean Veneuse -- personagem de Maran comentado por
Fanon -- de concretizar sua relação amorosa com uma europeia desnuda
como o processo de inferioridade circunstanciada por uma \emph{psique
abondônica} -- quer dizer, aquela cujo trauma de abandono na infância
impossibilita a realização de relacionamentos duradouros por uma
autocomplacência inferiorizada -- se efetiva a despeito da camisa de
força dada pela \emph{racialização}.

A busca por um retorno à pátria substancial e orgânica só revela a
impotência desse neurótico em realizar-se por se agarrar nas definições
impostas pelo modo de controle colonial\footnote{Aliás como já
  salientava Bauman (op. cit. p. 35): ``O anseio por identidade vem do
  desejo de segurança, ele próprio um sentimento ambíguo''.}. Por isso,
``Jean Veneuse não representa um exemplo das relações negro/branco, mas
o modo com que um neurótico, acidentalmente negro, se comporta\footnote{Idem,
  p.81}''. A posição de uma autoflagelação, de uma autopiedade, e da
desconfiança geral com todo o diferente marca a postura desse Fiódor
Karamazov negro e bondoso.

A questão é muito simples: tendo um trauma de abandono na infância
injustificado, logo se tenta justificá-lo pelo componente racial. O
paradoxo é que sendo preto não posso ser amado e se for amado, e
corresponder esse amor, nada me garante que não estou me aproveitando
dele por ser preto igual aos outros -- ``como os pretos que adoram carne
branca''. Resultado, não posso porque quero e não posso porque posso e
serei igual aos outros.

Independente de sua forma, o que Fanon parece sugerir é que a lógica
interna do movimento da subjetividade dessas personagens eivadas de
preconceitos colonialistas se vê solapada quando passa de um extremo
para o extremo oposto e se funda numa unidade supostamente mais elevada:
o embraquecimento ou sua negação em nome de um retorno às origens nada
mais é do que os limites impostos pela condição de desumanidade absurda
na colônia.

Se a primeira opção dos oprimidos é tentar se livrar daqueles que o
oprimem, enquanto a segunda é dele se aproximar negando-se a si mesmo,
ambas fracassam quando não percebem que a identidade de sua posição está
mediada pelo Outro de modo que para ultrapassar essa condição é
necessário transformar substancialmente o conteúdo dessa própria
posição.

Isso não significa querer se pôr no atual sistema de visibilidade
reivindicando uma representatividade limitante e limitada, mas
transformar radicalmente esse próprio sistema de visibilidade cuja raiz
é econômico-social. É como se para o negro restasse não apenas a negação
de sua posição imposta por um sistema de linguagem que o subordina -
negação que permanece em seus limites simbólicos - como ainda é
necessário negar o próprio espaço simbólico.

A linguagem, com efeito, torna-se uma via de mão dupla: por um lado, ela
lança o seu portador no mundo social, por outro, no caso dos
colonizados, ela subordina suas aspirações ao império da
\emph{metrópole}. Assim, no registro de visibilidade colonizador, a
suposta realização, desse indivíduo limitado, se dará quanto maior for
sua proximidade como os modos dos ``civilizados'', ou seja, quanto maior
for seu embranquecimento.

Com efeito, a \emph{Ideia} que se faz do negro, enquanto uma categoria
cuja constituição é de subordinado - de um assujeitado sem ser sujeito -
o reduz a caricatura e o coloca em termos limitados e limitantes
definidos pela própria linguagem. Por isso "compreende-se... que a
primeira reação do negro seja a de dizer \emph{não} àqueles que tentam
defini-lo''.\footnote{Idem, p.48, \emph{grifo meu}.} É dessa limitação
que surge uma espécie de antinomia colonizadora, uma espécie de
\emph{ou}, \emph{ou}: ou, se aproxima do civilizado, ou, o rejeita
estabelecendo uma espécie de fuga particularista.

O contra-argumento de Fanon, deixado de lado por grande parte daqueles
que dizem segui-lo, é que os próprios mecanismos subordinadores abrem
espaço para a resistência na medida em que fundamentam uma espécie de
excedente, de um \emph{não-lugar} para o negro que acaba por engendrar
uma posição política.

Paradoxalmente, a referência discursiva posta na linguagem da metrópole
confirma o negro como o fundamento estabelecido sobre o qual opera a
linguagem do colonizador. Quando o negro surge como estrutura patológica
na psique branca, como demônio, pecado e sexo abundante, é porque a
relação da identidade branca já está implicada na identidade negra.

Assim, é como reação à dominação do colonizador, efetivada pelo negro,
que a consciência se transforma em vontade política ativa para afirmar
sua identidade. Fanon revela que assim que se alcança essa consciência,
o indivíduo não só se integrou ao universo do colonizador, como agora, é
capaz de implodi-lo.

A morte do colonialismo deve ser "real", no registro de operações
sistêmicas, e simbólica, no registro linguístico que estruturam tais
operações.

Como é deliciosa a forma como Fanon posiciona o sistema de estrutura
simbólica para demonstrar isso ao comentar o livro de René Maran: "só
sei de uma coisa", diz o crítico citando o literato, "que o preto é um
homem igual aos outros, um homem como os outros, e que seu coração, que
só parece simples aos ignorantes, é tão complexo quanto o do mais
complexo dos europeus''\footnote{Idem, p.71}.

Está demonstrado com eloquência como se funda um Eu totalmente
dependente de um Outro negativo que aparece como componente necessário
da identidade e, ao reconhecê-lo, é capaz de desnudar os limites e se
colocar para além deles e da própria necessidade de reconhecimento ao
abolir as diferenças fictícias criadas socialmente pela exclusão.
Noutros termos, se explode o real absurdo ao desnudar completamente o
seu escândalo.

\chapter{Os significados da dialética}

Temos visto até agora que a posição crítica de Fanon não constitui um
retorno mágico à identidade senão seu descentramento contra a presunção
de universalidade baseada na noção de \emph{Sujeito}. O desnudamento do
processo de subordinação imperante na linguagem imprime um primeiro
movimento que nega a substância social da qual essa linguagem emerge, ao
mesmo tempo em que essa substância social se transforma quando seus
limites se tornam visíveis por essa particularidade. Nesse processo, a
identidade já absorveu e reestruturou a concretude social.

É essa posição eminentemente dialética que permite a Fanon escapar de
uma espécie de essencialismo do racismo e identificá-lo à sombra dos
processos modernos cujo cerne é a economia. Assim, quando Fanon fala de
estrutura, ele está falando dos processos econômicos-sociais que
engendraram não apenas a exclusão como a concorrência. Por isso, ao
rebater o essencialista Mannoni, expõe o seguinte:

\begin{quote}
``Poderíamos retrucar que este desvio da agressividade do proletariado
branco na direção do proletariado negro é, fundamentalmente, uma
consequência da estrutura econômica da África do Sul. Que é a África do
Sul? Um caldeirão onde 2.530.300 brancos espancam 13.000.000 de negros.
Se os brancos pobres odeiam os pretos não é, como nos faz entender
Mannoni, porque `o racismo é obra de pequenos comerciantes e de pequenos
colonos que deram duro durante muito tempo sem sucesso' Nada disso, é
porque a estrutura da África do Sul é uma estrutura racista''\footnote{Idem,
  p.86}
\end{quote}

Os processos analisados por Fanon dizem respeito a lógica do fracasso de
colonização que produz anomalias ligadas ao terreno histórico-social de
onde emergem. Sua posição desnuda os processos subjetivos e objetivos
socialmente conduzidos pelas modernas forças de produção e reprodução
social. Não é à toa que Fanon retruca as posições abstratas de seus
contemporâneos: ``Ao considerar abstratamente a estrutura de uma ou
outra exploração, mascara-se o problema capital, fundamental, que é
repor o homem no seu lugar''\footnote{Ibidem.}.

O feito de Fanon é combinar o caráter constitutivo do negro em sua
atividade no mundo com o viés patológico da própria noção de negro;
quando ambos são pensados juntos, como uma característica recíproca,
então conseguimos captar a própria patologia que constitui a realidade
colonizada. ``A inferiorização é o correlato nativo da superiorização
europeia. Precisamos ter a coragem de dizer: é o racista que cria o
inferiorizado''.\footnote{Id., p.90}

Esses processos inter-objetivos\footnote{Este termo tomei emprestado de
  Sílvio Rosa Filho.}, por assim dizer, desestruturam todo o complexo
social dos colonizados, solapam suas estruturas de identidades e
recriam, à luz do processo, novas formas de sociabilidade. Uma vez
desestruturada a identidade, o choque não permite mais um retorno ao que
foi anteriormente: ``Uma ilha como Madagascar, invadida de um dia para o
outro pelos ``pioneiros da civilização'', mesmo que esses pioneiros
tenham se comportado da melhor maneira possível, sofreu uma
desestruturação''.\footnote{Id., p.93}

Tais relações obviamente não incidem apenas na psique daquele que se vê
invadido por um Outro negador, mas na relação, para Fanon, entre a
consciência e o contexto social. É na medida em que a diferenciação do
processo discriminatório, surgido pela colonização, se efetiva que se
impõe para mim a alteridade que tenho que alçar. Fazer-se branco é uma
tentativa ilusória de obrigar o branco a reconhecer a minha humanidade.

À essa situação patológica, que dá movimento aos processos de
colonização tanto material quanto espiritual, Fanon encontra uma via de
superação concreta: o complexo de inferioridade só surge numa sociedade
patologizada em que o racismo é estrutural, só com a mudança das
estruturas sociais é que se pode ultrapassar está condição inumana. Não
há ilusões.

A busca é a de tornar o sofredor ``capaz de escolher a ação (ou a
passividade) a respeito da verdadeira origem do conflito, isto é, as
estruturas sociais''\footnote{Idem, p.96}. Com isso, Fanon torna-nos
ciente de que a forma de ultrapassar o estado posto pelo modo de
sociabilidade colonizado é uma escolha que negue, ainda que
abstratamente, o Todo concreto fundado por esta mesma sociabilidade. Aí
está colocada uma renúncia à esperança nostálgica de um retorno à pátria
perdida.

Não há esperanças de criar uma nova Ordem orgânica que tenha por função
abolir a individualidade. É, pelo contrário, a afirmação do indivíduo e
a negatividade abstrata impressa por uma identidade evanescente que faz
com que o Todo concreto seja modificado em sua raiz\footnote{Nesse ponto
  se encontra a força da argumentação de Fanon. A formação da identidade
  é um componente fundamental na experiência da consciência do negro, no
  entanto, enquanto componente relacional, embora, a colonização tenha
  castrado essa capacidade de uma relação equitativa com o outro, a
  resposta se dá na sua forma de se fazer conhecer. Essa
  individualidade, por definição, está como algo além do jogo imposto
  pelo modo sociossimbólico opressor, e como tal, irrompe com a inversão
  do racismo em racialização da própria identidade que desnuda a relação
  de exploração e opressão, além da unilateralidade de sua própria
  posição.}.

É por isso que estão distantes dos ensinamentos de Fanon tanto aqueles
que defendem uma submissão voluntária e a aceitação do indivíduo aos
pressupostos colonialistas -- tidos como a totalidade concreta; quanto,
aqueles que afirmam inexoravelmente um retorno às raízes
pré-coloniais...

\chapter{\emph{Weltanschauung} do negro}

É a partir desse \emph{não-lugar} que Fanon busca demonstrar a situação
de completo expatriado do negro. Em \emph{A experiência vivida}, talvez
o mais desconcertante e difícil capítulo de seu livro, temos o complexo
desenvolvimento das relações intersubjetivas que fundamenta a psique
negra numa espécie de \emph{Fenomenologia} existencial.

Esse marco conceitual nos permite abordar a situação de plena abertura
vivida pelo negro. A ontologia incapacitada, ou incompleta, pela
dolorosa cisão enfrentada pelo negro é o que sustenta sua realidade
psíquica. A questão que podemos colocar para Fanon, que diz que ``há, na
\emph{Weltanschauung} de um povo colonizado, uma impureza, uma tara que
proíbe qualquer explicação ontológica\footnote{Idem, p.103}'', e se essa
proibição não desnuda toda a fragilidade da operação hegemônica colonial
fundada no caráter mistificador socialmente produzido contra o negro?

Parece-me que a resposta é sim. Os caminhos barrados junto à ontologia
colocam para o negro uma outra alteridade que lhe é contraposta e na
qual sua referência não está localizada nos limites da \emph{vivência}
em seu próprio corpo com relação ao outro: o domínio colonial suprime de
si o mundo do colonizado e, com isso, elide também sua identidade. Com o
branco à espreita, o mundo estruturado do negro entra em colapso.

Isto elucida dois acontecimentos que são inter-relacionados no
pensamento de Frantz Fanon: por um lado, toda identificação hegemônica
se revela mutável -- inclusive aquela do negro com seu sistema de
referência -- por outro, ter isso revelado manifesta a consequência
contingente de uma luta que se dá no âmbito histórico-social.

Ora, é no próprio corpo que se marca a diferença do negro com relação ao
branco. Os desdobramentos dessa diferença cairão em fantasmagorias
psíquicas sexualizadas explicadas de modo profundo por Fanon, inclusive
no que diz respeito a sua redução à genitália. Aqui cabe, porém,
demonstrar que o campo de batalha entre os desejos secretos e as
proibições simbólicas são aquilo que objetificará o negro. Como
resultado, em sua formação (\emph{Bildung}) encontrará impossibilidades
de se descobrir como agente histórico e ativo. Ao negro em situação
colonial está vedado o componente relacional e com ele a possibilidade
de reconhecimento pelo Outro.

Essa espécie de fenomenologia do negro levada adiante por Fanon, quase
como Descartes ao escrever em frente de sua lareira, desnuda a relação
de objetificação corporal que destroçará a subjetividade ao abstrair
todo o \emph{ser Negro} a mera condição epidérmica.

Tal situação corrobora para uma redescoberta dos limites que o indivíduo
estabelecia para si mesmo: ocupar um determinado lugar, ir ao encontro
de outro e ver o outro desaparecer, assim como Fanon deixa claro, repõe
toda uma esfera de significações que permitem ao negro se reconhecer
enquanto tal naquilo que ele mesmo como martinicano -- ou seria um
brasileiro? --, era incapaz de se reconhecer, isto é, em sua condição de
negro.

O próprio indivíduo negro arrancado de sua certeza ao resistir essa
objetificação racista é, no fundo de seu âmago, marcado por ela. A
questão terrificante é que quando o indivíduo negro se descobre por fim
negro já pesa em suas costas a melanina e com ela todas as referências
que o mundo dominante e branco lhe traz, quer dizer: preconceitos, taras
raciais, fetichismo...

É natural que aqui esteja colocada uma questão importante para os
desnudamentos da dominação racial: como os negros subjetivam sua
condição?

Fanon responde esta questão de maneira genuína e original: ao querer ser
só um homem entre outros homens, me reduziram a minha cor, ao assumir
minha cor, assumo os ancestrais escravizados e linchados, porém, essa
situação permanecer vazia em sua estrutura simbólica, ultrapasso ela
quando descubro que todas as formas de apreensão que tenho de mim mesmo
passa pelo Outro; como negro apreendo a mim mesmo pelo branco, embora
sabendo que o branco nada sabe do meu Eu, é ele que me nega e ao me
negar me constitui.

Vou adiante e descubro que sou neto de escravos, assim como um branco
foi neto de camponeses explorados e oprimidos: ``na América, os pretos
são mantidos à parte. Na América do Sul, chicoteiam nas ruas e metralham
os grevistas pretos. Na África Ocidental, o preto é um animal. E aqui,
bem perto de mim, ao meu lado, este colega de faculdade, originário da
Argélia, que me diz: `Enquanto pretenderem que o árabe é um homem como
nós, nenhuma solução será viável'''\footnote{Idem, p.106}. Logo, todas
as diferenças se revelam como diferenças nenhuma. É a partir dessa
condição de expatriado que culmina num não-lugar reservado ao negro que
se dá a possibilidade de uma indiferença as diferenças à medida que a
descoberta sobre as diferenças revela o componente da estrutura social
que as fomentam.

O problema se desdobra: como os negros subjetivam sua condição de
explorados e oprimidos?

Não basta apenas reconhecer-se em sua condição, pois, este é o aspecto
mais simples porque imposto, é preciso compreender que, sendo impossível
livrar-se de um \emph{complexo inato} e deixar de afirmar-se como negro,
deve fazer-se \emph{conhecer} implodindo o próprio espaço de construção
simbólica estruturado pelas relações sociais de exploração.

O paradoxo é que se estruturando o corpo positivado do negro como algo
místico não se pode fundamentá-lo ontologicamente e é justamente isso
que possibilita uma efetiva resistência. Sendo o negro sobredeterminado
pelo exterior não é escravo da ideia\footnote{Na visão de Fanon,
  sustentada pela análise sartriana, são os judeus que são escravos das
  ideias que se fazem dele. O que ele quer dizer que ambos, o negro e o
  judeu, em seu infortúnio sofrem do racismo, porém enquanto o judeu
  pode disfarçar sua origem, o negro a carrega nos tons da pele.}, mas
da sua própria aparição.

E aqui a bela poesia de Victoria Santa Cruz vale como ilustração:

De hoy en adelante no quiero\\
laciar mi cabello\\
No quiero\\
Y voy a reírme de aquellos,\\
que por evitar -según ellos-\\
que por evitarnos algún sinsabor\\
Llaman a los negros gente de color\\
¡Y de que color!\\
NEGRO\\
¡Y que lindo suena!\\
NEGRO\\
¡Y que ritmo tiene!\\
NEGRO NEGRO NEGRO NEGRO\\
NEGRO NEGRO NEGRO NEGRO\\
NEGRO NEGRO NEGRO NEGRO\\
NEGRO NEGRO NEGRO\footnote{De hoje em diante não quero\\
  alisar meu cabelo\\
  Não quero\\
  E vou rir daqueles\\
  que para evitar - segundo eles -\\
  que para evitarmos algum dissabor\\
  Chamam os negros de gente de cor\\
  E de que cor?!\\
  NEGRO\\
  E como soa lindo!\\
  NEGRO\\
  E olha esse ritmo!\\
  NEGRO NEGRO NEGRO NEGRO\\
  NEGRO NEGRO NEGRO NEGRO\\
  NEGRO NEGRO NEGRO NEGRO\\
  NEGRO NEGRO NEGRO

  In: Me gritaron negra, Victoria Santa Cruz (tradução própria).}

É com isso, com esse aparecer, que a referência ao negro, o esforço
colonial para conter e categorizar mistificando-o, produz formas de
resistência como princípio ativo contra a força opressora. O poder
opressor gera a forma de resistência. O princípio ativo contra a força
opressora, a capacidade de mediar, não apenas elabora as formas de
estruturação do simbólico como ainda transforma o próprio núcleo da
identidade. Já que o negro não é reconhecido é preciso fazer-se
conhecer.

A consciência do círculo infernal, que captura o negro reduzindo-o e
abstratificando sua existência, é a que possibilita escapar desse
círculo. Fanon no seu caminho ao expor essa fenomenologia do negro
demonstra, porém, como alguns caminhos estão limitados a tornarem o
negro um objeto passivo, são eles:

\emph{a}) Um dos caminhos é a busca pelo conhecimento como forma
política, mas como sugere Fanon: \emph{A razão contra a irracionalidade
racista produz náusea}. Nesse ponto, é importante lembrar, ainda que
para nossos objetivos isso não tenha tanta importância, de como Fanon
estabelece uma relação de equidade no infortúnio entre o negro e o
judeu\footnote{O judeu e eu: não satisfeito em me racializar, por um
  acaso feliz eu me humanizava. Unia-me ao judeu, meu irmão de
  infortúnio. Uma vergonha! À primeira vista, pode parecer surpreendente
  que a atitude do antisemita se assemelhe à do negrófobo. Foi meu
  professor de filosofia, de origem antilhana, quem um dia me chamou a
  atenção: ``Quando você ouvir falar mal dos judeus, preste bem atenção,
  estão falando de você''. E eu pensei que ele tinha universalmente
  razão, querendo com isso dizer que eu era responsável, de corpo e
  alma, pela sorte reservada a meu irmão. Depois compreendi que ele quis
  simplesmente dizer: um antisemita é seguramente um negrófobo (Idem, p.
  112, \emph{grifo meu})}. Isso não se deve ao fato de que uma de suas
fontes inspiradoras para pensar a condição do negro seja Sartre em
\emph{Réflexions sur la question juive,} mas sim pela necessidade de
demonstrar que o racismo se produz de diversas formas e mesmo se
dedicando ao saber, se a estrutura sociossimbólica do capital em seu
processo permanecer inalterada, os horizontes de mudanças estarão
fechados.

\emph{b}) Fracassado em sua busca pela razão como forma de emancipação,
eis que surge o elemento da sensibilidade: ``O sacrifício tinha servido
de meio termo entre mim e a criação -- não encontrei mais as origens,
mas a \emph{Origem}. No entanto, era preciso desconfiar do ritmo, da
amizade Terra-Mãe, deste casamento místico, carnal, do grupo com o
cosmos''\footnote{Idem, p. 115}. Imersa na poesia de num retorno místico
às origens, essa consciência que se viu objetificada pelo colonialismo
busca uma saída no elemento mítico.

Ambas posições fracassam, mas, são constitutivas dessas consciências
como \emph{ilusões necessárias}.

Felizmente, porém, a consciência segue seu curso e, diante desse louvor
de uma união mítica e sensível com a mãe-terra, desconfiada, repõe em
curso a dúvida. Este é o movimento que Fanon faz deixando para trás as
crenças que invadem hoje grandes setores do movimento negro. Essa união
mística é só misticismo:

\begin{quote}
Fiz caminhadas até os limites de minha essência; eles eram, sem dúvida
alguma, estreitos. Foi então que fiz a mais extraordinária das
descobertas, aliás, propriamente falando, uma redescoberta. Revirei
vertiginosamente a antiguidade negra. O que descobri me deixou ofegante.
No seu livro \emph{L'abolition de l'esclavage}, Schoelcher nos trouxe
argumentos peremptórios. Em seguida Frobenius, Westermann, Delafosse,
todos brancos, falaram em coro de Ségou, Djenné, cidades de mais de cem
mil habitantes. Falaram dos doutores negros (doutores em teologia que
iam a Meca discutir o Alcorão). Tudo isto exumado, disposto, vísceras ao
vento, permitiu-me reencontrar uma categoria histórica válida. O branco
estava enganado, eu não era um primitivo, nem tampouco um meio-homem, eu
pertencia a uma raça que há dois mil anos já trabalhava o ouro e a
prata\footnote{Idem, p. 119}.
\end{quote}

O que Fanon aponta com grande lucidez é que, nesse percurso da
consciência, a identidade está sempre se desestruturando e repondo seu
movimento. Da tentativa de agarrar a razão contra o irracionalismo até a
tentativa de se agarrar a sensibilidade poética que estrutura uma
espécie de retorno às raízes culminando numa redescoberta dessas
próprias origens, o que está apontado é a passagem por uma espécie de
\emph{Grande Outro}, para citar Lacan, que articula a experiência que
essa consciência faz. A identidade é um momento necessário porque
evanescente.

\chapter{Sartre e a dialética espanada}

É nessa espécie de paciência frente aos fracassos que Fanon arma seu
arsenal crítico disposto a demonstrar os problemas que surgem ao não se
demorar nesses sintomas impostos por uma realidade totalmente
patológica. O modo como expressa essa relação desafortunada de um
significante vazio, evanescente e contraditório, de uma subjetividade
destroçada, molda a visão de mundo que o negro forma a partir de sua
relação com o próprio mundo.

O que a maioria daqueles que seguem Fanon não perceberam é que nesse
ponto ele está fazendo um exercício legado pelo pensamento especulativo.
Cada passo de sua formação são figuras neuróticas da consciência que
tentam se firmar naquilo que acreditam ser a verdade de si mesmas. No
entanto, as contradições e incertezas são o resultado dessa experiência
que passa de uma figura a outra sem poder se firmar mediante as
contradições emergentes de sua relação com o mundo colonizado. Aí, tudo
que é sólido se dissolve no ar.

Quando Sartre entra em cena, e aqui podemos pensá-lo como mais uma
figura da consciência, ao tentar desbaratar o jogo utilizando o
pensamento especulativo para, por fim, tornar clara as limitações do
negro, o que faz é somente mais um exercício que dá com os burros
n'água. Sartre torna-se uma figura da consciência de Fanon. Comentarei
mais detalhadamente essa cena imperdível:

\begin{quote}
Mas a coisa pode ser mais séria ainda: o negro, nós o dissemos, cria
para si um racismo antirracista. Ele não deseja de modo algum dominar o
mundo: ele quer a abolição dos privilégios étnicos, quaisquer que sejam
eles; ele afirma sua solidariedade com os oprimidos de qualquer cor. De
repente a noção subjetiva, existencial, étnica da negritude ``passa'',
como diz Hegel, para aquela -- objetiva, positiva, exata -- do
proletariado. ``Para Césaire, diz Senghor, o `branco' simboliza o
capital, como o negro o trabalho... É a luta do proletariado mundial que
canta através dos homens de pele negra de sua raça''. É mais fácil
dizer, menos fácil pensar. Não é por acaso que os mais ardentes vates da
negritude são, ao mesmo tempo, militantes marxistas. Mas isso não impede
que a noção de raça não se confunda com a noção de classe: aquela é
concreta e particular, esta universal e abstrata; uma vem do que Jaspers
chama de compreensão, e a outra, da intelecção; a primeira é o produto
de um sincretismo psicobiológico e a outra é uma construção metódica, a
partir da experiência. De fato, a negritude aparece como o tempo fraco
de uma progressão dialética: a afirmação teórica e prática da supremacia
do branco é a tese; a posição da negritude como valor antitético é o
momento da negatividade. Mas este momento negativo não é autosuficiente,
e os negros que o utilizam o sabem bem; sabem que ele visa a preparação
da síntese ou a realização do humano em uma sociedade sem raças. Assim,
a negritude existe para se destruir; é passagem e ponto de chegada, meio
e não fim último\footnote{FANON, 2008, p. 121 apud. Jean-Paul Sartre,
  ``\emph{Orphée noir}'', prefácio à \emph{Anthologie de la poésie nègre
  et malgache}, pp. XL e sqq.}.
\end{quote}

Se há uma posição relativa na ação dos negros, ela obedece ao percurso
não determinado, na verdade totalmente contingente, da realidade
histórica. É nisso que a passagem feita por Sartre de raça para classe
se revela rápida demais por ser uma posição esquemática que suprime de
si as particularidades constitutivas da compreensão sobre a raça.

A noção algo esquemática de uma progressão dialética como finalidade
indiscutível é aquilo que espana a própria apreensão de dialética
sartriana. Sartre se traí. A força das análises de Fanon reside no fato
de que os extremos (negro/branco) permanecem produzindo a cisão e o
único avanço obtido é a compreensão do interior dessa lacuna.

Não é simplesmente identificar mecanicamente o branco com o capital --
ainda que seja uma verdade factível -- mas, compreender que essa cisão é
uma abstração (metafisica) que dinamiza a realidade. O verdadeiro
significado está no vazio de seu conteúdo cujo sentido é gerado na
medida em que o movimento se revela. Daí a necessidade dos exercícios
que Fanon nos legou. Exercícios que parecem não ter tido
precedentes\footnote{Mesmo Mbembe não vai até as últimas consequências
  das lições deixadas por Fanon.}.

A \emph{identidade de opostos} nada tem em comum com a ideia de uma
resolução imposta que eleva a figura da consciência para um estágio
superior -- do tipo raça para classe --, pelo contrário, a luta se firma
no evanescer da experiência que formula uma nova negatividade encarnada
numa figura singular e, portanto, numa nova experiência que reescreve a
passada. E é por isso que Fanon continua ``este hegeliano-nato esqueceu
de que a consciência tem necessidade de se perder na noite do absoluto,
única condição para chegar à consciência de si''.

Se perder na noite do absoluto indica que cada estágio sendo necessário
é inútil em si mesmo. Só assim podemos chegar a consciência de si.

É aí que Sartre se traí de novo:

\begin{quote}
Pouco importa: a cada época, sua poesia; a cada época as circunstâncias
da história elegem uma nação, uma raça, uma classe para reacender a
chama, criando situações que só podem ser representadas ou superadas
pela poesia; ora o impulso poético coincide com o impulso
revolucionário, ora diverge. Saudemos, hoje, a oportunidade histórica
que permite aos negros dar ``com tal determinação o grande grito negro
que abalará os assentamentos do mundo''\footnote{FANON, 2008, p. 121
  apud. Sartre, idem, p. XLIV}.
\end{quote}

Nessa posição sintomática de um devir sem contingência, o espaço para a
liberdade é solapado. Foi só a história que produziu a poesia não os
homens? Com razão Fanon diz: ``pronto, não foi eu quem criou um sentido
para mim, este sentido (segundo Sartre) já estava lá esperando-me''
(FANON, 2008, p. 121) e mais abaixo retruca: ``contra o devir histórico,
deveríamos opor a imprevisibilidade''.

Ora, o que Sartre exclui de sua formulação é que se, por um lado, nada é
sabido que não esteja na experiência, por outro, o devir na história é
marcado por uma finalidade cuja contingencia lhe é constitutiva. É
factível a impossibilidade de se apropriar do futuro tendo em vista que
não podemos enquadrar a tessitura da história.

A negatividade se alimenta da luta por essa apropriação como uma
tentativa impulsionada pela consciência. Por isso, que contra esse tipo
de redução Fanon diz \emph{não}!

``Eu tinha necessidade de me perder absolutamente na negritude. Talvez
um dia, no seio desse romantismo doloroso...'' (ibidem) A abertura da
contínua atividade da consciência em sua busca de se apropriar do futuro
está aprisionada na retroversão, na qual: só a exposição completa e
objetiva da experiência demonstra seus percalços e suas ilusões
necessárias.

É precisamente nesse passo que a questão da liberdade se efetiva; a
liberdade, com relação as determinações pressupostas, só pode ser
efetiva contra esse pano de fundo. Não se pode prever as consequências
das nossas ações tendo em vista que, se assim procedesse, a liberdade se
reduziria a necessidade. Perderíamos a retroatividade que constitui
nossa experiência e mantém a abertura para a contingência radical.

Na luta contra essa contingência, se ergue o sujeito negativo, o negro,
no exercício de sua liberdade mantendo de pé a abertura ao futuro.

\begin{quote}
``A dialética que introduz a necessidade de um ponto de apoio para a
minha liberdade expulsa-me de mim próprio. Ela rompe minha posição
irrefletida. Sempre em termos de consciência, a consciência negra é
imanente a si própria. Não sou uma potencialidade de algo, sou
plenamente o que sou. Não tenho de recorrer ao universal. No meu peito
nenhuma probabilidade tem lugar. Minha consciência negra não se assume
como a falta de algo. Ela é. Ela é aderente a si própria\footnote{Idem,
  p. 122.}''
\end{quote}

Aqui realmente não se sabe se Fanon está ironizando ou demarcando uma
posição contra a postura dialética que encerra um trauma à consciência
negra. O fato é que diante dessa imposição ela já não pode ficar
impassível, ainda que tenha razão. Essa resistência da consciência
negra, que imediatamente \emph{é,} já é em si mesma a possibilidade do
movimento dialético. Tentarei explicar o porquê:

Essa individualidade da consciência entendida como negra, enquanto o
\emph{Um}, já está desdobrada em si mesma, quer dizer, é como se a
individualidade aqui estivesse elencada ao universal imediatamente, se
limitasse e se reconhecesse nele. Logo, o Eu=Eu da consciência-de-si
está posto à prova de saída porque sua identidade jaz ligada ao todo
concreto. É impossível escapar dele uma vez que essa consciência é ser
social. Como nos fez entender Fichte lá atrás entre o Eu=Eu da
consciência há uma infinidade de determinações.\footnote{FICHTE, J.
  \emph{A Doutrina da Ciência de 1794}, tradução de Rubens Rodrigues
  Torres Filho, pp. 35-176 in: \_\_\_\_\_\_. \emph{Os Pensadores}. São
  Paulo: Abril Cultural, 1984.}

A ação externa do mundo branco abala a calma organização desse Eu e seu
tranquilo movimento. O que aparece como ordem e harmonia de si para
consigo torna-se, através dessa ação exterior, uma transição de opostos,
em que cada qual se mostra como anulação de si mesmo.

Essa ``anulação de si mesmo'' pressupõe um corte radical imposto pela
oposição de duas tendências no interior de um mesmo plano simbólico
(branco/negro).

Podemos intuir daí que há dois universais abstratos que nascem e
precisam morrer juntos: 1) a perda gerada a partir do movimento imposto
pela oposição (feita pelo branco) é 2) o reconhecimento dessa própria
perda. Desse ponto de vista, não é possível nenhum acordo, ambas
posições são irredutíveis e, portanto, o conhecimento dessa resistência
antagônica é a condição de possibilidade da ação em si, isto é, de
implodir essa limitação simbólica.

Por isso, a posição irredutível defendida por Fanon é aquela capaz de
reunir os cacos quebrados do que se tem por negritude e a partir dela
implodir o mundo onde essa negritude foi concebida como
diferença/exclusão. Sartre com sua dialética espanada não percebeu que é
a posição irredutível dessa particularidade que abala os fundamentos
simbólicos do mundo branco.

Passar tão logo às determinações universalmente abstratas de classe é
não se dar conta das determinações raciais e sua fundamentação
determinada igualmente pela exploração do capital. É sua especificidade,
sua singularidade determinada e irredutível, aquilo que é capaz de
implodir o modo de sociabilidade baseado na exploração e opressão.

Infelizmente ainda hoje isso é motivo de discussão nos recantos mais
vulgares do marxismo. Contra ela afirmamos: a universalidade
hipostasiada da classe requer o seu negativo, isto é, a particularidade
que a compõe e a estrutura. Necessariamente, a consciência de classe é
dependente da particularidade e da especificidade dos seus componentes,
não o contrário. Para citar ironicamente Sartre: ``a existência precede
a essência''\footnote{Nesse sentido a posição de Fanon é mais sartriana
  que a do próprio Sartre como vemos nesse excerto: Quanto a nós,
  queremos constituir precisamente o reino humano como um conjunto de
  valores distintos do reino material. Mas a subjetividade que nós aí
  atingimos a título de verdade não é uma subjetividade rigorosamente
  individual, porque demonstramos que no cogito nós não descobrimos só a
  nós, mas também aos outros. (Cf. SARTE, J-P. O existencialismo é um
  humanismo, tradução Vergílio Ferreira, p.21 in: \_\_\_\_\_\_\_. Os
  Pensadores. São Paulo: Abril Cultura, 1973.)}.

\chapter{A radicalidade do pensamento de Fanon}

De novo o misticismo às cegas, sem responsabilidade ou quase isso.
Depois de mais de sessenta anos desde que Fanon nos legou sua obra,
estamos às voltas com velhos problemas: neorracismo\footnote{Aqui
  recorro a noção de Etienne Balibar segundo o qual: ``O racismo,
  verdadeiro fenômeno social totalizador, se inscreve em práticas
  (formas de violência, desprezo, intolerância, humilhação, exploração),
  discursos e representações que são outros tantos desenvolvimentos
  intelectuais do fantasma da profilaxia ou da segregação (necessidade
  de purificar o corpo social, de preservar a identidade do ``eu'', do
  ``nós'', mediante a qualquer perspectiva de promiscuidade, de
  mestiçagem, de invasão), e que se articulam em torno de estigmas da
  alteridade (apelido, cor da pele, práticas religiosas).'' Cf, BALIBAR
  \& WALLERSTEIN. \emph{Race Natión Classe: les Identités Ambigúes}. La
  Découverte: París, 1988 p.31} e busca pelo retorno às origens, o que
naturalmente são faces de uma mesma moeda.

Por meio de muita arrogância, pela primeira vez na história, operações
de polícia introjetada na psique dos potenciais descontentes se realizam
na busca de condenar qualquer voz dissidente ao estabelecido. De 1952 a
2018 o capital se transformou, se amoldou. Cada crise serviu para novo
impulso. O negativo foi sua base de sustentação.\footnote{Retirei essa
  ideia do importante livro de Grespan (Cf. GRESPAN, J. \emph{O negativo
  do capital: O conceito de crise na política de Marx.} São Paulo:
  Hucitec, 1998)}

E a voz cínica já se ergue: ``Fanon nada tem mais a dizer, precisamos
nos ater as novas epistemologias''. Novas epistemologias? Defender a
filosofia Banto sem o mundo Banto? Trata-se disso. Um escândalo, uma
regressão!

Ainda estamos aqui e ainda estamos vivos.

O mundo que acreditava ter esvaziado de significação seu entorno
descobre de repente o câncer em suas entranhas. Sob o signo da
catástrofe social, num horizonte francamente regressivo em que ``o tempo
do fim (da História) é antes de tudo um (novo) tempo de
guerra''\footnote{ARANTES, op. cit., p.63.}, de repente, ouve-se estalos
de chicote nas costas de centenas de pretos na Líbia em plena era
informatizada.

Um exército de defensores do único mundo possível se apresenta com os
seus comunicados midiáticos de uma vitória permanente. O ópio do consumo
paralisou a esquerda brasileira e as disputas intestinais reduzem-se a
luta pela gestão da barbárie. A morte de famílias inteiras naufragadas
em uma balsa no Mediterrâneo juntamente com crianças sendo revistadas
por soldados no Rio de Janeiro apresentam o coroamento da civilização.

Nunca houve tanta violência diária e, no entanto, nunca houve tanta
apatia. Com o retorno das desigualdades aos índices da era balzaquiana o
mundo torna-se cada vez mais negro!

\begin{quote}
Transferência maciças de fortunas para interesses privados,
desapossamento de uma parte crescente das riquezas que lutas anteriores
tinham arrancado ao capital, pagamento indefinido de dívida acumulada, a
violência do capital aflige agora, inclusive, a própria Europa, onde vem
surgindo uma nova classe de homens e de mulheres estruturalmente
endividados {[}...{]} Mais característica ainda da potencial fusão do
capitalismo e do animismo é a possibilidade, muito distinta, de
transformação dos seres humanos em coisas animadas, em dados digitais e
em códigos. Pela primeira vez na história humana, o nome Negro deixa de
remeter unicamente para a condição atribuída aos genes de origem
africana durante o primeiro capitalismo {[}...{]} A este novo carácter
descartável e solúvel, à sua institucionalização enquanto padrão de vida
e à sua generalização ao mundo inteiro, chamamos o \emph{devir-negro do
mundo}\footnote{MBEMBE, op. cit., p.18}.
\end{quote}

Se, isso não é uma vantagem miraculosa -- como cinicamente meu ex-mentor
sugeriu numa entrevista\footnote{Naturalmente trata-se de uma ironia com
  Frei David Santos, fundador da Educafro que até mais ou menos 2005 era
  de esquerda, momento em que ainda muito jovem militava em suas
  fileiras, após ser absorvida pelo PT a Educafro abandonou sua postura
  de transformação social e tornou-se ONG. Segundo o Frei ser negro
  agora é ter vantagem... (``Concurso da prefeitura de SP verifica cor
  da pele de cotistas aprovados'', Folha de São Paulo, 26/06/2017)} --
pode ser o estopim de uma nova forma de sociabilidade. Quando um ou
outro estudante buscou refletir sobre Fanon, então houve indícios de que
seu pensamento sobreviveu aos rebaixamentos, desvirtuações e
manipulações cínicas e nada ingênuas.

Quando uma parte do movimento negro renega Fanon, isso só demonstra sua
atualidade. E se esta discussão não é somente para apresentá-lo serve ao
menos para lhe fazer, ainda que modestamente, justiça. Por que Fanon?
Por que agora? \footnote{Estou me referindo a brilhante tese de
  FAUSTINO, D. M. ``Por que Fanon, por que agora?'': Frantz Fanon e os
  fanonismos no Brasil. 2015. 252 f. Tese (Doutorado) -- Programa de
  Pós-Graduação em Sociologia, Universidade Federal de São Carlos, São
  Carlos, 2015.} São questões que ressoam no solo de um mundo que se
ergue sob o signo da catástrofe da escravidão moderna e num país cuja
estrutura escravista está entranhada nas instituições liberais e, pior,
na formação psíquica dos indivíduos. Nunca tivemos uma democracia
racial, é fato, mas, tivemos uma ``democracia racista''.

Os místicos são só a cara da coroa de uma mesma moeda no cofre do
rentabilismo. Da Martinica às ilhas Salomão, não há um modo de
sociabilidade que não esteja sob domínio do Império. Enquanto na Europa
grupos identitários voltam a reivindicar suas origens arianas, no Brasil
grupos débeis paulistas querem se separar do resto do pais. Esses
sintomas, porém, não bastaram para demonstrar para alguns setores do
movimento negro a loucura patológica de reivindicar às origens... ou
como disse Fanon, a Origem...

Olhados deste lado do atlântico com sessenta anos de desvantagens frente
a formulação de Fanon, poderia me perguntar: as figuras da consciência
não avançaram? Permanecemos imobilizados? O desmoronamento do ``bloco
socialista'' não deveria ser entendido como a inelutabilidade do próprio
processo de desmoronamento do sistema fundado pela escravidão moderna? O
retorno folclórico das caricaturas históricas não deu em farsas senão em
comédias intragáveis.

A crise tornou-se forma de governo.

Se a filosofia banta não conhece a miséria metafísica da Europa, a
miséria metafisica da Europa impôs seu mundo. É isso que Fanon quer
demonstrar ao atacar as formas místicas e reacionárias de tentar reviver
aquilo que foi morto pela máquina. Se a existência dos bantos se situa
no plano do não-ser é justamente porque sua sociedade é uma sociedade
fechada e ainda não tinha conhecido a violência da história.

Violência que se impôs a ferro nos calcanhares e fogo nos peitos. Não é
em vão que antes de comentar a filosofia banta, Fanon cite um longo
trecho do desgraçado mundo do \emph{apartheid}. Essa loucura de atingir
uma universalidade mítica e imediata dada por formas de vida que
desapareceram com o choque colonial é no mínimo cínica quando não formas
de mercado sob a insígnia de afroempreendedorismo.

Alioune Diop como representante máximo desse tipo de posicionamento que
tende ao universal sem mediação, numa busca regressiva pelas origens, é
ironizado: ``O preto se universaliza, mas do Liceu Saint-Louis, em
Paris, um deles foi expulso: teve a ousadia de ler Engels'', e continua
``Já adivinhamos Alioune Diop a perguntar-se qual será a posição do
gênio negro no concerto universal. Ora, afirmamos que uma verdadeira
cultura não pode nascer nas condições atuais''.

E quais condições são essas? De 1952 a 2018 eles dirão; muita coisa
mudou. Nós diremos; muitas coisas mudaram, mas a exploração, e sua
consubstancial opressão, continua em escala ainda pior.

As novas coordenadas ideológicas efetivadas pelo ruir da modernidade a
partir dos anos 1970 são determinadas por dois pressupostos que
arrasaram quarteirões: por um lado, os direitos e valores tornaram-se
historicamente particulares, não podem ascender à universalidade; por
outro, há a suspeita universalizada que destitui qualquer noção mínima
de corpo político senão aquela já estruturada pelo jogo eleitoral,
qualquer noção que esteja para além da ordem do dia é atacada como
ilusória e oportunista.

A loucura da busca pela identidade hipostasiada só indica que o mundo do
trabalho ruiu.

A pergunta: Por que Fanon? Por que agora? Talvez, tenha nisso sua
resposta. Já sabemos que tais pressupostos são antagônicos a formulação
de Fanon.

O mecanismo fundado pela ideologia em tempos de capitalismo
financeirizado e altamente manipulatório não se baseia mais no
engajamento do indivíduo como sujeito capaz de alterar as coordenadas
pressupostas do esquema. Ironicamente é como se todos já estivessem
naquela universalidade do não-ser banto.

O liberalismo em tempos de financeirização propõe uma espécie de
neutralidade ao direito que escapa daquela determinação social, como a
economia estamos agora na esfera de um direito livre -- isto é, sem a
imposição da população -- que pode efetivar uma ordem política desejada
sem a necessidade de sujeitos políticos.\footnote{Podemos concluir que o
  componente jurídico que se seguiu ao golpe no Brasil se serve dessa
  noção.}

Logo, as soluções certas são reconhecidas pelo fato de que não precisam
ser escolhidas\footnote{A esse respeito ver RANCIÈRE, J. \emph{O Ódio à
  Democracia}. São Paulo: Boitempo, 2014}. Nada melhor que um técnico
para tirar as dúvidas; um governo dos mais capazes. Governar sem povo,
porque o próprio povo se tornou não apenas indiferente senão inútil para
o estabelecimento das vias do sistema, parece ser uma prerrogativa
acertada, pelo menos para a elite econômica. Aliás tanto nos EUA quanto
no Brasil está verdade azeda o estômago.

Toda a questão das lutas é reduzida para a esfera da visibilidade e da
representatividade que tem seus lastros na própria forma de uma
democracia golpeada em época de um Eu-empresa que impõe a concorrência
onde não há. A identidade sem relação com o outro é a bola da vez,
enriqueça-a e venda-a como produto por meio de um volumoso currículo de
ações solidárias. Quer dizer, aceita-se de antemão a derrota para logo
em seguida transformá-la em triunfo mercadológico.

Ora, acima eu havia chamado atenção para a noção de identificação,
agora, ela retorna em sua forma sintomática para refletirmos sobre os
dias atuais. Por um lado, o mercado aposta na identificação dos grupos;
não precisa existir democracia se nossa identificação for guiada por
líderes e técnicos capazes de fornecer o melhor para nós. Por outro, a
tentativa de questionar tais pressupostos, ainda que tenha razão, de
acordo com a ideologia dominante sempre acaba em assassinato, maior
opressão e desequilíbrio social que pode pôr tudo a perder. Por fim, o
recado é claro: a transformação social internacional é uma utopia de
assassinos sedentos pelo poder.

A ideologia atual deixa evidente que as formas políticas capturadas pelo
mercado são só uma aliança oligárquica entre ciência e riqueza que exige
todo o poder. Os discursos que se voltam para os particularismos,
inclusive em toda sua caricatura, (quem não assistiu as propagandas
políticas de Hillary Clinton?) retomam o velho princípio da filiação em
uma comunidade enraizada no sangue, na cor da pele, na
religião\footnote{Com esse diagnóstico da ideologia atual não fica
  difícil entender porque grandes setores da esquerda progressista
  nacional receberam de braços abertos um filme terrível e reacionário
  como \emph{Pantera Negra}. O filme apresenta claramente qual seria o
  projeto que poderia fomentar um desenvolvimento técnico e econômico
  sem precedentes -- segundo a visão imperialistas é claro: um Estado
  sem intervenção da democracia. Se o Negro até hoje viveu às margens da
  sociedade, oferta-se a ele a adesão ao consenso eterno que repudia os
  conflitos antigos e dobra-se às soluções dos especialistas que só
  podem discuti-las com os representantes escolhidos pelos deuses que
  compõe a oligarquia.} e no respeito a todas desde que elas não se
misturem...

Atomizar as comunidades e indivíduos, apelar para a característica
particular, gerar identificação são as premissas básicas do controle
social exercido na \emph{era da emergência.} Posso ter contato com
outros grupos, mas sem estabelecer com eles relações, eis o pressuposto
posto do controle atual.

É como se estivéssemos na prisão do seriado de \emph{Orange is the New
Black}\footnote{Seriado famoso que teve a lucidez de demonstrar em toda
  sua força a incapacitante noção de identidades estanques e
  não-relacionais no interior da prisão. Todas as formas de controle
  social e gestão da miséria ficam estampadas em sua narrativa sob um
  único atributo: defender as comunidades negra, latina, branca e
  religiosa sem deixar que elas se relacionem. O recado é bem claro:
  estamos todos numa prisão sendo conduzidos por gestores da miséria.
  Falar sobre este seriado, contudo, equivaleria um capítulo à parte que
  fugiria de nosso tema.} cujos gestores fossem nossos governantes e
cada um tivesse sua comunidade própria e não se misturasse com as
demais.

Naturalmente numa sociedade forçosamente miscigenada como a nossa, tais
imposições do mercado imperialista entrariam em curto-circuito. Espalhar
essas ideias por aqui tem encontrado um terreno insólito cujo adubo é
paradoxalmente fornecido pela esquerda e sua classe-média que dentro de
seus confortáveis apartamentos tornam o \emph{tour} pela favela algo
exótico.

Mas, a lucidez ainda brada: ``o problema negro não se limita ao dos
negros que vivem entre os brancos, mas sim ao dos negros explorados,
escravizados, humilhados por uma sociedade capitalista, colonialista,
apenas acidentalmente branca''\footnote{FANON, \emph{op.cit}., p.170}.

Ao retirar o essencialismo e a cristalização categorial, Fanon permitiu
pensarmos para além dos limites pressupostos pelo jogo. A radicalidade
de seu pensamento ecoa ainda hoje com a lucidez que golpeia o misticismo
e a obscuridade que, infelizmente, grassou em grande parte da
esquerda...

\part{O MOVIMENTO NEGRO E O MISTICISMO}

\begin{quote}
Tal como não deve haver ``capitalismo branco'', ``capitalismo amarelo''
e por aí vai. O capitalismo é sempre algo sem raça e ainda antirraça. As
situações de vantagem ou desvantagem de uma ou outra raça no sistema
capitalista de países específicos decorre de contextos históricos. Razão
pela qual, para extinguir os males advindos do capitalismo, não adianta
querer ``identitarizar'' o capitalismo. (\emph{Bobby Seale})
\end{quote}

\chapter{Os limites do problema}

Este pequeno ensaio não trata de uma pesquisa histórica tampouco
sociológica. O esforço aqui é o de esboçar uma história do
desenvolvimento das ideias para, com isso, desmistificar posições
teóricas que se tornaram força material e impregnaram as ações de grande
parte do \emph{Movimento Negro}.

Mas, repetindo uma pergunta feita lá atrás por Lélia Gonzalez: será
possível falar do Movimento Negro?\footnote{GONZALEZ \& HASENBALG.
  \emph{Lugar de Negro}. Rio de Janeiro: Marco Zero, 1982.} Sabemos que
o Movimento Negro é só uma abstração para indicar lutas que se baseiam
na compreensão da estrutura racial do país e nas formas ou de minimizar
tais resultados -- grande parte dos setores -- ou de superá-la --
posições minoritárias. Ele é múltiplo e mais dinâmico que nossas
categorizações.

A posição hegemônica, aquela de minimização dos resultados catastróficos
de uma estrutura econômica-social racializada, contribuiu para o
desenvolvimento das lutas em diversas frentes obtendo, não neguemos,
alguns êxitos contra os resultados devastadores e assassinos do racismo
brasileiro. Daí que a história do MNU (\emph{Movimento Negro Unificado})
é necessariamente o filtro por onde as ideias hegemônicas passaram.

A despeito dos avanços obtidos, principalmente no que se refere ao
debate sobre o racismo, entretanto, a violência e o assassinato das
pessoas de cor aumentaram nas periferias. No exato momento em que
grandes setores do movimento negro foram cooptados para a máquina
burocrática do Estado, sob a égide do PT, houve o aumento exponencial do
número de assassinatos de jovens no Brasil.\footnote{Cf. \emph{Atlas da
  Violência} 2017}

A região nordeste, abandonada pelos olhares clínicos dos especialistas,
vive hoje em estado de guerra permanente. Enquanto na região norte num
único presídio tivemos a morte de 56 presos em condições nas quais os
``bárbaros'' se envergonhariam.\footnote{In: \emph{Matança em presídio
  de Manaus é uma das maiores desde Carandiru}, G1, 02/01/2017}

Os números de assassinatos compreendidos entre 2005 a 2015 tiveram um
aumento de 17,2\% entre indivíduos de 15 a 29 anos. Com 59 mil
assassinatos por ano dos quais a maioria recai nas costas da população
negra. Vivemos uma catástrofe social difícil de encontrar parâmetros no
cenário mundial.

Essa dolorosa realidade solapou a ideia, veementemente combatida pelo
MNU, de \emph{democracia racial}. Hoje graças ao empenho de seus atores
políticos e a gritante realidade racialmente cindida, qualquer pessoa
com razoável coeficiente cognitivo não leva a sério essa noção.

Também os resultados da era lulista não foram tão bons para aqueles que
vivem nas periferias desse país. Com todo o ódio de classe jogado nas
nossas costas da população negra, não me parece mais uma opção apostar
no jogo político posto nos limites da representatividade. Se
fracassamos, essa crítica é uma tentativa de fracassar melhor.

No olhar retrospectivo se comprova que toda aquela tentativa de
hegemonia à esquerda recaiu em regressão; uma violência desmesurada
impregnada pelas formas de administração e controle dos corpos impresso
pela economia de mercado. Uma verdadeira calamidade social que incidiu
numa necropolítica violenta e da qual o Estado de exceção é só uma
normalidade histórica e colonial.

Como diz Achille Mbembe num diagnóstico preciso dos novos desdobramentos
de organização social sob égide do capitalismo contemporâneo: ``os novos
processos de racialização visam marcar esses grupos de populações, fixar
do modo mais preciso possível os limites no seio dos quais elas podem
circular, determinar do modo mais exato possível os espaços que elas
podem ocupar, em suma, assegurar as circulações num sentido que permita
afastar as ameaças e assegurar a segurança geral. Trata-se de selecionar
esses grupos de populações, de os marcar a um tempo como ``espécies'',
``séries'' e como ``casos'', no seio de um cálculo generalizado do
risco''.\footnote{Achille Mbembe em \emph{A universalidade de Frantz
  Fanon}.}

A invasão das favelas com tanques do exército só confirma a tese.

Ora, se só no fim se compreende o começo, nossa aposta é precisamente
demonstrar os pontos de suturas e cordialidade com o \emph{status quo}
que fora levado adiante por grande parte da esquerda. Penso que a
crítica radical à hegemonia identitária, em sua limitação a esfera de
visibilidade representativa no interior do atual sistema, é mais que
central.

Não se pode esquecer, contudo, que a necessidade da identidade é na
verdade a resposta política a um sistema cuja universalidade é
excludente. Tenhamos calma e não caiamos no engodo de reduzir tudo a
\emph{abstração-real} chamada classe. Para evitar isso, Fanon e seu
pensamento especulativo, como ficou evidente linhas acima, nos trouxe
grandes lições.

A compreensão da necessidade do particular demonstra que é a própria
incompletude que fomenta a totalidade. Dizíamos acima que a existência
precede a essência, nada mudou.

O \emph{Outro} contraditório é o que nos constitui a partir do momento
que expressa uma consciência negativa ao nosso Eu=Eu. Nossa identidade,
momento no qual a experiência fundamenta uma subjetividade, depende do
elemento externo negativo que necessita de uma outra
consciência\footnote{Temos uma interpretação desse duelo entre
  consciências que foi sinalizado por Cossetin quando diz que: ``A
  ênfase na corporeidade marca o encontro das consciências-de-si e tem
  como principal objetivo o desejo de reconhecimento cuja origem está
  num confronto que expressará a ascensão da consciência sobre a sua
  existência corporal objetiva e meramente natural. Por tal razão é que
  o estremecimento corpóreo será a experiência do modo e da sensação da
  consciência-de-si para si mesma, a experiência da diferença consigo
  mesma e também o meio imediato pelo qual ela se manifesta para outros,
  forma de sua objetividade. Para que o reconhecimento efetivamente
  ocorra, contudo, é necessário que ambas as consciências se deem conta
  de que se agir e existência dependem uma da outra'' (Cf. COSSETIN, V.
  L. F\emph{. A dissonância do Absoluto: linguagem e conceito em Hegel}.
  Ijuí: Editora Unijuí, 2012 p. 101). Tal interpretação destaca um
  movimento interessante a respeito da questão dos corpos, na luta entre
  Senhor e Escravo, no entanto, perde de vista a coisidade e o trabalho
  como coerção e, ao mesmo tempo, realização da consciência. Mais
  profunda nessa questão, a postura de Judith Butler ao refletir sobre a
  questão dos corpos, sem perder de vista a categoria de trabalho,
  permite-lhe enxergar a questão da subordinação ilimitada empreendida
  pelo ato de trabalhar e pela falta do ato declamatório, justificando
  assim a fuga do estoicismo etc.}. Ela sempre é mutável.

Por esse motivo, o hipotético-leitor já deve ter se dado conta que a
apreciação de Fanon, que fizemos na primeira parte, serve como fio
condutor dessa crítica. Sendo Fanon nosso Vírgilio, não tenhamos medo de
adentrar as orlas daquilo que concebemos como \emph{Movimento Negro}
hegemônico.

Não farei aqui, como disse, uma história sobre esse ``movimento'', me
limitarei somente a demonstrar como os componentes teóricos fermentado
por seus intelectuais foi aquilo que nos guiou até essa sinuca de bico
em que, por um lado, nos digladiamos pelas misérias ofertadas pelo
poder, enquanto, por outro, massacres perpetrados pelo Estado ocorrem
impunimente e somos incapazes de agir.

\chapter{Contra o misticismo do trabalho libertador}

Um dos diagnósticos que já podemos fazer de saída é que grande parte da
tradição marxista ortodoxa apostou numa evanescência total do componente
racial à estrutura produtiva.

O construto teleológico concebido como progresso, a noção de um
desenvolvimento histórico em linha ascendente e a aposta no trabalho
como suposta libertação incapacitaram uma crítica mais fecunda e que
decifrasse a forma como o componente de \emph{desigualdade racial} fora
integrada ao modo de funcionamento do capital.

O engodo causado por esse tipo de noção é acreditar que há uma
progressão histórica necessária capaz de absorver as demandas produzidas
no seio da sociedade civil burguesa quando, pelo contrário, essas
produções de demandas são o próprio movimento no interior dessa
sociedade na forma de seu excedente perpétuo comandado pelo capital.

Isso significa que a estrutura do capital no seu elemento, para citar
velhos fantasmas, desigual e combinado, produz anomalias sociais que na
verdade fornecem ao modo de produção e reprodução social os componentes
necessários para manter o grau de valorização do capital intacto, sua
taxa de lucro e com isso a realização do processo total de circulação.

O processo pelo qual a produção e reprodução de mercadorias
historicamente determinadas pelo trabalho repõe seu contínuo movimento é
também aquele que fornece os componentes necessários para a realização
da circulação. A questão é que a própria mercadoria estabelece a forma
de relação social e não há por detrás dela nenhum conjunto de relações.

Historicamente, enquanto o negro era escravo, isto é, enquanto sua mão
de obra não se constituía como mercadoria, ele permanecia no interior de
uma relação de exploração total. A partir do momento em que o capital
encontrou sua força na forma de excedente produzido pelo valor com o
dispêndio da força de trabalho, então, o antigo escravo passou a ser
descartável.

Uma pequena mostra do trabalho de Florestan Fernandes diz um pouco desse
processo brutal em que a liberdade recém-conquistada se confundiu com
uma invisibilidade social radical. ``Eliminado para os setores residuais
daquele sistema, o negro ficou à margem do processo, retirando dele
proveitos personalizados, secundários e ocasionais''\footnote{FERNANDES,
  F. A integração do Negro na sociedade de classes: o legado da raça
  branca. São Paulo: Globo, 2008 p. 36}. Noutras palavras, a
possibilidade de integração do negro na sociedade foi solapada pelo
processo de concorrência com o branco.

O problema é que esse processo concorrencial no mercado de trabalho
jamais se esgotou, e a composição do próprio mercado fundamentou áreas
nas quais o artifício racial se tornou determinante. Não é difícil
perceber a discrepância racial até hoje existente em serviços como o
doméstico -- verdadeira herança escravagista -- e os setores da
construção civil, só para ficar em dois exemplos.

Voltando a balizar nossa crítica à história das ideias, é preciso
concluir daí que a famosa \emph{dialética do senhor} e \emph{escravo}
hegeliana, mal interpretada por grande parte dessa tradição, foi
positivada e seus resultados foram vistos com olhar otimista. O que
quero dizer com isso é que houve sempre uma posição de otimismo com os
resultados do desenvolvimento do trabalho por parte da vulgata marxista.

Foi só com Judith Butler que um outro modo de interpretação foi
construído a partir da noção de trabalho e sujeição\footnote{BUTLER, J.
  \emph{A vida psíquica do poder: teorias da sujeição}. Trad: Rogério
  Bettoni. Belo Horizonte: Autentica, 2017.}. Evidenciando de saída os
problemas encontrados na noção de progressão histórica e o \emph{status}
de um sujeito que resume em si a Universalidade, Butler retoma a noção
de \emph{Dominação e Escravidão} de um ponto de vista distópico que
muito tem a nos ensinar\footnote{Não se pode esquecer, contudo, de
  Althusser que pensa a noção de sujeição como interpelação por exemplo,
  algo que será determinante para crítica de Butler. (Cf: ALTHUSSER, L.
  \emph{Ideologia e Aparelhos Ideológicos de Estado}, p.105-142 in:
  ZIZEK, S. (org.) \emph{Um mapa da ideologia}. Rio de Janeiro:
  Contraponto, 1996).}.

Em primeiro lugar, invertido os sinais da equação senhor/escravo, como
resultado negativo não há uma saída luminosa para o escravo no interior
dessa dialética. ``O escravo surge como corpo instrumental cujo trabalho
provê as condições materiais da existência do senhor, e cujos produtos
materiais refletem tanto a subordinação do escravo quanto a dominação do
senhor''\footnote{BUTLER, op. cit., p.43}.

Em segundo lugar, a própria noção de corpo instrumental tem um
significado importante para nós: o negro reduzido a cor da pele, e
desumanizado em sua humanidade, sofre no próprio corpo os resultados da
exploração imposta pelo senhor. Aliás, ele se reduz ao corpo (não nos
esqueçamos da profunda discussão de Fanon sobre a genitália).

Num primeiro momento o escravo é posto como mero instrumento e,
portanto, reflexo do senhor. É como se na relação colonizada estivesse
preso a descorporiedade do branco. O branco é um desejo sem corpo que
obriga o negro a agir como seu corpo. Porém, o escravo sabe que não age
como uma extensão do corpo do senhor e por isso sabe que pode ser um
agente autônomo. No entanto, nessa lógica, o escravo ainda age como mero
reflexo pois ainda se encontra no interior da estrutura de dominação
imposta pelo senhor.

É essa forma de mero reflexo que deve ser negada pelo escravo -- temos
aqui uma voz de Fanon que por outros caminhos chega ao mesmo resultado.
Nos pressupostos dados pelo senhor, foi concedida uma falsa autonomia, a
ação do escravo ainda está presa no interior da lógica colonizadora e
quanto mais ele se concebe uma autonomia por meio do seu trabalho mais
escravo se torna. Com efeito, as formas de exploração via trabalho é o
próprio limite a ser superado quer dizer: a estrutura mesma da relação
capital/trabalho deve ser implodida.

Hegel deixa claro que a liberdade conquistada pelo escravo nesse
processo não é verdadeira e, por isso, ele se volta para o interior de
si mesmo...

Ao contrário dessa exposição, porém, o trabalho passou a ser o sumo bem,
e quem é capaz de trabalhar passou a ser glorificado. Logo, o vagabundo,
identificado como aquele que não encontra trabalho, torna-se um cidadão
de segunda ordem visto com desconfiança geral e diminuído em sua
humanidade.

Nos processos coloniais o negro foi aquele que, alijado dos processos
modernizantes da indústria e comércio, tornou-se estigmatizado pela
teologia do trabalho. Podemos lembrar aqui as diversas formas de como a
malandragem e a vadiagem tornaram-se um componente epidérmico
estruturado a partir do racismo naturalizado e, como efeito, até hoje o
negro é visto como um vagabundo até que prove o contrário\footnote{A
  análise que Candido empreende sobre Memórias de um Sargento de
  milícias é sem dúvida um dos caminhos mais lúcidos para se compreender
  como a malandragem efetiva um componente de integração que inclusive
  coloca em risco algumas das ``virtudes'' da metrópole (in: CANDIDO, A.
  "Dialética da Malandragem (caracterização das Memórias de um sargento
  de milícias)" in: Revista do Instituto de estudos brasileiros, nº 8,
  São Paulo, USP, 1970, pp. 67-89.}.

Com um processo de integração racial totalmente reduzido ao processo de
valorização e produtividade do capital, a absorção da mão-de-obra negra
nunca fora um requisito necessário e, por isso, a cor da pele tornou-se,
para as polícias, sinônimo de trabalhador ou vagabundo.

Numa sociedade estruturada em nome do trabalho, e que carrega como
emblema da sua bandeira o horroroso lema ordem e progresso, esse aspecto
estrutural de sua organização caiu como chicotes do \emph{capitão do
mato} no lombo do negro.

O \emph{etos} protestante do trabalho, desenvolvido por uma gama de
teóricos, sublinhou ao máximo ``a oposição entre trabalho e
não-trabalho''\footnote{Ora, apesar de não concordarmos com o fim em si
  do processo de produção de mercadoria, uma das críticas mais
  contundentes sobre a teologia do trabalho aparecem na obra de Jappe.
  (in: JAPPE, A. As aventuras da mercadoria: para uma nova crítica do
  valor. Lisboa: Antígona, 2006 p. 103)}. A questão é que tal oposição
não ficou somente entre trabalhadores/capitalistas, mas também entre
trabalhadores/não empregados.

Igualmente, o componente racial como atributo absorvido pelo mercado de
trabalho fundamenta uma divisão no próprio seio da classe trabalhadora.
Assim, não são os negros que são divisionistas senão o próprio sistema
que integra a massa trabalhadora. Podemos perceber como a questão do
negro adentra essa perspectiva por um prisma antagônico e alheio aquela
noção de dignidade pelo trabalho creditada a grande parte da esquerda.

Aliás, talvez um dos fatores comuns em ambos espectros políticos é o
elogio acrítico ao trabalho. Tanto a esquerda quanto a direita
fetichizam a sujeição que atende pelo nome de trabalho.

A posição do negro, contudo, paradoxalmente se firmou fora daquela
unilateralidade da defesa de classe confundida como a defesa da labuta.
A questão aqui é tanto mais complexa: a defesa por uma sociedade do
trabalho nega a sua oposição constitutiva: o capital. Separados os dois
polos reciprocamente antagônicos que fundamentam a realidade
contemporânea -- trabalho/capital -- chegou-se numa superestimação de
uma sociedade do trabalho cuja a verdade é o pórtico de Auschwitz.

Quando digo isto espero ressaltar toda a tragédia humana envolta na
noção de trabalho como dignidade do homem em seu artifício racializado.
Sabemos que a noção de raça juntamente com a defesa religiosa do
trabalho durante a Segunda Guerra mundial foi o componente central da
\emph{solução final}. Cumprir o dever do trabalho e defendê-lo cegamente
são assuntos dos quais os seus resultados jamais deveriam ser
esquecidos:

\begin{quote}
Na questão da cooperação, não havia diferença entre as comunidades
altamente assimiladas da Europa Central e Ocidental e as massas falantes
do iídiche no Leste. Em Amsterdã assim como em Varsóvia, em Berlim como
em Budapeste, \emph{os funcionários judeus} mereciam toda confiança ao
compilar as listas de pessoas e de suas propriedades, ao reter o
dinheiro dos deportados para abater as despesas de sua deportação e
extermínio, ao controlar os apartamentos vazios, ao suprir forças
policiais para ajudar a prender os judeus e conduzi-los aos trens, e
até, num último gesto, ao entregar os bens da comunidade judaica em
ordem para confisco final\footnote{ARENDT, H. \emph{Eichmmam em
  Jerusalém}. Trad: José Rubens Siqueira. São Paulo: Companhia das
  Letras, 1999 p.134, grifos meus.}.
\end{quote}

Ora, então chegamos em dois resultados: primeiro, aquela noção de
absorção do componente racial pelo modo de produção capitalista se
revelou falha uma vez que o modo de produção e reprodução do capital
absorveu o componente racial em sua estrutura delegando locais e áreas
``especiais'' para os pretos. Segundo, a própria defesa acrítica de uma
sociedade do trabalho fez com que a crítica se mantivesse nos limites
impostos pelo próprio capital relegando essa estruturação racialmente
desigual como algo sem muita importância.

Paradoxalmente, porém, isso não indica que o capital não possa absorver
o Negro; é a sua estrutura competitiva e em si mesma vazia (quer dizer,
o capital é uma fantasmagoria, uma abstração-real) que promove a disputa
entre o todo social. ``A Ordem simbólica não é apenas sempre-já
pressuposta como o âmbito único da existência social do sujeito: essa
própria Ordem só existe, só é reproduzida na medida em que os sujeitos
nela se reconhecem e, por repetidos gestos performativos, nela assumem
reiteradamente seus lugares''\footnote{ŽIŽEK, S. \emph{O sujeito
  incômodo: o centro ausente da ontologia política}. Trad: Luigi
  Barichello. São Paulo: Boitempo, 2016, p.281.}.

Com isso posto já podemos compreender as limitações com que temos que
lidar atualmente no interior do Movimento Negro. Se, por um lado,
abandonar a esfera da visibilidade é um suicídio que implica vidas e
formas de subsistência, por outro, tomar a desgraça por redenção, isto
é, adotar as limitações impostas pelo modo como o poder econômico se
estrutura com esta esfera de visibilidade, é se colocar ombro a ombro
com a exploração.

O que vou demonstrar doravante é que na maioria das vezes os teóricos do
Movimento Negro adotaram esta postura.

\chapter{A origem do mito e a construção de um epígono }

Apesar de tudo que foi exposto poder soar como novidade, está impressão
é um engano. Há muito tempo se estuda a questão negra e a questão da
África de um ponto de vista que coloca em xeque aquilo que foi atribuído
tradicionalmente como África.

Foi Mudimbe quem buscou descontruir aquilo que se convencionou chamar de
Africanismo evidenciando como o problema de pensar África já é um
problema Ocidental e o substantivo africano já é uma invenção para
controle dos corpos negros\footnote{MUDIMBE, V. Y. \emph{A invenção da
  África: Gnose, Filosofia e a Ordem do conhecimento}. Luanda: Edições
  Pedago, 2013.}.

Salienta-se, contudo, que as posições, em geral, mais do que meras
diferenças epistemológicas, se revelam como diferenças políticas. Por
trás de uma posição teórica que pode soar ingênua há mecanismos
conscientes e inconscientes que podem se tornar força material de gestão
e controle, ou de quebra da gestão e do controle.

Os exemplos disso já estão desenhados na própria história recente do
continente africano. Muitos de seus países conseguiram emancipar-se
apenas no século XX e grande parte das teorias deram corpo às diversas
práticas. Praxis diversas que muito sangue derramou e seguem abertas em
suas variações.

É desanuviando esses princípios que podemos nos acercar do instigante
debate sobre África sem cair naquele romantismo reacionário e hipócrita
que vem tomando grande parte do debate no cenário conservador atual. A
famosa onda conservadora atinge todos os meridianos ideológicos.
Desconfiemos dos heróis, portanto.

Para entender a posição de Abdias do Nascimento é preciso apreender seu
contexto histórico eivado de disputas teóricas que aconteciam na vida
política do continente africano.

Por um lado, existia um pensamento hegemônico que buscava resgatar as
supostas raízes africanas e consolidar um Estado africano, ou seja lá o
que for. Algo que pudesse reunir os negros do globo distanciados pela
diáspora violenta motivada pela invasão dos colonizadores. Como esse
texto não deixa de ser expressão política, considero esse pensamento
muito próximo da direita e a partir da década de 70 totalmente integrado
às formas de gestão da barbárie capitalista.

Por outro lado, existia um pensamento minoritário encarnado
principalmente na figura de Frantz Fanon que reivindicava uma nova
universalidade a partir da destruição sociossimbólica impressa pelo
domínio colonial burguês. Isto é, uma verdadeira universalidade já que o
embrião e desenvolvimento capitalista impediu o florescimento de uma
comunidade livre e igual de pessoas e constitui uma universalidade
excludente.

Naturalmente, Abdias coerente com sua práxis política iniciada na Frente
Negra Brasileira e desdobrada na Ação Integralista Brasileira será
adepto da primeira\footnote{Ação integralista Brasileira era fascista,
  não só tinha tons fascistas como alguns ``abdistas'' fazem supor.
  Tinha um componente totalmente autoritário e sua busca pela integração
  passava pela uniformização social com base no sonho pela Ordem e
  Progresso. Respondia por um nacionalismo chauvinista que de fato nunca
  saiu da teoria de Abdias, seja por uma África mística, seja por um
  Brasil cuja identidade negra funde guetos separados travestidos de
  Quilombos. (``A importância de Abdias do Nascimento para a história do
  Brasil'', Brasil de Fato, 10 novembro 2014)}.

Seja como for, analisar as múltiplas contradições daquilo que ficou
conhecido como \emph{Pan-Africanismo}\footnote{É fato que também o que
  se chamou Pan-africanismo é heterogêneo e existem posições à esquerda
  e à direita do processo. Du bois e Marcus Garvey são as figuras
  proeminentes que representam esse antagonismo político-ideológico no
  seio do Pan-africanismo. O primeiro de esquerda enquanto o segundo
  admirador do fascismo.} seria uma tarefa titânica e fugiria ao tema
central desse pequeno ensaio que visa tão somente desmistificar posições
enraizadas no Movimento Negro.

Cumpre dizer que nem tudo que reluz é ouro e nem toda posição que se
aparenta como progressista de fato o é. Os movimentos de libertação
nacional na África em sua multiplicidade acabaram por consolidar e
manter a estrutura de exploração do capital. Quando isso encontrou
estruturas arcaicas de fetichismo religioso, ao contrário das previsões
``civilizatórias'', o que se viu foi uma radical exploração em nome
desse fetichismo.\footnote{Não se pode esquecer da terrível tradição
  milenar escravagista da Mauritânia, por exemplo, onde mais de cento e
  cinquenta mil pessoas são escravas.}

Fanon, até o fim, combateu esse tipo de posição ainda que lutasse ao
lado da \emph{Negritude.}\footnote{Negritude foi um importante movimento
  de poetas e críticos que se encontraram na Sorbonne: Aimé Césaire,
  René Depestre, Léopold Sédar Senghor são alguns dos principais nomes
  que depois se converterão em revolucionários nas guerras de
  independência. Este último tinha uma diferença radical com as posições
  de Fanon, pois reivindicava posições estritamente identitaristas.
  Também o fato de ter surgido da Sorbonne não é algo pouco importante.
  E daí que surge a noção de uma África idealizada.}

Hoje, se não dá mais para manter ilusões, convém mostrar como elas foram
engendradas. Como esse romantismo conservador foi estruturado é a
primeira forma de combate.

\chapter{Em busca da África perdida?}

Foi só com Mudimbe que as posições identitárias começaram a ser
questionadas, para não falar da desconstrução da noção
\emph{Pan-africanista}, por isso, não é de se espantar que este filósofo
seja pouco conhecido por aqui. Fazendo uma espécie de história das
ideias sobre a África, Mudimbe desnuda o condão que funda uma noção
mística de África homogênea.

É em meados do século XIX que, sobre o cavalo branco do iluminismo, cada
vez mais as ditas ciências humanas se voltam para a compreensão dos
africanos. A ethnè (ετηνοσ)\footnote{Povo em grego} é então
particularizada e funda uma nova doutrina: a \emph{etnologia}. Arma de
guerra a serviço das metrópoles, Mudimbe nos faz lembrar que a etnologia
já era um saber baseado numa diferença (inferioridade) frente aos povos.

Sobretudo, no continente africano, a etnologia foi utilizada como um
discurso que buscava fundar uma alteridade africana particular e
homogênea. Se, por um lado, procurava descrever os modos e vivências dos
nativos, por outro, objetivava uma verdadeira política de domesticação
dos modos e costumes dos povos.

Esse exercício de verdadeira colonização era empreendido por duas
medidas reciprocamente complementares demostradas por Mudimbe: \emph{a})
por meio da análise das ``instituições'' nativas; \emph{b}) através da
busca do homem ingênuo rousseauniano; o bom selvagem como uma figura
ideológica no interior do grande continente.

Por aí já é possível perceber como um discurso alienígena funda uma
noção de África que será levada adiante principalmente pelos negros da
diáspora.\footnote{Aqui marco essa distinção para evidenciar que grande
  parte da população do continente africano permaneceu aquém dessas
  categorizações.}

\begin{quote}
Poder-se-á pensar que esta nova configuração histórica significou, desde
as suas origens, a negação de dois mitos contraditórios: nomeadamente, a
``imagem hobbesiana de uma África pré-europeia, onde não existia a noção
de Tempo; nem de Artes; nem de Escrita; uma África sem Sociedade; e,
pior ainda, marcada pela perpetuação do medo e pelo perigo de uma morte
violenta'': e ainda a ``imagem rousseauniana de uma era africana
dourada, plena de liberdade, igualdade e fraternidade''\footnote{MUDIMBE,
  op. cit., p.15 apud. Hodgkin, 1957, p.174-5)}
\end{quote}

É desse modo, que a velha antropologia é erguida para lidar com o
``primitivo''. Unindo as descrições das normas, as formas da consciência
e a tentativa de captar a projeção individual com o aporte das ciências
naturais, os estudiosos europeus buscaram compreender e designar a
estrutura cognitiva dos africanos coisificando-os.

São estes estudiosos que vão erguer o mito de estruturas pré-lógicas no
homem africano e caracterizar o Ocidente como o local da fria razão, ao
passo que a África é o local dado as estruturas sensitivas e intuitivas.
Lugar onde seus indivíduos detém uma estrutura pré-lógica dominada pelas
formas de representação coletivas estritamente dependentes da
participação mística.

Ora, já vemos como alguns mitos vão sendo erguidos por, desculpe-me ter
que frisar, intelectuais brancos pagos para categorizar e definir. Mitos
que serão abraçados acriticamente inclusive por todos os membros da
\emph{Negritude}.

Assim, os estudos etnológicos empreendidos por uma antropologia
interessada terminam num etnocentrismo ideológico e conceitual como
forma de controle das populações do continente. A naturalização
etnográfica contribuiria para uma noção de \emph{raça}\footnote{Ao
  reduzir o corpo e o ser vivo a uma questão de aparência, de pele ou de
  cor, outorgando à pele e à cor o estatuto de uma ficção de cariz
  biológico, os mundos euro-americanos em particular fizeram do Negro e
  da raça duas versões de uma única e mesma figura, a da loucura
  codificada (MBEMBE, op. cit., p.11)} que seria coroada como uma forma
de natureza irredutível e particularizada.

É nessa posição antinômica -- entre uma capacidade de cognição lógica e
uma pré-lógica -- que mesmo teóricos de envergadura de Lévi-Strauss se
veem no interior de uma armadilha implicada em descrever o Outro sem
dele e nele reconhecer-se. No entanto, se, por um lado, os problemas
levantados por Lévi-Strauss ficam às voltas com a antinomia produzida
pelo conhecimento antropológico, por outro, ele é o primeiro a
demonstrar a não existência de selvagens contrapostos aos
civilizados.\footnote{LEVI-STRAUSS, C. \emph{O pensamento selvagem}.
  Trad. Tania Pellegrini. Campinas: Papirus, 1989}

Com arcabouço filosófico a noção de Outro e si-mesmo utilizados por
Lévi-Strauss já não são meras sombras de uma \emph{epistème} vazia, mas
estruturas conceituais que partem da relação sociossimbólica concreta.

Ficam expostos os pressupostos da adesão romântica e hiperfetichista de
uma África desenhada e narrada por uma antropologia que tinha sobretudo
a missão civilizatória de domesticação do ``primitivo''. \emph{A
invenção da África} redefiniu o quadro teórico e consequentemente a ação
política no interior dos ignorados países africanos.

Uma África sensual e sensível, pré-lógica e marcada pela intuição foi o
reino dos céus fornecido por uma literatura interessada no controle e na
fundamentação racial. Esse mito se manteve e, infelizmente, no
descompasso brasileiro em seu atraso com relação ao centro dos debates
vem ganhando força ultimamente.

É fácil prever que os desdobramentos dessas ideias são mais dialéticos
do que as considerações imanentes que se fazem sobre elas. Naturalmente,
as delimitações colonizadoras do que vem ser África quando tomadas pelos
colonizados foram, em raríssimos casos, subvertidas em seu sentido.

Contudo, é agora, no momento que escrevo, que um novo tipo de
questionamento começa a ser esboçado e ser reconhecido pela sua
capacidade e alcance universal num registro realmente
emancipatório.\footnote{E aqui falo, sobretudo, de Achille Mbembe apesar
  da sua limitação socialdemocrata.}

\chapter{Uma ilusão necessária contra um mito perigoso}

Quando vozes feéricas erguem bandeira a favor da proibição dos
relacionamentos entre os povos, o cheiro de enxofre polui o ar e diante
da pupila fantasmas mal desencarnados voltam a dançar.

Ora, chocar-se com esse tipo de posição não deveria ser monopólio apenas
daqueles que lutam por uma sociedade igualitária e livre, senão de todos
aqueles que conhecem minimamente um pouco de história. Sabemos onde isso
acabou.

Há no argumento um construto lógico que aproxima rapidamente a
inter-relação ao genocídio. De fato, a violência radical do processo de
miscigenação não pode nem deve ser esquecida.

O objetivo aqui é eliminar a ideia de que os relacionamentos atuais
sigam os mesmos termos violentos do processo de colonização sem,
contudo, cair na armadilha, igualmente funesta, de ver nos processos
inter-raciais fonte de reconciliação racial. Nem tanto ao mar, nem tanto
a terra. O que irá definir a superação do \emph{status quo}
definitivamente não serão os casamentos.

Se, a própria noção de miscigenação carrega o estigma das raças e foi
utilizado pelas elites como uma forma de ocultar a violenta
discriminação por meio do mito da ``democracia racial'', também este
mito retirou da crítica a capacidade de pensar nas possibilidades de
superação efetiva do quadro proposto pelo colonialismo.\footnote{A
  própria noção de ``mulato'' advém de um meio-termo entre negro e
  branco. Como fica exposta na carta do racista João Lacerda: nem tão
  ``inferior'' quanto o negro, nem superior ao branco... É preciso
  lembrar como tais princípios estavam expostos no processo de
  republicanização do Brasil acompanhando inclusive o lema da bandeira.
  Em seu ``magnânimo'' artigo dedicado ao Marechal Hermes da Fonseca,
  João Batista Lacerda, médico e cientista de ``grata estirpe'' assim
  finalizava: a importação -- sim como objeto -- em uma vasta escala, da
  raça negra ao Brasil, exerceu influência nefasta sobre o progresso
  deste país: ela retardou por muito tempo seu desenvolvimento material,
  e tornou difícil o emprego de suas imensas riquezas naturais''.(Cf.
  João Batista Lacerda. \emph{Sobre os mestiços no Brasil}. Primeiro
  Congresso Universal das raças. Londres 26-9 de julho de 1911) Logo se
  vê como a fundação e a tentativa de justificar a ``mestiçagem'' é
  tardia e veio para tentar aplacar um processo inexorável tentando
  justifica-lo à sombra da ciência positivista. O artigo completo
  demonstra de forma factível todo o racismo envolto nas análises
  teóricas que justificavam práticas políticas voltadas em sua maioria
  para a separação entre raças. Nosso ``valente'' doutor -- espero que
  tenhamos humor para entender ironias -- foi lá para tentar demonstrar
  como o aspecto da mistura poderia ser bom a partir da aniquilação do
  componente negro da sociedade. Felizmente, faltou-lhe experimentação
  histórica.}

Resta, contudo, claro que o processo de miscigenação foi acomodado pelos
princípios positivista que se baseavam na noção de inferioridade
biológica e na aposta de uma limpeza étnica dos genes num longo prazo.
Camuflou-se assim a violência sexual contra as mulheres negras e
retirou-se do horizonte a concreta contribuição dos negros para o
desenvolvimento da colônia.

Contudo, o processo cientificista positivista naufragou e a miscigenação
resultou no maior contingente populacional negro fora da África. Já são
54\% de negros no Brasil. População que vagueia pelas ruas sofrendo a
violência diária da polícia que tem o trabalho de eliminar um dos
maiores \emph{exércitos de reserva negros}\footnote{Por \emph{exército
  de reserva negro} entenda-se um dos maiores contingentes populacionais
  negros desempregados e precarizados do mundo.} do mundo.

Aquilo que anima a crítica de Abdias Nascimento é sua contraposição
radical e produtiva a noção de "democracia racial". A tensão
propriamente existente entre a miscigenação e o construto freyriano de
uma lânguida mistura entre as raças que culminaria num paraíso racial
são os termos que serão, com razão, demolidos pelas análises do autor de
\emph{Genocídio do Negro Brasileiro}\footnote{NASCIMENTO, A. \emph{O
  genocídio do negro brasileiro: o processo de um racismo mascarado}.
  Rio de Janeiro: Paz e terra, 1978.}. A questão básica a ser posta em
vista é que não se pode opor simplesmente os dois extremos,
\emph{miscigenação}/\emph{identidade}, e postular uma interação entre
eles.

A vida tanto objetiva quanto subjetiva oscila entre uma negatividade
radical perturbando o equilíbrio socialmente existente e impondo uma
nova ordem sociocultural que busca estabilizar a situação. Quero dizer
com isso que se, por um lado, Gilberto Freyre se prende em
unilateralidade - por ter uma visão conservadora do processo e se apegar
demasiadamente rápido a universalidade violenta e excludente; por outro,
contrapor estes termos através de um apelo à manutenção da identidade,
ainda que seja muito mais coerente e eficaz para a transformação efetiva
da ordem excludente, reflui em unilateralidade se não ultrapassa as
limitações da própria individualidade limitada ao interior da ordem
estabelecida.

Por isso, há dois pontos interconectados que precisamos determinar: 1) a
auto-formação do ser, enquanto ser social, não reside apenas na
adaptação a uma forma cultural pré-estabelecida seja ela no continente
africano ou nas colônias, esta formação ocorre quando se resiste aos
próprios limites impostos por ela; 2) a própria noção de raça incide na
cor por meio de uma espécie de necessidade ontológica vazia do
\emph{negro}: não basta falar ou confundir os dois termos, o negro e a
constituição da ficção raça, é a identidade dos dois termos que indica
uma contradição radical que dinamiza os processos sociais tanto na
colônia, como no continente africano, como ainda na própria Europa.

Usando um refinado arsenal crítico Abdias, porém, não conseguirá sair
dessa dicotomia que, embora, traduza elementos produtivos e fecundos,
algo como uma abertura crítica radical, o restringe à elevação
fetichista da nacionalidade, da identidade e da preservação cultural
como cerne da práxis política.

Com saudável iconoclastia, Nascimento desmonta o mito da
\emph{miscigenação} conservadora assentando sua análise no chão
histórico. Como demonstra, sobretudo as mulheres negras sobejam sob o
peso desse fetiche, porque foram elas as que mais sofreram os abusos
sexuais na colônia.

É interessante notar que desde a década de 1960 até hoje houve um
aumento de mais de 200\% nos casamentos inter-raciais que passaram de
8\% na década de 1960 para 31\% em 2010\footnote{A pesquisa foi
  realizada para um pós-doutorado de Lia Vainer Schucman na USP em 2017
  com apoio da Fundação de Amparo à Pesquisa do Estado de São Paulo
  (Fapesp), colaboração de Felipe Fachim e supervisão de Belinda
  Mandelbaum, coordenadora do Laboratório de Estudos da Família do
  Instituto de Psicologia (IP) da USP.}, contudo, a estratificação
social e um sexismos subjacente a estes casamentos permanecem sendo a
toada das uniões. Sexismo porque a maioria dos casamentos são entre
homens negros e mulheres brancas, algo como um troféu ou vingança tão
bem explicitado por Fanon\footnote{FANON, op. cit.}. Estratificação
social porque eles ocorrem somente entre membros das classes populares
indicando o racismo na estrutura de classe.

Paradoxalmente a miscigenação atual é algo interna ao proletariado.

Como esta análise visa desnudar o que está presente na ordem do
discurso, podemos concluir que o choque da maioria das pessoas frente a
noção de anti-miscigenação advém sobretudo, da universalidade ideológica
que a miscigenação diz promover. Acentuando a ideologia por detrás dessa
noção Abdias Nascimento coloca sobre o teto a violência implicada no
mito de ``democracia racial''.

\begin{quote}
Postula o mito que a sobrevivência dos traços da cultura africana na
sociedade brasileira teria sido o resultado de relações relaxadas e
amigáveis entre senhores e escravos. Canções, danças, comidas,
religiões, linguagem, de origem africana, presentes como elemento
integral da cultura brasileira, seriam outros tantos comprovantes da
ausência de preconceito e discriminação racial dos brasileiros
"brancos"\footnote{NASCIMENTO, op. cit. p.55}.
\end{quote}

Quando se desnuda as relações promíscuas e violentas que são ocultadas
pelo termo miscigenação um curto-circuito parece ocorrer. A suposta
universalidade parece conter a cor \emph{branca} e funcionar com a total
subordinação do negro como um negativo indesejado. Assim, o negro fica
fora do circuito fechado dessa universalidade sendo aceito somente
enquanto perda de si mesmo, enquanto negação de sua identidade.

Isso nos leva ao centro de nossos questionamentos: a identidade surge
como resultado de um paradoxo entre o presente (conscientemente
refletido) e um passado (memória). A rememoração é o componente fecundo
que por meio do desvio pelo passado constitui nossa própria experiência
do presente. Com efeito, a história - o cadáver - é disputado na
formação da própria subjetividade. Abdias do Nascimento segue esta
trilha se agarrando no resgate de uma individualidade despedaçada e
suprimida.

Afirmo que a posição de Abdias Nascimento é uma disputa firme e valente
pelos cadáveres que sucumbiram no Oceano, mas é preciso ressaltar que a
interação entre passado e presente precisa ser mais que interações e
relações, precisa se interpenetrar e se tornar uma autorreferência capaz
de abrir o horizonte para aquilo que era até então posto como
impossível.

O que quero dizer é que se, por um lado, de fato a noção de miscigenação
foi a borracha que visava apagar os crimes ocorridos no interior da
colônia cujos resultados estão impregnados no nosso cotidiano, por
outro, esta demonstração não precisa incidir no retorno abrupto de uma
identidade \emph{não relacional}. A identidade é um fenômeno relacional
que advém da interação entre conjuntos diferentes de atividades no mundo
circundante. Ela é sempre aberta, historicamente determinada e algo
passageiro.

***

Também as formas de controle sobre a cultura e religiosidade africana
serão marcas da intransigente crítica de Abdias Nascimento. A religião
tornada caso de polícia, a implosão de terreiros, as prisões arbitrárias
compõem o enredo surdo de uma desfaçatez que se une a ideologia
alcunhada; "democracia racial". A perseguição que destituía o negro de
seus recursos simbólicos torna-se razão de Estado que o impede de se
achegar a compreensão de si mesmo.

O controle radical e violento exercido sobre as formas de expressão
religiosa -- para se ter uma ideia os terreiros na Bahia só precisaram
deixar de ser registrados na polícia em 1976\footnote{NASCIMENTO, op.
  cit., p. 105} -- foram uma forma de submeter o negro para que não
pudesse resistir ao modo de vida imposto pelas formas de exploração e
opressão colonizadoras. Essas máculas vieram acompanhadas da noção de
sincretismo.

Tal noção oculta não só uma violência que desestrutura a própria matriz
simbólica do negro como oculta também um processo de resistência que vai
sendo aos poucos elaborados a fim de manter viva as tradições
religiosas. Agarrado aos elementos constitutivos de sua religiosidade, o
negro tentar manter sua estrutura que como resistência provoca a
suspeita do poder metropolitano atingindo até mesmo a era republicana.

Esse passo dado por Abdias Nascimento, embora detenha elementos que
incidem numa reação ao choque cultural, ergue também uma disputa por
essa cultura. É preciso, no entanto, salientar que o processo de
autorreferência mantida pela religião destacada do seu lugar originário
designa o momento em que a atividade religiosa - e com ela a identidade
- não circula mais em torno do local que a produziu, mas gera seu
próprio "rito". A consciência negra passa a pôr a si mesma como algo
\emph{outro} do que fora, isto é, a produzir-se sob nova condição.

É essa impossibilidade teórica de Abdias Nascimento, causada sobretudo
porque parte das limitações concernentes à estrutura ligada a raça como
elemento hipostasiado, que o faz cometer algumas idiossincrasias
conceituais e regredir perigosamente o escopo de abertura que sua
própria crítica havia possibilitado.

Noutros termos, quando se aferra ao resgate puro de uma cultura
impossibilitada, porque historicamente destroçada, nosso crítico fecha a
abertura crítica, com implicações políticas seríssimas, que sua própria
análise propicia. Em todo seu caminho, vemos se debater com esses
limites e mesmo afirmar que somente por uma revolução as limitações da
raça poderiam ser solapadas. No entanto, ele não sabe se essa revolução
é racial ou social, embora chegue aos mesmos resultados de Marx no
\emph{Manifesto do Partido Comunista}:

\begin{quote}
O ponto de partida da classe dirigente branca foi a venda e compra de
africanos, suas mulheres e seus filhos; depois venderam; o sangue
africano em suas guerras coloniais; e o suor e a força africanos foram
vendidos, primeiramente na indústria do açúcar, no cultivo do cacau, do
fumo, do café, da borracha, na criação do gado. {[}...{]}"Venderam'` os
espíritos africanos na pia do batismo católico assim como, através da
indústria turística, comerciam o negro como folclore, como ritmos,
danças e canções. A honra da mulher africana foi negociada na
prostituição e no estupro. Nada é sagrado para a civilização ocidental
branca e cristã. Teria de chegar a vez da venda dos próprios deuses. De
fato, os orixás estão sendo objeto de recentes e lucrativas
transações\footnote{NASCIMENTO op. cit., p.119}.
\end{quote}

Este curto-circuito ocorre porque Nascimento tem uma noção identitária
de identidade. Para ele sem identidade resta a alienação. Abdias não vê
que a alienação é constitutiva da identidade como processo de
organização vazia, contínua e permeada de colisões que a estruturam e a
reestruturam. Nesse sentido, por vias outras comete o mesmo erro
daqueles que veem a alienação como um empecilho para chegar ao ser
\emph{em-si}. Por isso, ele quer dar corpo a descorporizada \emph{classe
dirigente}, quer dar cor ao sistema de visibilidade que funciona sobre o
modo de exploração.

Isso marca a diferença radical entre Abdias Nascimento e Frantz Fanon.
Este último sabia que a substancialidade de qualquer significante perene
que possa fundamentar uma plenitude do Eu está barrada pela própria
forma como a consciência se põe no mundo. Ou melhor, é lançada no mundo.

Se, a ilusão necessária de Abdias Nascimento vai até a possibilidade de
desnudar o tema tabu: \emph{democracia racial}, a partir do momento em
que busca o império da identidade como não relacional, sua abordagem
torna-se \emph{reativa} culminando na adoção utópica -- ou seria
distópica? -- de um Estado africano \emph{por vir}.

É assim que sua análise estética reduzida sempre ao conteúdo explícito,
torna-se equivalente a censura de Platão aos poetas. Se, a forma
artística tem autonomia quanto ao contexto e é uma tentativa de unir
elementos dispersos e heterogêneos no mundo e configurá-los pela sua
união, num todo artístico criado, então ela responde por esse mundo
demonstrando-o em toda sua hipocrisia, paixão, vício e
virtude.\footnote{LUKÁCS, G. \emph{A teoria do romance}. São Paulo:
  Editora 34, 2010.}

A autonomia da arte implica sua liberdade de trazer à luz aquilo que é
produzido à sombra. Faz isso na reunião de elementos heterogêneos,
dispersos e que indicam que não há sentido dado para vida senão aquele
produzido na ação humana. Por isso, quando Abdias condena os artistas
negros por não se expressarem enquanto tais, perde de vista não só as
possibilidades que se evidenciam nas formas impressas por estes
artistas, como também, recai num proselitismo rebaixado.

A obra de arte responde por seu mundo, se esse mundo é racista a obra de
arte conterá tais elementos, ou subvertendo-os, ou demonstrando-os, ou
ainda fazendo ambos. Impedido de chegar em tais conclusões dado o ponto
de partida identitário, a crítica de Nascimento à Machado de Assis é
medíocre. Para ele, Machado de Assis é simplesmente um aculturado que
privilegiou a classe-média branca em suas obras.

Reduzindo a obra de Machado à superfície aparente - classe social e
epiderme -, não consegue compreender as tensões e críticas radicais
contidas na fina ironia de Machado de Assis. Nem sequer se dá conta da
formação racista, hipócrita e violenta contida em seus personagens
aparentemente brandos e eruditos. Quer dizer, Abdias Nascimento não se
dá conta da crítica mordaz machadiana que desnuda as relações raciais
cujo cinismo sangrento assenta as raízes de uma elite plutocrata e
racista que se mantém no poder até hoje.\footnote{A esse respeito o
  clássico ensaio de Schwarz ensina muito (cf. Schwarz, R. \emph{Um
  mestre na periferia do capitalismo}. São Paulo: Editora 34, 2000).}

Há, sem dúvida, diferença entre, por um lado, a busca pela identidade
(\emph{relacional}) como fundamento de uma subjetividade atuante nos
processos sociais e, por outro, o identitarismo (identidade
hipostasiada) como elevação de características originárias que não podem
``se misturar''.

O discurso de Abdias do Nascimento se funda nessa tensão que ao afirmar
a identidade e desnudar os mitos empreendidos pela "democracia racial"
se vê atolada na busca por uma raiz inexistente. Como resultado
fundamenta um discurso ideológico cujo conteúdo ganha força material
pelo seu vazio de significado e torna-se facilmente cooptável pelas
forças hegemônicas.

São tais forças que com sua ideologia atestam a luta entre os conteúdos
particulares promovendo a universalidade ideológica capaz de estabelecer
uma organização social. Esta universalidade está sempre em disputa, daí
a importância da afirmação da identidade desde que se veja nela um
elemento evanescente, isto é, formado negativamente pelo embate com o
\emph{Outro} negativo. Como vimos, não é essa identidade que está em
cena no desenvolvimento teórico de Nascimento.

Como parte de um pressuposto local e de uma identidade estanque, Abdias
ignora as particularidades tanto dos descendentes de africanos da
diáspora quanto daqueles negros em continente africano. A
particularidade de um negro brasileiro fatalmente se contrapõe a
particularidade de um negro em Angola. Na relação de ambos o elemento
epidérmico se desfaz para se projetar o ambiente socioeconômico e
cultural que formam suas individualidades. Fanon sempre ilustrou a
diferença entre um negro da Martinica e outro de qualquer colônia, isto
porque, contrariamente a Abdias do Nascimento, era anti-essencialista.

\chapter{Uma crítica necessária}

É interessante notar como as análises empreendidas e o apontamento sobre
a violência colonial feitos por Abdias são verdadeiros no que se refere
ao desmascaramento da ideologia de ``democracia racial'', enquanto os
pressupostos por trás de suas conclusões, todavia, jogam no campo da
integração como um componente afirmativo.

A integração do negro se baseia, desse modo, na prioridade à preservação
da cultura, da tradição, dos costumes e ao desenvolvimento identitário
de sua particularidade\footnote{O hipotético-leitor atento já percebeu
  que esta é a bula papal do integralismo...}. Não é a exploração
constituída pelo sistema, que arrancou os negros do continente africano,
a chave da dominação e da exploração da população negra, segundo
Nascimento, mas, sim o ``roubo'' de sua identidade pelo embranquecimento
social.

A questão sintomática da posição de Abdias do Nascimento é que a
proclamação da diferença se limita ao quadro referencial posto. Ora, se
muitas vezes ``o desejo da diferença emerge precisamente dos lugares
onde se vive mais intensamente a experiência de exclusão''\footnote{Mbembe,
  op. cit. p. 304}, a possibilidade de ultrapassá-la, enquanto gesto de
poder, necessita estar intrínseca a um projeto mais vasto.

É natural que para aqueles que passaram pelo processo de dominação
colonial e supressão de sua alteridade, a proclamação da diferença é
central, e daí o apelo da identidade. Entretanto, ela só pode
ultrapassar seus limites quando se põe no sentido de endossar um mundo
livre do peso da raça.

A raça, ao contrário das previsões otimistas do marxismo vulgar, foi
tomada como componente estrutural no mercado de trabalho e não absorvida
equitativamente por este. Abdias também sabia disso, todavia, restrito a
metafisica da raça (metafisica, no sentido aqui atribuído, como um
componente fetichista que impõe uma dinâmica social) busca, reafirmá-la
ao procurar integrar o componente racial como possibilidade de sanar a
desigualdade.

E nesse ponto se encontra todo o construto que fecunda suas análises:
Nascimento torna o conceito de raças a-histórico.

Quando faz o debate histórico parte de uma unilateralidade latente cujo
mérito foi o de desmontar a noção de ``democracia racial''. Quando parte
para o debate econômico, para se livrar rapidamente da noção de luta de
classes, entrega-se ao interior dos modos de operacionalização social
efetivado pelo processo de produção e reprodução do capital sem
questionar sua sociabilidade baseada na organização e divisão entre as
raças, as classes e as nacionalidades.

Não sendo ingênuo suficiente para negar as divergências e complexas
diversidades dos grupos africanos que foram arrancados de suas terras,
Nascimento, para justificar seu pan-africanismo, necessita de um
componente que possa unificá-los sem comprometer sua argumentação:
\emph{a religião}.

As chamadas ``culturas irmãs'', quer dizer um grupo imenso e
heterogêneo, são reduzidas à insígnia da oralidade e politeísmo. Tal
como os primeiros etnólogos fizeram, parte-se aqui de um mito formulado
pela ``ciência''. É, por fim, o Candomblé o sinal de unidade -- mesmo
havendo dicotomias entre os Yorubas, os Ewe do Benin e do Congo etc.

A argumentação torna-se toda etnóloga e utiliza os elementos místicos
como fundadores de uma nova forma de subverter o \emph{status quo}. É
óbvio que as religiões de matriz africana sofreram o policiamento
grotesco enviesado por uma política racista. Defender a liberdade de
culto é uma premissa básica para uma construção democrática.

Também, a conclamação por um retorno aos vínculos orgânicos não é o
problema em Abdias do Nascimento senão sua formulação entre este suposto
retorno e a busca por uma mobilização tecnológica capaz de fundar uma
ideologia corporativista estetizada: a busca por um Estado que põe para
si mesmo o elemento de modernização radical ligado à defesa orgânica de
vínculos étnicos.

Abdias do Nascimento ao desnudar o racismo institucionalizado na
sociedade brasileira não vai além das premissas básicas. Não despe as
relações de poder baseadas nas formas do desenvolvimento econômico que
fundamentam as instituições. E embora questione a construção ideológica
do racismo como uma forma biopolítica que divide os cidadãos naqueles
que podem ser mortos e os que não podem, reivindica a participação no
projeto.\footnote{MBEMBE, A. \emph{Necropolítica: biopoder, soberania,
  estado de exceção, política da morte}. Arte \& Ensaios revista do
  ppgav/eba/ufrj n. 32, dezembro 2016 pp.123-151}

Parando a meio caminho, identifica a dominação como algo do
branco-europeu e em sua retórica anticapitalista não identifica o modo
de sociabilidade capitalista como o cerne a ser combatido senão a luta
pela afirmação cultural como \emph{leitmotiv} da ``transformação''
social.

Essencializando um inimigo por meio da epiderme, o poder derivado das
forças econômicas em jogo desaparece e o capitalismo não é mais o
problema, mas a ferramenta que se usa para a ``manutenção racial
branca\footnote{NASCIMENTO, A. Quilombismo. São Paulo: Editora Vozes,
  1980 p.16}''. Em sua argumentação culturalista e etnóloga o que impede
a integração do negro na vida coletiva é o não-reconhecimento dos
vínculos culturais e ideológicos entre os afrodescendentes da diáspora e
os africanos.

Ora, tudo isso não quer dizer que Abdias não faça a crítica ao
capitalismo. O problema é que ele oculta que o espírito foi transformado
em capital no próprio continente africano e uma vez ocorrido o processo
em que o termo \emph{nativo} foi empregado para aqueles que estão
enterrados em seu local de nascimento, não há retorno. O processo é
irreversível.

Sabotando a roda violenta da história, Abdias do Nascimento crê na
possibilidade de salvaguardar as origens de uma África existente somente
nos delírios de etnólogos como possibilidade de redimir inclusive o
opressor:

\begin{quote}
A restituição aos africanos daquilo que era antes unicamente seu, neste
momento histórico de crise aguda do capitalismo, apresenta
necessariamente implicações de relevante função ecumênica. Pois uma vez
mais a redenção do oprimido em sua plena consciência histórica, torna-se
em instrumento de libertação do opressor encurralado nas prisões a que
foi conduzido pela ilusão da conquista.\footnote{Idem, p.42}
\end{quote}

Para justificar suas posições parte então de uma prerrogativa que
traçará a rota de suas argumentações até o fim: \emph{as culturas
africanas}. Sem dizer o que são tais culturas ou demonstrar a
multiplicidade de culturas existente no continente africano, o líder do
MNU apenas dirá que estas estão fundamentadas na organização social
coletiva, criatividade, redistribuição e propriedade de forma
equitativa. A propriedade não deixa de existir nas supostas ``culturas
africanas'', muito embora propriedade seja um substantivo criado,
mantido e defendido a ferro e fogo pelos colonizadores.

\begin{quote}
As culturas africanas são aquilo que as massas criam e produzem: por
isso elas são flexíveis e criativas, assim como bastante seguras de si
mesmas, a ponto de interagir espontaneamente como outras culturas,
aceitando e incorporando valores científicos e/ou progressistas que por
ventura possam funcionar de modo significativo para o homem, a mulher e
a sociedade africana\footnote{Idem, p.46}
\end{quote}

Seja o que for, o entendimento de culturas africanas é genérico e não
leva em conta a multiplicidade da África. A África para Abdias é um
país, ou melhor, uma pátria. Seu projeto de unidade pan-africana é,
sobretudo, nacionalista\footnote{A esse respeito é preciso assinalar que
  o Pan-Africanismo foi heterogêneo em suas posições políticas. A
  começar pelos seus principais idealizadores Marcus Garvey à direita do
  processo e William Edward Burghardt Du Bois à esquerda.}; visa
edificar o \emph{ser nacional}:

\begin{quote}
Na estrutura da presente fase da ``ajuda técnica'' as formas avançadas
de tecnologia do capitalismo industrial, além de não cooperar na
construção, em verdade instigam e promovem a penetração do capital
monopolístico internacional e a alienação do autoconhecimento nacional.
Esta ``ajuda'' tecnológica e científica estará apta a tomar os rumos da
libertação somente quando os valores capitalistas que regem e regulam
seus mecanismos não forem utilizados para deter o desenvolvimento da
consciência dos povos e da independência nacional\footnote{Idem, p. 73}
\end{quote}

Se os valores capitalistas que regem e regulam seus mecanismos forem
utilizados em prol do desenvolvimento da consciência dos povos e da
independência nacional, então sem problemas.

Com tais posições, Abdias do Nascimento não era só contra o pensamento
de Marx, ele era na verdade antimarxista e antimarxiano\footnote{Isso
  pode iluminar, a meu ver, suas escolhas e posições políticas durante o
  regime militar, Abdias do Nascimento preferiu o PDT ao PT -- por este
  na época ser muito classista. Do mesmo modo, nunca se arrependeu de
  ter sido um integralista e foi até o fim coerente com sua posição
  juvenil amadurecendo-a ao longo dos anos e encontrando no
  pan-africanismo, nacionalista e identitário, os resultados conceituais
  de que necessitava.}. Radicalmente contrário ao pensamento dialético
que impõe, além de outras coisas, a necessidade de se mesurar o choque
de culturas e o desenvolvimento histórico destas em suas trocas, para
ele os intelectuais que partissem das análises de Marx fracassariam ao
tentar compreender o desenvolvimento das raças. Hipostasiada a noção de
raça em sua safra ``conceitual'', é óbvio que Nascimento vê em Marx um
forte concorrente ao seu misticismo retórico.

Segundo ele foi ``Marx (quem) substituiu a categoria humana dos
africanos pela categoria econômica''\footnote{Ibidem.} e não o
capitalismo. E triunfante conclui: ``não aceitamos que uma pura mágica
conceitual possa apagar a realidade terrível da opressão dos brancos
europeus contra todo continente e sua raça negra'', apaga-se com isso o
componente socioeconômico e no seu lugar Abdias do Nascimento ergue o
princípio cultural essencialista. Não é o capital o problema, mas o
branco-europeu.

Ora, sabemos que o anseio por uma vida autêntica em comunidade não pode
ser reduzido ao significante de anseios totalitários, há no caráter
utópico e não ideológico dessa posição algo que deve ser afirmado. A
dificuldade nessa posição é como tais anseios serão articulados e
funcionalizados.

Se, se partir de um ponto especifico ante a exploração totalizadora do
capital (o domínio do capital financeiro, a ``influência judaica'', a
``epiderme'' dos indivíduos que compõe as elites, a ``influência dos
estrangeiros'' no desmonte da nacionalidade, etc.,) fatalmente deixará
de implicar-se numa transformação estrutural para atuar no interior da
limitação posta. É aquilo que João Bernardo chamou de revolta na/pela
ordem\footnote{Cf. BERNARDO, J. \emph{Labirintos do fascismo: na
  encruzilhada da ordem e da revolta}. 2015}.

Ora, é exatamente por isso que a ilusão necessária proposta por Abdias
Nascimento, em seu desnudamento do racismo estrutural brasileiro,
encontrou seus limites que precisam mais que nunca serem ultrapassados.

Por outro lado, não se pode jogar o bebê junto com a água, pois as
tecnologias de reificação do negro conduziram a processos históricos que
o dilaceraram em sua humanidade. Por isso, a noção que Abdias Nascimento
tem sobre as políticas de reparação se aproxima daquela necessidade,
vislumbrada por Mbembe, de recuperação social dos laços que foram
quebrados e instauração de uma alteridade recíproca sem a qual a
possibilidade de uma consciência comum do mundo estaria vedada. A
reparação social do negro está para além dos processos econômicos,
trata-se de um reconhecimento de sua humanidade.

\chapter{Ao pé do muro}

Somos convidados pelas circunstâncias históricas a desafiar os limites
socialmente impostos. A cada vinte e três minutos um jovem negro é
assassinado nesse país\footnote{Mapa da Violência, da Faculdade
  Latino-Americana de Ciências Sociais (Flacso).}. No momento em que as
balas do Estado perfuraram o corpo frágil de Marielle\footnote{Marielle
  a quinta vereadora com mais votos no Rio de Janeiro, era negra,
  lésbica, socialista e ativista dos direitos humanos, foi assassinada
  logo após de ser nomeada para monitorar a intervenção federal no Rio
  de Janeiro e denunciar a violência policial.} acabou-se qualquer
ilusão com as limitações da representatividade. Nossa resiliência enfim
se esgotou.

Se, ``o processo histórico foi, para grande parte da nossa humanidade,
um processo de habituação à morte do outro''\footnote{MBEMBE, op. cit.
  p. 305}, teremos que reinventar o próprio sentido de comum superando
as lesões sem deixar cicatrizes por meio da partilha de nosso destino.

O ato soberano de definir, num \emph{horizonte decrescente de
expectativas}\footnote{ARANTES, op. cit.}, quem morre ou quem vive se
realiza com a morte de centenas de negros. A necropolítica realizada
diuturnamente, num país que jamais abandonou sua posição de periferia,
carrega em seu jardim regado a sangue negro, o sonho de modernização e
progresso\footnote{MBEMBE, op. cit.}.

É a situação que ganha através dos saberes divergentes que ela suscita,
diz Isabelle Stengers\footnote{STENGERS, I. \emph{No tempo das
  catástrofes -- resistir à barbárie que se aproxima}. São Paulo: Cosac
  Naify, 2015}, e, se ela estiver correta, a situação atual solapou
nossos referenciais teóricos, nossos saberes até aqui foram coniventes e
buscaram atuar nas limitações sistêmicas. Ignoramos até ontem o
crescimento do assassinato da juventude das periferias, essa ignorância
cobrou seu preço e, infelizmente, com mais sangue.

A política reduzida a antipolítica, isto é, uma forma de guerra
permanente no qual o elemento democrático é esvaziado em nome das
``limitações'' econômicas enviesadas pela busca por lucro \emph{ad
infinitum}, impõe ao grosso da população a morte como resolução dos
possíveis \emph{conflitos vindouros}.

É preciso revisitar nosso entendimento sobre a noção de soberania e
como, no Brasil, essa noção longe de buscar a autonomia dos indivíduos
foi somente a instrumentalização dos corpos em prol de valorização do
capital sustentada por uma elite econômica composta por Bentinhos. O
Estado aqui fora sempre de \emph{Estado de} \emph{exceção}.\footnote{Isto
  é liberal na aparência, conservador e racista na realidade.}

Se, contudo, Achille Mbembe estiver certo e a noção de negro estiver
sendo apartada da condição epidérmica, doravante o elemento
revolucionário é o negro. É para ele que os imperativos
contra-insurgentes e as guerras de ocupação são perpetradas. Do Rio de
Janeiro à Palestina, da Turquia à Afrin passando pela Catalunha e pelos
desabrigados de Detroit, essa condição parece se perpetuar. A
Intervenção Federal que cerca a favela, relembrando os velhos guetos
poloneses, é somente mais um indício dessa ``tendencial universalização
da condição negra''.\footnote{MBEMBE, 2014, p.16}

Se o negro se tornou sobretudo uma condição de experimentar a si mesmo
como forma de vida imposta pela ``gestão dos destroços do
presente''\footnote{ARANTES, op. cit. p. 91}, ou ele se torna comunista
ou não será nada.

Isto porque ``a raça foi a sombra sempre presente sobre o pensamento e a
prática das políticas do Ocidente, especialmente quando se trata de
imaginar a desumanidade de povos estrangeiros -- ou
dominá-los''\footnote{MBEMBE, 2016, p.128.}. Excepcionalmente numa
posição em que se supere essa forma de política condicionada e
condicionante pela morte, poderemos exercer a indiferença as diferenças
num horizonte de revolução social.

Em uma época na qual a forma de aparição do negro foi redefinida em
função da configuração geral das hostilidades pelo Império do mercado, a
mais lamentável confusão diz respeito à ``regressão identitária''. Nesse
sentido, a perda da ilusão concernente a desalienação radical é a pedra
de toque de uma nova interpretação.

Já sabemos haver uma alienação radical que é constitutiva da nossa
própria ordem simbólica. Ordem esta que impõe que nossa verdade esteja
fora de nós mesmos; uma linguagem que descentra nossa identidade e
impede o sonho de se apropriar da plenitude. Um sonho podado pela
realidade que nos atravessa. Portanto, não temos ilusões.

Nem a identidade, nem a universalidade, nem a classe constituirá um novo
horizonte. Mas a relação entre esses limites. Não se trata da busca por
um paraíso perdido, mas da construção de possibilidades do que era
impossível a partir da compreensão de nossas condições de possibilidade.

É nesse momento que rememorando nosso terrível passado podemos redefinir
as coordenadas do presente:

\begin{quote}
Qualquer relato histórico do surgimento do terror moderno precisa tratar
da escravidão, que pode ser considerada uma das primeiras instâncias da
experimentação biopolítica. Em muitos aspectos, a própria estrutura do
sistema de colonização e suas consequências manifesta a figura
emblemática e paradoxal do estado de exceção. Aqui, essa figura é
paradoxal por duas razões. Em primeiro lugar, no contexto da
colonização, figura-se a natureza humana do escravo como uma sombra
personificada. De fato, a condição de escravo resulta de uma tripla
perda: perda de um ``lar'', perda de direitos sobre seu corpo e perda de
status político. Essa perda tripla equivale a dominação absoluta,
alienação ao nascer e morte social (expulsão da humanidade de modo
geral). Para nos certificarmos, como estrutura político-jurídica, a
fazenda é o espaço em que o escravo pertence a um mestre.\footnote{MBEMBE,
  2016, p.131}
\end{quote}

Noutros termos, estivemos enquanto colônia até agora na vanguarda dos
processos de terror e controle social empregados pelo Estado. Sabemos
disso, sentimos em nossa pele diariamente. Nossos mortos já somam
milhões. Doravante se trata do esforço por uma verdadeira democracia que
não se sustente mais na economia predatória do capital da qual emanam as
formas de gestão da barbárie.

É necessário empurrar o ídolo ladeira abaixo.

As formas de gestão do capital em sua fase manipulatória\footnote{ALVES,
  G. \emph{Trabalho e subjetividade: o espírito do toyotismo na era do
  capitalismo manipulatório}. São Paulo: Boitempo, 2011.} não são apenas
monopólios das grandes transnacionais. Elas invadem o espaço geográfico
das cidades; determinam o ritmo empenhado pelo fetiche ao lucro;
congregam vidas em torno do seu objetivo; objetiva a vida dos indivíduos
reduzindo-os ao currículo e invadem sua subjetividade dilacerada pelo
modo de operação de sua linguagem.

A política moderna vinculada a esfera das demandas econômicas, tornou-se
uma política de resultados determinada pelo nível de investimentos e o
esperado retorno. Enquanto isso, ela exerce o ato soberano sobre quem
vive e quem morre com intuito de conter as insurgências. A polícia é o
braço assassino dessa política que paradoxalmente torna a condição negra
algo universal.

Não há mais política somente polícia. Os espaços de discussão
``democrática'' reduzem-se à discussão sobre a manutenção do lucro e a
redução dos danos que incidem no corte abrupto dos direitos da maioria
da população. Todos os partidos apegados a gramática e ao \emph{modus
operandi} da situação a-histórica do capital, ``já não há mais futuro'',
tornaram-se partidos da ordem.

Ainda que os partidos de esquerda sejam vítimas do ato soberano imposto
pela oligarquia só podem superar a sua condição se se superarem enquanto
partido, isto é, enquanto órgão burocrático limitado a gestão das
demandas do capital e organização interna.

O capital nunca cessou com sua guerra aos opositores e se a mão
invisível do mercado é leve com seu riso de raposa nos países centrais,
por outro lado, nas antigas colônias, ao sul do globo, rege sua
população com mão de ferro eliminando à bala todos os que contestam sua
hegemonia. Sob o disfarce de guerra ao tráfico se elimina, com pólvora e
chumbo, um contingente populacional preto e supérfluo anestesiado por
uma violência estatal constante que sustenta e se sustenta pelo
narcotráfico e contrabando de armas promovidos pelo próprio Estado.

No horizonte de desgraças se ``como instrumento de trabalho, o escravo
tinha um preço e como propriedade tinha um valor''\footnote{MBEMBE,
  2016, p.131}, na era das expectativas decrescentes ele torna-se um
excedente que precisa ser eliminado. O mundo espectral de horrores e
crueldade fundamentado pela tecnologia de morte já pode quantificar do
espaço aéreo quanto um míssil pode conter de gastos ao assassinar de uma
vez cinquenta ou mais possíveis insurgentes.

A condição negra chegou a Palestina e sua neocolonização imposta por
Israel. A condição negra se expressa no extermínio das forças
revolucionárias de Afrin sob um silêncio implacável dos países
civilizados. A condição negra se expressa na expulsão de famílias dos
EUA pela etnologização da política. A condição negra se expressa no
assassinato de crianças em favelas pelas balas ``perdidas'' da polícia
militar.

Implacavelmente, os gestores da miséria avançam como um colosso
indestrutível cedendo aos seus lacaios benesses e espalhando \emph{think
tanks} para jovens ovelhas que queiram se ajoelhar ao seu império. Ao
mesmo tempo, mantém, inseparável de sua política, o ódio fomentando pela
divisão racial e nos arcaísmos de sua estrutura impõe a luta intestinal
pela sobrevivência no mercado de trabalho. Esse é o papel que a esquerda
durante muito tempo representou: convencer os milhões de miseráveis que
o sistema ainda é viável. Com os tiros na emboscada não podemos
acreditar mais nisso.

Contra esta situação de miséria atual somente uma organização negra --
entendida aqui como condição universalizada -- capaz de confrontar a
gramática e estabelecer desde já alternativas que se coloquem além das
formas estatais de controle poderá ser efetiva. A união de todos
explorados, dos párias e da plebe, e não a unidade programática, é a
única forma de responder a altura os desafios que a ofensiva do
capitalismo impõe.

Se o fantasma das gerações passadas pesa na consciência dos vivos, não
há motivo para nostalgia com as formas carcomidas por duzentos anos de
capital. O desbloqueio do novo é tarefa dos que se voltam contra a
universalidade do capital e que compreendem que só com sua superação
será possível uma vida digna. Nós negros sabemos o que essa
universalidade nos impôs e continua impondo.

Já passou o tempo de choramingar e encontrar prazer numa melancolia
eterna que, nostálgica, lembra-se dos velhos tempos democráticos. Foi no
lombo do negro que a república se perpetuou elidindo de si a
participação do negro, assassinando-o friamente pela fome ou pela arma.
Foi no lombo do negro que o significante vazio de democracia se tornou
uma ponte ideológica para que, sob os olhos civilizados da Europa
branca, países fossem invadidos e crianças fossem assassinadas para
levar seus ``benefícios''.

É sobre o capitalismo que os direitos humanos servem para apresentar um
homem que já não tem mais nada e olha desesperado o seu filho cruzar a
fronteira sozinho. É em nome disso que moradores são cercados pelo
exército na maior favela do mundo, enquanto a burguesia nacional nos
seus jornais se orgulham de dar essa lição ao ``mundo civilizado''. E em
nome dessa política que companheiras são assassinadas na segunda maior
cidade do país. E estão procurando quem foi. Sabemos quem foi. Todos
sabem!

Disputar esse significante vazio é tarefa urgente: de que democracia
estamos falando? Precisamos de um pastor ou deixaremos nosso destino nas
mãos de oligarquias? São esses limites ou seremos capazes de implodi-lo?
A democracia radical -- livre dos modos de existência do capital -- será
negra ou não será, o \emph{devir-negro do mundo} precisa ser subvertido.
A transformação será obra daqueles que efetivamente construíram esse
país.

Está na hora de novamente reivindicarmos a violência fanoniana: como uma
prática de ressimbolização social advinda da ruptura radical com a
estrutura sociossimbólica atual. É através da violência escolhida e não
mais sofrida que como colonizados poderemos obter uma reviravolta sobre
nós mesmos e nosso destino. Revolução é sofrimento.

Só assim nos libertaremos do colonialismo psíquico e objetivo que
engendrou nesse país forças fundamentalmente necropolíticas animadas por
um instinto genocida e disseminadas pelo modo de sociabilidade
competitivo e classista do capital. Só assim deixaremos de produzir
mártires e só assim deixaremos de nos ocupar com os tons de nossa
epiderme. Só assim estaremos aptos a viver nossas vidas com todas as
nossas potencialidades. Só assim criaremos um mundo em que caibam todos
os outros.



\end{document}
