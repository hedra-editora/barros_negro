\begin{itemize}


\item \textbf{Lugar de negro, lugar de branco?} busca desmistificar a naturalização do lugar da raça na discussão moderna e de sua força instituinte: a escravidão moderna. Com forte alicerce em uma leitura crítica de Frantz Fanon, o ensaio repensa o identitarismo, que ganha espaço nas militâncias, ao relacioná"-lo à procura mística de uma África que, historicamente, é indissociável do processo de produção capitalista.
 Ao transpor o problema da raça e do significante negro para um novo patamar, o livro lança novas hipóteses para o movimento negro e aponta para sua potência em superar as relações mercantilizadas nos trópicos.
  
\item \textbf{Douglas Rodrigues Barros} é um jovem escritor. Atualmente está concluindo doutorado
em filosofia. É ex"-operário, experiência que talhou profundamente sua escrita e
pensamento. Para homenagear Marighella, gosta de se referir a si como ``apenas um
mulato cearense'', apesar de entender que ``mulato'' é um substantivo carregado de
adjetivação racista. Publicou em 2016 \emph{Cartas estudantis} pela editora Multifoco e em
2017 \emph{Os terroristas} pela editora Urutau. 
%Nunca quis escrever sobre a questão racial, mas o fato da polícia assassinar 45 jovens pobres e negros da periferia em apenas 10 dias de operação, e isso ser algo corriqueiro nesse país, não lhe deu alternativas.

\end{itemize}