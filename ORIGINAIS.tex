\chapter*{}

\vfill
\begin{flushright}
\emph{Para Douglas Belchior e Adervaldo José dos Santos pelo~compromisso
com a luta antirracista.}
\end{flushright}
\thispagestyle{empty}

\chapter*{Apresentação}
\addcontentsline{toc}{chapter}{Apresentação, por \emph{Tales Ab'Saber}}

\begin{flushright}
\emph{Tales Ab'Sáber}
\end{flushright}

Douglas Rodrigues Barros é um escritor que atua tanto na esfera da
ficção e do romance quanto da teoria e do pensamento crítico. Com
formação em humanidades e filosofia na nova Universidade brasileira --
exatamente a que recebeu o influxo reparador social mínimo da tardia
política de cotas brasileira, hoje sobre o ataque degradante do
neo"-obscurantismo anticrítico e anticientífico que grita alto no Brasil,
configurando o real devir negro de toda uma instituição social e seus
sujeitos\ldots{} -- seu trabalho mantém constante contato, enriquecido pela
experiência e com a experiência da crítica, com o mundo popular e o
percurso histórico da classe trabalhadora urbana de São Paulo, da qual
não oculta fazer parte e se posiciona como narrador emancipado.

Sem perder o contato com as mazelas da violência brasileira sobre a vida
do trabalho, buscando investigar mentalidades e modos possíveis de
pensar as condições de existência históricas que são franqueadas à
classe em seus romances recentes, ele também assumiu e dedicou
importante trabalho ao excedente de violência e sentido patológico da
vida social que é o circuito de símbolos, práticas e subjetivações que
envolvem os descaminhos da clivagem racial, e sua própria racialização
da vida, estratégia torpe do poder que duplica e aprofunda a história de
recusas dos direitos negados dos pobres, dos pobres negros, em nosso
capitalismo, de origem colonial, escravista e antissocial.

Este livro, que também é um debate, apresenta o trabalho crítico do
autor junto a uma das suas comunidades políticas, pondo em questão as
hipóteses de fundo que movem ações do movimento negro contemporâneo no
Brasil e esclarecendo uma diferença significativa sobre o modo e a
dinâmica do pensamento desde uma perspectiva fundamentalmente crítica do
problema, ou da solução, negra. O livro também é a enunciação de uma
política da leitura das potências sociais e das energias conceituais que
envolvem a obra de um grande autor. Ao acatar com precisão o impacto do
trabalho teórico de um autor fundamental do campo crítico contemporâneo,
Frantz Fanon, que disparou um sem número de movimentos e modalidades de
engajamento em todo mundo pós"-colonial, e ao reafirmar a constante
instabilidade do seu trabalho forte pela superação de toda posição de
violência, implicada em seu desmonte dialético da racialização, que só
pode ser anticapitalista, Douglas Rodrigues Barros nos mostra como toda
uma tradição de compromisso intelectual e critica opera: aumentando a
energia radical dos conceitos para a transformação social necessária.

Confiando na universalidade negativa da razão, que deve pesar
igualmente, entre a configuração do objeto e a do próprio sujeito no
objeto, podemos observar como, para Fanon, de modo muito diferente das
fixações imaginárias e ``místicas'', como diz Douglas, de parte do
movimento antirracista definidor das políticas para os negros de hoje, a
fragilidade das falsificações do racismo branco ocidental implica a
crítica da falsa integridade da identidade negra, ela também realizada
em algum momento neste processo social de distorção e mistificação de
tudo.

Demonstrando com clareza, em conjunto com as intensidades subjetivas
pessoais do estilo, o modo radical de Fanon encaminhar seu pensamento,
constantemente em balanço e movimento que critica a subjetivação racista
situando a experiência negra como o outro negativo da falácia da
integridade branca, assim revelada, no mesmo movimento que critica o
apego de entificação fixada da própria condição negra, para também ser
livre dela, \emph{Lugar de negro, lugar de branco?} repõe em cena a ordem
moderna de uma razão em trabalho, única universalidade virtual que não
cede diante do terror.

Este trabalho sem pouso da crítica só pode se resolver após desmontar e
suspender as camadas de violências e dispositivos ideológicos que deram
destino à necropolítica colonial e a autoimagem alucinada de
superioridade da Europa branca moderna, bem como os seus efeitos
dialéticos na ideia moderna do negro. Assim, ao se suspender o lugar
histórico falsificado da violência branca, a sua autoimagem, deve"-se
chegar à suspensão do lugar determinado desde aí das violências
incorporadas à ideia de alguma identidade negra.

Seria este o projeto da superação histórica da violência racial, e seus
sujeitos e assujeitados, determinada pela própria história do vínculo de
capitalismo colonial escravista mercantil e a formação do presente. Sem
temer a vida crítica da razão, tal trabalho não teme a própria cor, e
seu lugar sem lugar na ordem branca, evitando qualquer virtual política
da paranoia, acentuando o horizonte de emancipação pós"-capitalista da
empreitada.

Assim, implicado humanamente e pessoalmente na intensidade do debate que
põe em cena, \emph{Lugar de negro, lugar de branco?} é contribuição
renovadora e esperançosa, sem concessões, para a desalienação mais
radical ao redor de uma questão premente da contemporaneidade dos
últimos 500 anos do processo de terror da expansão mundial do Capital.

\chapter{\emph{Uma conversa com o hipotético-leitor}}

Este curto ensaio foi atropelado pela \emph{ordem}. Quando escrevia a
parte final, me vi a amontoar tantos conceitos e pintá"-los tal como pede
nosso contexto, que fui derrubado pela terrível notícia do assassinato
de \index{Franco, Marielle}Marielle Franco e \index{Gomes, Anderson}Anderson Gomes, ocorrido em 14 de março de 2018,
data do nascimento de \index{Alves, Castro}Castro Alves, de \index{Jesus, Carolina Maria de}Carolina Maria de Jesus, de
\index{Nascimento, Abdias do}Abdias do Nascimento e, por contingência, do próprio autor.

Eu não conhecia \index{Franco, Marielle}Marielle Franco, porém seu assassinato foi sentido em
minha pele. Choramos sua morte. O seu sangue era o nosso. Erro seria
crer que o grande dissídio que nesses últimos tempos se estabeleceu no
seio da esquerda se assentasse numa diferença estratégica e que o
martírio de uma valorosa companheira pudesse fundamentar uma outra
prática que ousasse, se não reverter o descalabro do país, pelo menos
defender aqueles que se prontificam a estar em suas fileiras.

Desde então tudo mudou, mas nada saiu do lugar.

Seria muito importante que qualquer militante e crítico de esquerda não
fosse mais o mesmo depois dessa morte. As poucas ilusões com a política
oligárquica brasileira deveriam ser desfeitas pelos tiros dados em
\index{Franco, Marielle}Marielle e no seu motorista. É uma política de morte que funciona sob
pressupostos rentáveis e significante colonizado.

Nós sabíamos disso? Talvez, mas até então duvidávamos. Hoje não se pode
duvidar mais. Essas mortes, sem dúvida, têm o peso do nosso fracasso.
Devemos nos responsabilizar por elas e por outras tantas que ocorrem
longe da segunda maior cidade do país. Só assim poderemos dar um basta.

``Nós desconfiamos do entusiasmo'',\footnote{\versal{FANON}, F. \emph{Pele negra,
  máscaras brancas}. Tradução de Renato da Silveira. Salvador: \versal{EDUFBA},
  2008.} assim se expressa \index{Fanon, Frantz}Fanon na introdução de sua obra como quem
cita uma verdade lúcida desperta pelos sinais daqueles que não tiveram
irmandade com as coisas e foram esmagados por fileiras de carruagens
atadas às costas. Entusiasmar"-se é tornar"-se impotente.\footnote{O
  entusiasmo para \index{Fanon, Frantz}Fanon tem o mesmo sentido de catarse para Brecht, qual
  seja: impossibilidade de crítica dada as condições de aceitação
  presumidas nele.}

Com alguns farelos recentemente caídos da mesa da elite econômica,
durante um curto intervalo de tempo,\footnote{Trata"-se do tempo das vacas
  gordas em que a crise mundial propiciou um forte investimento e
  crescimento nas importações de \emph{commodities,} e o superávit
  primário passou de 3,7\% para 4,5\%. Com isso, houve a captura de
  grande parte da esquerda tanto material quanto espiritualmente.}
entretanto, a esquerda e grandes setores do \emph{movimento negro}
pareciam se entusiasmar e abandonaram qualquer princípio que não o de se
incluir no jogo.

Tornar"-se \emph{colaborador},\footnote{Colaborador como colaboracionista
  (remetendo aos anos hitleristas) implica executar um trabalho
  independente de suas sequelas. A esse respeito, mas não no sentido
  aqui exposto, \emph{Cf}. \index{Arantes, Paulo E.}\versal{ARANTES}, P. E. \emph{O novo tempo do mundo:} e
  outros estudos sobre a era da emergência. São Paulo: Boitempo, 2014.
  p. 101-140.} no entanto, tem um preço a ser pago: a elevação do mito
à verdade, o abandono e até a acusação contra qualquer posição que pense
para além das miudezas e misérias cotidianas sob a égide do mundo da
mercadoria.

Competiu, desse modo, a parte da esquerda -- ou melhor, à esquerda
hegemônica -- realizar o trabalho sujo com zelo:\footnote{Trabalho muito
  bem demonstrado por \index{Oliveira, Francisco de}Francisco de Oliveira. ``Sindicatos de
  trabalhadores do setor privado também já estão organizando seus
  próprios fundos de previdência complementar, na esteira daqueles das
  estatais. Ironicamente, foi assim que a Força Sindical conquistou o
  sindicato da então Siderúrgica Nacional, que era ligado à \versal{CUT},
  formando um ``clube de investimento'' para financiar a privatização da
  empresa; ninguém perguntou depois o que aconteceu com as ações dos
  trabalhadores, que ou viraram pó ou foram açambarcadas pelo grupo
  Vicunha, que controla a Siderúrgica. É isso que explica recentes
  convergências pragmáticas entre o \versal{PT} e o \versal{PSDB}, o aparente paradoxo de
  que o governo de Lula realiza o programa de \versal{FHC}, radicalizando"-o: não
  se trata de equívoco, mas de uma verdadeira nova classe social, que se
  estrutura sobre, de um lado, técnicos e intelectuais \emph{doublés} de
  banqueiros, núcleo duro do \versal{PSDB}, e operários transformados em
  operadores de fundos de previdência, núcleo duro do \versal{PT}. A identidade
  dos dois casos reside no controle do acesso aos fundos públicos, no
  conhecimento do `mapa da mina'". (\emph{Cf}. \versal{OLIVEIRA}, F. \emph{Crítica à
  razão dualista/O ornitorrinco}. São Paulo: Boitempo, 2003, p. 146.)}
se se matam milhares de jovens anualmente, a maioria negros, que
importa? Pensemos em nosso próximo candidato...

Na rede social, agora convertida em Areópago, desfilam mil gênios
consagrados às verdades provisórias em busca de \emph{likes} e
comentários. O Facebook, separando cada um no seu nicho próprio,
construiu a \emph{máquina do mundo} que procura não só a
\emph{implicação} como a \emph{mobilização total} de seus usuários. A
subordinação e organização ``do e para o'' trabalho agora passam por
essa ``ferramenta'', que alterou radicalmente as relações de
sociabilidade das pessoas comuns.

Por outro lado, ainda será necessário refletir sobre a insensibilidade
social e a invisibilidade do massacre cotidiano que se perpetua aqui
desde que o colonizador chegou.\footnote{O que em todo caso não faremos
  aqui.} Se antes a carne negra era a mais barata e rentável do
mercado, agora, é necessário dizimar o seu excesso. Entre passado e
presente, a infâmia que atende pelo nome de \emph{racismo}.

O passado, porém, é lição para se meditar, não para reproduzir, dizia
\index{Andrade, Mario de}Mario de Andrade,\footnote{\versal{ANDRADE}, M. \emph{Pauliceia desvairada}.
  Barueri: Ciranda Cultural, 2016, p. 18.} e conquanto não estejamos
dispostos a ruir sob os maus auspícios de um romantismo estéril, devemos
perguntar o que é ser negro atualmente, sem cair na cilada de uma
identificação remota com um passado inexistente.

A \emph{identificação} é a forma de ligação emocional mais profunda e,
com ela, dificilmente se ultrapassam as limitações que forjam a
experiência concreta na formação do \emph{eu} com o mundo.\footnote{\versal{FREUD}.
  S. \emph{Psicologia das massas e análise do eu}. Porto Alegre: \versal{LPM},
  2017.} É nessas lições de \index{Freud, Sigmund}Freud que temos um grande aprendizado sobre
o funcionamento da \emph{psicologia das massas.}
Psicologia que hoje em dia foi
capturada pelo gozo escópico do narcisismo autorreferenciado das redes
sociais.\footnote{Essa noção me foi passada pelo grandioso artigo de
  \index{Lemos, Patrícia do Prado Ferreira}Patrícia do Prado Ferreira Lemos: \emph{Entre o olho e o olhar:} o
  gozo escópico no Facebook.}

Sendo assim, o que governa a identificação é a simpatia que impõe não
somente a imitação de características em comum, como também sua defesa
acrítica. É pela característica afetiva da identificação que a ligação
mútua entre indivíduos da massa é produzida. Essa se encontra por vezes
numa qualidade particular em comum, numa cor em comum, num fenótipo em
comum e numa história mítica em comum.

É daí que a multiplicidade que constitui o \emph{eu} particular deixa de
importar: o que importa é aquilo no seu \emph{eu} exterior que se parece
comigo: um cabelo em comum, uma roupa em comum, um hábito em comum, por
fim, uma \emph{raça}.

Tanto a identificação com o branco quanto a identificação com o
\emph{negro} eliminam de si qualquer capacidade reflexiva mais profunda.
Ser igual no infortúnio ou no privilégio significa que alguma coisa
sustenta essa condição. Sabemos o que é.

Nesse sentido, \index{Fanon, Frantz}Fanon segue sendo o arsenal crítico contra a leviandade e
preguiça daqueles que falam em seu nome e que se tornaram reles
colaboracionistas.\footnote{Há diferenças entre os colaboracionistas e os
  francamente fascistas que podem ser resumidas grosseiramente da
  seguinte maneira: os fascistas faziam e sabiam o que faziam, tinham
  clareza das ordens que seguiam e onde queriam chegar, ao passo que os
  colaboracionistas só estavam tentando trabalhar e seguir a vida
  normalmente sem se envolver com algo para além do que o limite
  impunha, ou seja, ``eles faziam, mas não sabiam o que faziam''. Essa
  matéria pode ser melhor estudada no ensaio de \index{Arantes, Paulo E.}Paulo Arantes (2014, p.
  101-141) \emph{Sale Boulot} no livro já citado, ou ainda, em \index{Arendt, Hannah}Hannah
  Arendt (\emph{Cf}. \versal{ARENDT}, H. \emph{Eichmann em Jerusalém.} Um relato sobre a
  banalidade do mal. São Paulo: Cia das Letras, 1999.)} Isto impõe uma
reflexão crítica sobre a hegemonização de um determinado setor do
movimento negro que impôs uma pauta na qual alguns temas são francamente
reacionários. Atualmente, essa hegemonização impossibilita qualquer
horizonte para além das formas impostas socialmente pelo modo de
sociabilidade capitalista e, portanto, nosso arsenal se voltará contra
essa mesma hegemonia.

Urge imaginarmos um mundo no qual a mercadoria não dê as referências
vitais, urge lutarmos por outras formas de sociabilidade, urge criarmos
outra dinâmica de vida em que o componente racial não seja decisivo na
escolha de quem deve morrer. A utopia não é acreditar que o capitalismo,
já apodrecido, necessita morrer, utopia é acreditar que um dia sua forma
de reprodução social deixará de criar identificações, espaços demarcados
de sociabilidade mediados pelo dinheiro, pela mercadoria, e deixará de
fomentar o racismo, que lhe é
constitutivo e inerente.

Este curto ensaio foi atropelado pela \emph{ordem}, mas não esmoreceu
diante daquilo que era sua tarefa: desiludir"-nos com os misticismos para
enfrentar a barbárie imposta.

É o que tenho a oferecer como trabalho de luto. 
Um trabalho que espero possibilite o amadurecimento de nossa luta.
\begin{flushright} 
\emph{O autor}
\end{flushright}

\partepigraph{Eu lhe direi: é o meio, é a sociedade que é responsável pela sua
mistificação. Isso dito, o resto virá por si só. E sabemos do que se
trata. Do fim do mundo. (\index{Fanon, Frantz}\emph{Fanon})}{}
\part{FANON CONTRA O MISTICISMO}
\removeepigraph

\chapter*{Nem Casa Grande\\ Nem~senzala}
\addcontentsline{toc}{chapter}{Nem Casa Grande, nem senzala}

Seria absurdo dizer que as condições em que se conduziu a racialização
no Brasil não foram determinantes para a produção e reprodução do
capital e, com ele, sua moderna divisão do trabalho. À parte o
misticismo da diferença racial, que imprimiu nos corpos negros uma
tentativa de subordinação e que faz dessa diferença a inconfessável
política de exceção naturalizada em todas as cabeças brasileiras, existe
essa crença, essa crença pegajosa em relação à cor, em relação a todo um
continente epistêmico criado para \emph{nadificar} aqueles que sempre
estiveram no interior da produção, produzindo riquezas, mas foram dela
alijados.

Foi a criação de um consenso -- na ciência, na filosofia, na arte, na
espiritualidade ou na religião -- como linha divisória e prática, a
partir da qual se nutrem as formas de organização do Estado como poder
soberano sobre a decisão de morte.\footnote{\index{Mbembe, Achille}\versal{MBEMBE}, A. \emph{Necropolítica}. São Paulo: \versal{N}-1 Edições, 2017.} Uma sacada pragmática que convoca a onipresença da segurança estatal, o aparato
despótico que marcará, de uma vez por todas, o inimigo a ser combatido.
Assim caminha a grande festa da República, cautelosa, naturalizada e
distorcida, que nunca termina de contorcer o chamado ``direito
democrático'' e, a cada segundo, se vê ameaçada por alguma nova
``diferença'' por ela mesma superficialmente criada.

Se aquele ``animal preto, que possui lã sobre a cabeça, caminha sobre
duas patas'',\footnote{\versal{VOLTAIRE}. \emph{Tratado de metafísica}. São Paulo:
  Abril, 1978, p. 62.} como dizia o cafona do \index{Voltaire}Voltaire, encontrou agora
a possibilidade de se repensar a si mesmo, é porque abandonou o corpo
debilitado e os ícones do ressentimento em que se apegava. Longe de
qualquer movimento fascistóide que repinta orgulhoso o brasão da
racialidade, o corpo negro, marcado e identificado como inimigo, se
ergue na luta que ``sempre teve
como meta a abertura para um mundo verdadeiramente comum''.\footnote{\index{Mbembe, Achille}\versal{MBEMBE},
  A. \emph{Crítica da razão negra}. Lisboa: Antígona, 2015, p. 297.}
Que ele tenha se erguido, com esforço, ao posto de reflexividade, de
mediação, de consciência"-de"-si, muda tudo. Não é o apego por um mundo
putrefato em vias de ser ultrapassado que lhe dá ânimo, mas sim o vigor
da experimentação, de um olhar em que do \emph{não"-ser} possa brotar o
\emph{novo}. Fora daquela ilusão socialmente necessária, marcada por um
conservadorismo de verniz nacionalista, se distingue outro tipo de
posição política em que um acontecimento de verdade está prestes a
abrochar e advir como efetividade. É ele que estabelece a continuidade
existencial, nem secreta, nem óbvia, dos corpos negros como portadores
de uma comunidade efetiva e vindoura. Que sela o destino do
proletariado, ou melhor, retoma seu sentido clássico; não como redutora
classe operária, mas como a universal classe dos \emph{Condenados da
terra}.

A engrenagem trepida\ldots{} A bruma
neva\ldots{} pouco importa que os corpos reduzidos à diferença racial
encontrem ou não motivos conscientes para resistir à \emph{ordem}
existente; vemos claramente que os corpos reduzidos pela racialidade se
tornaram um excedente populacional que em fins de capitalismo tardio
precisa ser identificado, controlado, categorizado, separado, cercado
por muros, por Unidades de Polícia Pacificadoras e, por fim, extirpado.
O mesmo odor de merda que exala da Casa Grande em seu hálito mortal e
necropolítico! O controle dos corpos negros, ou mais além, dos corpos
indomáveis, certamente não é uma vã medida coercitiva da vida em suas
limitações burguesas. É a luta incessante do aparato repressor do
Estado/Capital na vã tentativa de impedir o desejo de criar algo
vertiginoso. Quando a dialética Casa Grande/Senzala finalmente for
suprimida, então com ela serão suprimidas as diferenças raciais
superficialmente criadas para controle universal dos corpos e
territorial dos espaços.

\chapter{O não-ser que é}

Como uma ficção ganha força material, molda a apreensão de um mundo e
torna"-se motor do real? Como a realidade reduzida à aparência torna"-se
ela própria só o aparente que por trás de si nada oculta senão seu vazio
constitutivo? Se ``os adjetivos passam e os substantivos
ficam'',\footnote{\index{Assis, Machado de}Machado de Assis, Diário do Rio de Janeiro, \emph{Balas de estalo}, 16/5/1885.} o adjetivo \emph{negro} dado aos
indivíduos que viviam no antigamente conhecido ``continente
sombrio'',\footnote{Para os antigos, a África era uma espécie de porta
  para o mundo dos mortos.} rapidamente foi substantivado e
não apenas se limitou a substituir o real pelo aparente, como
desmistificou compulsoriamente e demonstrou que o real é a própria
aparência.

A visão de conjunto sobre essas operações mágicas em curso
histórico"-filosófico permitiu considerarmos que \emph{ser} e
\emph{não"-ser} são complementos do real, como diria o velho \index{Sartre, Jean-Paul}Sartre, ``à
maneira da sombra e da luz''.\footnote{\versal{SARTRE}, J. P. \emph{O ser e o
  nada}: ensaio de ontologia fenomenológica. Petrópolis, \versal{RJ}: Vozes,
  2009, p. 53.} Não há movimento conceitual sem uma linguagem capaz de
exprimir, simultaneamente, a condição histórica e sua fissura
constitutiva. Então, se concordamos com \index{Mbembe, Achille}Mbembe que a grande violência
colonialista foi ter reduzido o indivíduo africano ao adjetivo
\emph{negro}, isto é, ter reduzido o indivíduo à aparência, igualmente
concordamos que a aparência é o próprio motor que subjaz num espírito
científico guiado pelo Entendimento e suas limitações
empírico"-descritivas. A crítica \emph{à invenção do Negro} impõe esse
ponto de partida.

Ainda levará um tempo para que nós compreendamos como essa posição do
negro -- como \emph{não"-ser} que \emph{é} -- estrutura não só a
realidade existencial do branco como ainda possibilita a abertura para o
advento do \emph{novo}. Antes disso, porém, se faz necessário elucidar,
na base de um bom combate epistemológico, como o pensamento
moderno/científico foi responsável pela construção fictícia das raças
que, no íntimo de uma abstração real guiada pela valorização do capital,
sustentou e deu legitimidade, pelo discurso de ``verdade'', a um modo de
sociabilidade exploradora e predatória que atende pelo nome de
capitalismo.

\chapter{Entre Luzes e Sombras}

Toda consciência ocidental está reunida neste ideário: ``o entendimento
que vence a superstição deve imperar sobre a natureza desencantada''.
Reside na incrível estranheza do homem ocidental em relação ao novo
mundo, estranheza que exige, pelo poder da ciência, que ele se faça dono
dos indivíduos e controlador da natureza, o impulso temerário por aquilo
que não se limita às limitações de sua sociedade e de seus costumes. Ao
se subtrair ao existente, a investida científica fornecida pelas Luzes,
o ilustrado não desfez a fobia ante os inexplicáveis fenômenos, mas
passou a descrevê"-los e desvirtuá"-los para melhor controlá"-los:
``O saber que é poder não conhece
barreira alguma, nem na escravização da criatura, nem na complacência em
face dos senhores do mundo.''\footnote{\index{Adorno, Theodor W.}\versal{ADORNO} e \index{Horkheimer, Max}\versal{HORKHEIMER}.
  \emph{Dialética do esclarecimento}: fragmentos filosóficos. Rio de
  Janeiro: Zahar, 2006, p. 18.}

É, portanto, a fobia o principal componente dessa proto"-identidade
europeia. Esse perfil sombrio, mecânico, hostil, absurdo, tentou, de
forma vulgar, encantar seu processo de dominação sobre os povos
colonizados por meio da \emph{tolerância}.
É óbvio que esse conceito oculta
uma normatividade da apropriação do excedente, enquanto busca o controle
da margem admissível em relação à própria medida europeia. Quando
\index{Diderot, Denis}Diderot, \index{Buffon, Conde de}Buffon e \index{Voltaire}Voltaire buscaram afastar"-se da metafísica apenas descrevendo os fenômenos ``naturais'', acabaram por instituir, ainda que
a contragosto, uma das mais admiráveis ilusões: a neutralidade
científica como componente de dominação.

\chapter{A patologia da identidade}

A autoficção da identidade europeia -- uma autocontemplação de si mesmo
que enclausura a potencialidade da \emph{diferença} -- tornou então
hiperidentificatório o significado de \emph{Negro}.

\begin{verse}
O branco é o símbolo da divindade ou de Deus.\\
O negro é o símbolo do espírito do mal e do demônio.\\
O branco é o símbolo da luz\ldots{}\\
O negro é o símbolo das trevas, e as trevas exprimem \qb{}simbolicamente o mal.\\
O branco é o emblema da harmonia.\\
O negro, o emblema do Caos.\footnote{\index{Montabert, Jacques-Nicolas Paillot De}\versal{MONTABERT}, J-N. P. \emph{Traité complet de la peinture}, 9 vols (Paris: J.-F. Delion, 1829–51), vol. 7, pp. 422--3. Kératry, Annuaire de l'École.}
\end{verse}

A novidade de uma condição alheia àquele mundo, condição que durava mais
do que a imaginação europeia poderia supor(tar) e da qual não se
conseguiu apreender toda a significação -- que ultrapassa o espoliar dos
povos e impulsiona a acumulação primitiva do capital --, ameaçou desde
sempre e repetidamente a consolidação do prosaico mundo burguês. Houve,
a princípio, a fantasia selvagem das grandes narrativas de aventureiros
que foram lidas vorazmente nas alcovas requintadas da decadente nobreza.

Os europeus, cuja educação e civilidade nunca combinaram com
neutralidade epistêmica, não podiam então acreditar que uma organização
social vinda das selvas entranhadas no ``Continente noturno'' pudesse
amadurecer a ponto de atingir qualquer noção de liberdade. Com todo o
rancor de sua relação normativa com a vida, eles se sentiam vivamente
atingidos; os costumes narrados dos povos de África eram distorcidos e
retorcidos: comiam carne crua, bebiam demasiadamente e faziam sexo à
vontade. Um horror para a imaginação europeia que perdura até hoje!

Sob o signo do exótico, a aparição do termo \emph{Negro} no dicionário
moderno foi paralela a um projeto de conhecimento e de governança que se
instaura a partir do desenvolvimento da própria modernidade. \emph{Raça}
e \emph{Negro}, produzidos pelo advento do moderno nas suas formas de
controle e segregação fazem parte de um delírio manipulatório, como
evidencia \index{Mbembe, Achille}Mbembe.

Que o signo \emph{raça} guarde suas contradições inerentes impondo uma
dialética forçosa não nos é algo indiferente; que a dialética do
Esclarecimento, fechada sob as limitações do capital, tenha criado suas
áreas sombrias, ou melhor, negras, com ``a presença da dominação dentro
do próprio pensamento como natureza não reconciliada''\footnote{\index{Adorno, Theodor W.}\versal{ADORNO} e
  \index{Horkheimer, Max}\versal{HORKHEIMER}. \emph{Dialética do esclarecimento:} fragmentos
  filosóficos. Rio de Janeiro: Zahar, 2006, p. 45.} é algo sentido nos
próprios corpos; que a denominação \emph{Negro} passou a traduzir ``o
ser"-outro fortemente trabalhado pelo vazio, e cujo negativo acabava por
penetrar todos os momentos da existência'',\footnote{\versal{MBEMBE}, A.
  \emph{Crítica da razão negra}. Lisboa: Antígona, 2015, p. 32.} é algo
ainda por ser aprofundado. \index{Mbembe, Achille}Mbembe, que é inigualável em demonstrar o que
importa, enterrou a noção de igualdade europeia com uma impactante
fórmula: ``o negro não existe enquanto tal. É constantemente produzido.
Produzir o negro é produzir um vínculo social de submissão e um corpo de
exploração''.\footnote{\versal{MBEMBE}, \emph{ibidem}, p. 33.}
Também \index{Fanon, Frantz}Fanon em algum ponto de
\emph{Peles negras} é categórico: é o branco que cria o negro. Em todo
caso o fundamento da identidade \emph{negra} foi concomitante com o
delírio narcísico branco e europeu.

Ora, se \index{Diderot, Denis}Diderot foi um digno mediador do que a situação da nova ideia de
homem continha de universalista e desafiadora, por outro lado, \index{Voltaire}Voltaire,
mas não só ele, introduziu nas \emph{Luzes} a particularização
identificatória que estamos buscando: a divisão de subespécies de
homens. Por subespécies, o ilustrado designava o princípio da identidade
e o da diferenciação, em que as desigualdades de desenvolvimentos seriam
determinadas entre si pela natureza das supostas espécies humanas. Se a
descrição substitui a explicação por medo compulsivo da metafísica,
bastava então descrever as estruturas corporais das diferentes
subespécies para empreender sua divisão e apreender suas diferenças
constitutivas que reverberavam no ``progresso'' de seu desenvolvimento.

A ficção útil do termo \emph{raça} e do termo \emph{negro} será lenha na
fogueira das animadas controvérsias entre monogenistas e
poligenistas\footnote{\index{Santos, Gislene Aparecida dos}Segundo Gislene Aparecida dos Santos: ``[\ldots{}]
  os monogenistas continuam apoiando"-se nos argumentos climáticos,
  geográficos, culturais para explicar as diferenças entre os homens e
  os poligenistas, remetendo"-se às origens separadas''. \versal{SANTOS}, G. A.
  \emph{A invenção do ser negro:} um percurso das ideias que
  naturalizaram a inferioridade dos negros. Rio de Janeiro: Pallas,
  2006, p. 47.} que ainda têm no séc. \versal{XVIII} um horizonte de busca da
igualdade. No alvorecer de 1800, com a entrada da noção de evolução
darwinista na discussão entre as diferentes escolas, contraditoriamente,
os significantes \emph{negro} e \emph{raça} tornam"-se a"-históricos e
imutáveis, fomentando a justificativa da exploração e das desigualdades
imperantes no então recém"-nascido modo de sociabilidade capitalista,
momento em que a realidade racial superou a igualdade cidadã do direito.

Funda"-se aí a produção do Negro, cujo vinculo social de submissão e
exploração lhe é inerente. O negro torna"-se um corpo no qual se realizou
a mais absoluta violência expropriadora; certamente, isto é o que
fundamenta a visão de mundo, não só das vítimas dessa redução
ontológica,\footnote{Ainda como diz \index{Mbembe, Achille}Mbembe (\emph{op. cit.}, p. 39): ``O negro
  não existe enquanto tal. É constantemente produzido. Produzir o negro
  é produzir um vínculo social de submissão e um corpo de que nós
  chamamos de estado de raça corresponde, assim o cremos, a um estado de
  degradação ontológica''.} como do corpo social diferenciado por
este significante identitário redutor. Em nenhum momento pode"-se afirmar
que essa redução de indivíduos ao corpo é uma anomalia pertencente ao
passado, mas é, sim, a força motora oculta da modernidade capitalista, o
lugar do espaço sócio"-político no qual ainda vivemos e que produz
diferenciação racial e muros asseguradores \emph{nacionais}.

O delírio da diferenciação racial encontraria então na \emph{Lettre à
madame de Graffigny} de \index{Turgot, Anne Robert Jacques}Turgot a justificação plausível da exploração
europeia: ``a desigualdade social radical se inicia na natureza física desigual dos `diferentes'
humanos.'' Sendo o branco europeu mais próximo da
racionalidade e desenvolvido tecnicamente, naturalmente sua posição
seria a de comando. O delírio seria assim um exercício de expurgar as
ações passionais e encontrar uma desculpa para o preenchimento de um
vazio constitutivo. Um vazio que constituiu a modernidade e sua
exploração capitalista. O negro, portanto, é colocado nesse não"-lugar do
delírio, que ora tem lastro de libertação libidinal, ora de regressão
violenta.

\emph{Negro} e \emph{raça} constituem assim os polos convergentes de um
mesmo delírio europeu: a redução do corpo e do ser vivo a uma questão de
aparência. Como tal, elimina"-se de si a noção de reconhecimento das
diferenças como constitutivo do Eu europeu; há, portanto, aquilo que
\index{Mbembe, Achille}Mbembe chamou de \emph{alterocídio}, ou seja, o Outro como objeto
ameaçador que precisa ser extinguido.

\chapter{É a ciência, estúpido!}

Nascia, em um mesmo movimento, a concepção de \emph{raça}, ainda não de
todo determinada, e a noção esvaziada -- e por isso ideológica -- de
\emph{neutralidade científica}. O reverso da moeda iluminista --
enriquecida com o comércio negreiro, com a colônia de \emph{plantation}
-- apresenta o progresso que alguns homens (dominantes) teriam realizado
em relação aos inferiores (dominados). Estamos diante da fase
embrionária da auto"-identificação alucinatória que encontraria sua
verdade na práxis colonialista; a espécie humana torna"-se doravante um
corpo fragmentado.

A razão, sobrepondo"-se à hierarquia, é aquela que, paradoxalmente,
justificará essa mesma hierarquia por meio da diferença dos corpos; a
autoidentificação alucinatória europeia tornou o \emph{nós} reduzido à
territorialidade e a partir daí a universalidade tornou"-se,
paradoxalmente, restrita. O delírio da diferença que serviria para
justificar as teorias racialistas acabará por ser formulado por aquilo
que agia silenciosamente por trás das costas dos bons teóricos; a
colonização que permitiu a riqueza europeia precisaria também encontrar
justificação espiritual.

Foi \index{Hegel, G. W. Friedrich}Hegel quem demonstrou esse cinismo constitutivo do paradoxo
formulado pela era das Luzes; preparado o cenário hierárquico -- a
subespécie humana -- faltava apenas a máscara menos inadequada de
desresponsabilização tensionado pelo modo de exploração radical da
escravização dos povos não"-europeus. Foi quando o discurso racional
moderno, ao mesmo tempo que previa uma universalização ideal, fomentou a
diferenciação hierárquica concreta através da construção do significante
\emph{raça}. Foi quando a patologia da identidade foi inaugurada para o
controle dos corpos excedentes. Foi quando se percebeu que o Bentinho de
\index{Assis, Machado de}Machado de Assis, com sua violência e frivolidade, em todo caso
ilustrada, não é monopólio só da elite brasileira, mas uma figura
universal das teses morais da Europa. Foi quando ``a epistemologia,
campo filosófico profundo da justificação da ética tecnocrática, se
sobressaiu e representou, no discurso ideológico, a unidimensionalidade
da atividade científica''.\footnote{Devo essa noção a \index{Silva, Luiz Ben Hassanal Machado da}Luiz Ben Hassanal
  Machado da Silva.}

Paradoxalmente, o novo discurso da ``verdade'' científica foi o que
possibilitou amalgamar sob o signo da liberdade a utilidade prática da
mão"-de"-obra livre e suas benesses se comparadas com o trabalho executado
``por escravos boçais e preguiçosos''. ``Eu desejaria'', diz
candidamente \index{Bonifácio, José}José Bonifácio, ``para bem seu, que os possuidores de
grandes escravaturas conhecessem que a proibição do tráfico de carne
humana os fará mais ricos''.\footnote{\index{Sousa, Otávio Tarquínio de}\versal{SOUSA}, O. T. \emph{O pensamento
  vivo de \index{Bonifácio, José}José Bonifácio}. São Paulo: Martins Fontes, 1965, p. 37.} Em
todo caso a diferença radical já havia sido produzida e naturalizada.

\chapter{A permanência de uma noção}

Quatrocentos anos depois, quando eles já celebravam o esquecimento de um
passado, no qual ``o tempero do mar foi lágrima de preto'',\footnote{\versal{EMICIDA}. \emph{Boa Esperança}. São Paulo: Sobre Crianças, Quadris, Pesadelos e Lições de Casa\ldots{}, 2015.}
e quando seus elementos dissociativos mais definidos pareciam se
integrar suavemente às instituições democráticas, novos ecos do mesmo
racismo chegavam com a morte de refugiados em suas praias.

Os ecos do racismo mostram"-se
agora mais confusos, às vezes porque a pacificação prometida já não dava
conta da nova guerra em que, todavia, estavam envolvidos, e também
porque a nova espoliação prometida para a manutenção da vida capitalista
impunha uma nova condição à maior parte da população do globo, isto é, a
condição negra, ou aquilo que \index{Mbembe, Achille}Mbembe certeiramente chamou de \emph{o
devir negro do mundo}. Erguíamos um pouco a cabeça por curiosidade, para
depois apreender nossa repetida formulação social que passa,
decididamente, por uma contínua acumulação primitiva violenta dos povos
ditos ``inferiores'' e antidemocráticos que no lado sul do globo
continuam a se revoltar quando as áreas centrais do capital já comemoram
o \emph{fim da história}.

Voltamos então a assumir uma
posição de denunciar as raízes da dita ``inferioridade'', a criação da
racialidade, sua verdade no racismo e outras verdades da ``velha
filosofia''. Assim, deixamos de evitar o olhar que, na Ilustração,
desfaz o enigma ao apontar que as Luzes não ocultam mais suas sombras
constitutivas. Debaixo do terrível axioma de \emph{tolerância},
promovido pelo pedante \index{Voltaire}Voltaire, já se ocultava toda a impossibilidade
de reconhecimento dado aos ``povos primitivos''. ``Só tolero aquilo que
não reconheço como parte de mim'', e assim os ilustrados forjaram uma
noção de homem restrito aos limites europeus.

Certamente a emergência dessa descoberta conferiria aos ilustrados, de
hoje e de ontem, um sentimento desagradável de história imediata.
Resguardando a beleza da tolerância, e da razão como meio de sua
prática, se confirmava, então, a certeza de que essa ``descoberta'' se
podia guardar longe das vistas, principalmente se se pudesse manter um
pensamento da envergadura do de \index{Fanon, Frantz}Fanon distante das academias latinas.

Todavia, em troca de um reconhecimento institucional incerto -- incerto
em virtude da impossibilidade lógica de uma desidentificação no interior
do capitalismo -- os abolicionistas e, posteriormente, os defensores dos
direitos humanos se engajaram em garantir a paz social ao Capital
fundamentando um direito à exclusão, isto é, o direito que já não
garante ``direitos''. ``O direito é, portanto, neste caso, uma maneira
de fundar juridicamente uma certa ideia de Humanidade enquanto estiver
dividida entre uma raça de conquistadores e uma raça de
servos.''\footnote{\index{Mbembe, Achille}\versal{MBEMBE}, \emph{op. cit.}, p. 111.}

A cada época, a forma de aparição do corpo negro se redefine em função
da configuração geral da dominação, e, enquanto persistirem os
fundamentos socioeconômicos que criaram as diferenças raciais não se
superará o significante redutor. O estado de degradação do termo
\emph{raça,} que hoje recai atualmente sobre os palestinos e muçulmanos,
já adentra o espaço territorial europeu colocando o excedente de
desempregados na condição negra; o axioma é o seguinte: é preciso
identificar, taxar e controlar todos e, principalmente, o refugiado.

Os antropólogos discutem se a primeira aparição do termo \emph{negro}
deve ser identificada no Ocidente Clássico, ou deve ser investigada a
partir da fundamentação da divisão da espécie humana pela raça
recentemente criada do ponto de vista histórico: o que importa, aqui, é
que em ambos os casos se trata de um olhar retroativo que visa desnudar
a fundamentação de uma diferença utilizada para a organização estatal,
cuja forma de controle tornou"-se um poder soberano capaz de decidir quem
vive e quem morre. Isto é, cuja decisão de quem tem ou não o status de
cidadão passa pela cor ou pelo costume. Quem pode decidir que o status
de cidadão passa pela cor ou pelo costume?

Ora, se a questão é assim formulada, sua resposta só pode indicar que é
no berço do capital, nas relações coloniais e imperialistas, que a
diferenciação de raça se tornou possível; caso contrário, haveria o
risco de cair num essencialismo banal. A mais lamentável confusão em
relação a isso diz respeito ao fetiche identificatório que racializa a
política no momento em que novamente o regresso às interpretações
biológicas visam fundamentar distinções raciais e explicar os atrasos
econômicos de países subdesenvolvidos. É contra esse pano de fundo que a
centralidade nas ideias de \index{Fanon, Frantz}Fanon se apresenta.

\chapter{Linguagem e identidade}

Fica evidente que investigar \index{Fanon, Frantz}Fanon é colocar"-se um problema importante:
O que o presente significa para \index{Fanon, Frantz}Fanon?\footnote{Utilizamos aqui a mesma
  ironia de \index{Adorno, Theodor W.}Adorno na abertura de seu ``Skoteinos'' que inverte a questão: o
  que \index{Hegel, G. W. Friedrich}Hegel significa para o presente para o que o presente significa
  para Hegel. \versal{ADORNO}, T. W. \emph{Três estudos sobre Hegel}. São Paulo:
  Editora Unesp, 2013, p. 71.} A linguagem é o \emph{leitmotiv} de suas
descobertas. Atribuir importância à linguagem em sua relação com a
\emph{psique} implica se pôr sob a perspectiva socialmente construída e
desconsiderar o critério de refletir sobre as limitações que as formas
da linguagem impõem, especialmente por ser a linguagem formadora da
consciência.

Assentar esse problema implica tomar posição. Não deve importar a
originalidade dos novos termos -- conceitos de última hora fundados na
``conjuntura'' dos últimos dias, nem as ``novas epistemologias'' que não
criticam profundamente aquilo que possibilita a manutenção do racismo e,
principalmente, possibilita o rebaixamento da teoria em nome de um
suposto cálculo de acesso ao texto por meio de uma linguagem
publicitária.

É preciso dizer que o campo do saber contém em si um elemento de poder
e, portanto, de disputa. Essa disputa, para nós -- críticos e militantes
antirracistas --, não deve ser a da tentativa de tornar a teoria
popular, senão a de fazer com que o popular se torne teórico. Basta de
\emph{petit"-nègre},\footnote{\emph{Petit"-nègre} é um modo supostamente
  ``cordial'' e redutor da subjetividade do indivíduo negro analisado
  por \index{Fanon, Frantz}Fanon, que faz uma ampla crítica ao rebaixamento da linguagem por
  parte dos colonizadores que inconscientemente utilizam diminutivos,
  gírias e mutações na própria maneira de falar para se tornarem
  inteligíveis ao negro. Naturalmente, essa é uma forma de reduzir o
  negro, como se para ele houvesse um empecilho natural, ligado à sua
  biologia e ao caráter, para entender conceitos e palavras não
  recorrentes no seu vocabulário supostamente infantil. A velha frase
  tornada senso comum durante boa parte do século \versal{XX}: ``por ser um
  negro, você escreve/fala muito bem'' é também naturalizada pelos
  indivíduos negros que tomam o \emph{petit"-nègre} como um componente
  próprio. Assim, ao invés de se pôr alheio à linguagem da metrópole, ou
  do Império -- fugas em gírias e particularismos locais -- \index{Fanon, Frantz}Fanon se
  lança ao cerne dessa linguagem para demonstrar os mecanismos de
  dominação existentes nas teorias e ao demonstrá"-los expor sua
  limitação e a subversão possível que daí advêm.} como se não fossemos
capazes de entender \index{Kant, Immanuel}Kant\ldots{}

Ora, entender a posição de \emph{estruturação evanescente} da identidade
implica apreender o modo pelo qual ela se desdobra historicamente em sua
relação com a linguagem e o solo social que a produz. De saída, a
\emph{ficção} daquele \emph{Eu=Eu} estanque perpassa uma inumerabilidade
de coisas e \emph{Eus}. E é então, a partir dessa negatividade ao Eu=Eu
que, por fim, o Eu pode pôr"-se a si e se conhecer provisoriamente. Esta
posição advém do momento em que uma consciência se vê negada por outra
consciência igualmente determinada. Dá"-se a luta entre iguais que acaba
por fundamentar uma desigualdade na relação, ao mesmo tempo que presume
uma igualdade buscada como fundamento social em contraposição à
desigualdade objetivada tanto simbolicamente quanto
socialmente.\footnote{Fazendo grosso modo uma arqueologia dessa noção de
  incompletude do Eu=Eu, podemos observar que essa forma de abordar a
  formação da individualidade, do sujeito moderno, acompanha a
  fundamentação da própria modernidade. Já está presente no
  \emph{cogito} cartesiano, e seu desdobramento passa por \index{Kant, Immanuel}Kant, \index{Fichte, Johann Gottlieb}Fichte e atinge seu ponto de fundamentação social em \index{Hegel, G. W. Friedrich}Hegel, quando analisa a
  assim chamada dialética do Senhor e Escravo, demonstrando as relações
  de dominação e poder envoltas na luta entre as consciências. Mais
  tarde essa posição teórica será absorvida, com suas diferenças
  constitutivas, pela psicanálise freudiana. que capta a impossibilidade
  de completude do sujeito moderno. Na França de \index{Fanon, Frantz}Fanon, teremos um amplo
  debate sobre essa forma de abordagem da individualidade no
  existencialismo de \index{Sartre, Jean-Paul}Sartre, que influenciará sobremaneira a teoria
  psicanalítica do próprio \index{Fanon, Frantz}Fanon. Por isso, remontei à fundamentação da
  individualidade aqui para também questionar o mito sociológico segundo
  o qual houve um momento do século \versal{XX} em que a identidade pôde repousar
  numa identificação com o Estado nacional, ou com a classe. Ora, a meu
  ver, o que mantinha um afastamento da noção de identidade não era o
  pertencimento ao Estado Nacional, ou a identidade de classe -- mesmo
  porque a classe é uma desidentificação subjetiva que propicia uma
  identificação objetiva guiada por um objetivo em comum, que nada tem
  em comum com os anseios de uma identidade.}

Mas, e quando a posição dessa desigualdade dada pelo Outro não é sequer
formulada? Quer dizer, quando não há sequer uma estruturação da disputa
entre um Eu e o Outro? Demarca"-se com isso uma limitação que impõe uma
invisibilidade, um não reconhecimento peremptório de um não"-Outro
inexistente, \emph{um nada}. É exatamente por isso que a elevação da
identidade não"-relacional implica uma subordinação colonizada: porque
elide as contradições da ficção de um Eu sempre em transformação e se
coloca no elemento conservador de um olhar hegemônico que ignora e
coisifica aquele que não é tido como um \emph{Outro}. O negro tornou"-se,
dessa maneira, \emph{o nada}.\footnote{Como disse \index{Mbembe, Achille}Mbembe: ``ao reduzir o
  corpo e o ser vivo a uma questão de aparência, de pele ou de cor,
  outorgando à pele e à cor o estatuto de uma ficção de cariz biológico,
  os mundos euro"-americanos em particular fizeram do Negro e da raça
  duas versões de uma única e mesma figura, a da loucura codificada'',
  em que valham os pesos dessa construção simbólica, a construção de
  ``raça'' possibilitou inúmeras catástrofes durante a modernidade (\emph{Cf}.
  \index{Mbembe, Achille}\versal{MBEMBE}, A. \emph{Crítica à razão negra}. Lisboa: Antígona, 2014, p. 11).}
Isso porque a condição própria da existência se efetiva enquanto um
olhar do Outro e uma compreensão entregue a nós por esse Outro, que
ocorre no chão social.

A construção da identidade é com efeito um processo em devir, algo que
jamais pode atingir um ponto de estabilidade; torna"-se uma tarefa
instável que, possibilitando questionar os hábitos e tradições, pode
também chafurdar no lamaçal da identificação estanque e narcísica. O que
queremos salientar com isso é que aquela identidade unificada e estável
é ilusória, patológica; o processo de identificação, por meio do qual se
firma uma suposta identidade cultural, nunca repousou na falta de
movimento, senão na consequente mudança ininterrupta do Ser com relação
ao mundo e ao Outro.\footnote{Para concluir com \index{Hall, Stuart}Stuart Hall: ``A
  identidade plenamente unificada, completa, segura e coerente é uma
  fantasia'' (\emph{Cf}. \versal{HALL}, S. \emph{A identidade cultural na
  pós"-modernidade}. Rio de Janeiro: \versal{DP}\&\versal{A}, 2011, p.13.)}

Nascidas como ficção, dizia \index{Bauman, Zygmunt}Bauman,\footnote{\versal{BAUMAN}, Z.
  \emph{Identidade:} entrevista a Benedetto Vecchi. Rio de Janeiro:
  Zahar, 2005.} as contradições inerentes da identidade ganham,
contudo, grande relevância quando a sensação de \emph{pertencimento},
seja o de uma classe ou o de um Estado"-nação, entrou em declínio.
Naturalmente essa noção elimina da fundamentação social a possibilidade
de análise crítica; é como se houvesse um determinado momento histórico
na modernidade em que uma harmonia do sujeito consigo mesmo fosse
possível e fornecida pelas condições de um Estado forte, isto é, dos
trinta gloriosos anos de \emph{Welfare State}, o que em todo caso é uma
quimera. Claro que esta posição mais reforça o mito do que o desvenda.
Desde que a modernidade se instaurou com o capital gestado na
escravidão, tudo que era sólido e seguro se dissolve no ar. A própria
noção de sujeito traz consigo a incompletude que reforça o movimento
social da mercadoria. Tanto é assim que um dos nossos primeiros sujeitos
plenamente modernos é Dom Quixote!

Essa crise que joga os indivíduos na busca de uma identidade imóvel,
portanto, é social, denota como as estruturas simbólicas que sustentavam
um discurso sobre o Eu entram em colapso a partir do momento em que um
sentido de formação social -- a modernidade -- entra igualmente em
colapso.\footnote{\index{Harvey, David}\versal{HARVEY}, D. \emph{The condition of post"-modernity}.
  Oxford: Oxford University Press, 1989.} Noutras palavras, é como se o
entrave na circulação de mercadorias fundada por uma crise permanente
impossibilitasse esse discurso de um sujeito capaz de guiar o próprio
destino.

As lições que se tiram disso são: 

\begin{enumerate}[label=\alph*)]
\item
No mundo dominado pela sociabilidade capitalista busca"-se reduzir o Eu a um produtor/consumidor
de mercadorias; 
\item
A crise da modernidade não podendo mais concluir o circuito de constante reposição da falta -- quer dizer, o consumo
realizado repõe o desejo de mais mercadorias -- as seguranças de
realização desse indivíduo são solapadas e na busca de reafirmação,
desse Si impossibilitado, novos apegos simbólicos são criados via
identidade;
\item
Embora haja a tentativa de redução do Eu a um agente do
circuito da mercadoria, é a abertura dada pela impossibilidade de
preenchimento desse Eu que permite a possibilidade de alteração das
coordenadas sociais comprimidas pelo capital. 
\end{enumerate}

Em suma, não existe possibilidade de pertencimento no capitalismo.

Por isso, a afirmação da diferença de tratamento da linguagem entre um
\emph{negro} e um \emph{não"-negro}, ou entre um \emph{negro} e um
\emph{branco},\footnote{\versal{FANON}, F. \emph{Pele negra, máscaras brancas}.
  Tradução de Renato da Silveira. Salvador: \versal{EDUFBA}, 2008.} feita por
\index{Fanon, Frantz}Fanon, serve, assim, para dar uma sacudidela aos ossos da estrutura de
uma diferença de \emph{raças} simbolicamente criada e ainda hoje
naturalizada. E aqui é importante dizer que \index{Fanon, Frantz}Fanon, em seu famoso livro,
teve um \emph{insight} genuíno que possibilita entendermos não apenas
esses processos contraditórios, como avançar para além deles.

A invenção da raça sob os pressupostos da exploração colonial impõe ao
negro uma realização impossível. \emph{Um Real} impossível, traumático,
em que a rede simbólica de reconhecimento mútuo está fechada. O modo
simbólico da linguagem como resultado de uma contingente luta complexa
pelo poder sociossimbólico é abordado aí no sentido da exclusão que esse
processo efetivou para o negro.

O negro, nesse sentido, não é um Outro do branco em sua universalidade
colonizadora, mas um \emph{inexistente} numa universalidade que elide ao
negro qualquer possibilidade de reconhecimento. Contraditoriamente,
porém, o negro só existe em relação a essa exclusão do domínio branco.
Ultrapassar essas limitações é o fim previsto na violência clínica
fanoniana.

Talvez seja por isso que \index{Fanon, Frantz}Fanon advirta desde o início que, em se
tratando de uma análise psicológica do negro, não se devem esquecer os
elementos que fundamentaram essa ordem sociossimbólica, quer dizer: não
podemos perder de vista a história socioeconômica que engendrou essa
noção de diferenciação.

Em termos simples, enquanto o branco alçou"-se à condição de
proto"-sujeito, para o negro essa condição está vedada pelos processos de
colonização. Entretanto, ao contrário do que parece sugerir, a condição
buscada na análise de \index{Fanon, Frantz}Fanon não é a resolução pura e simples dessa
condição de um não"-sujeito para a condição de um sujeito como ponto de
síntese e resolução dos conflitos.

\index{Fanon, Frantz}Fanon de nenhuma forma poderia incidir nessa ingenuidade, pois sabia
que, independente dos espaços de simbolizações, a lacuna e a castração
se mantêm inalteradas. O negro como um invisibilizado, como um
inexistente que não é um nada, que é um ser nada, mas não um nada ser,
inexiste nas condições de possibilidade de um mundo formatado pela
colonização.

Por isso, esmiuçar as patologias sociais criadas pelo sistema de
linguagem dominante requer um desnudamento da relação de sujeição; se o
negro é o \emph{nada} graças a sua invisibilidade radical, deve
tornar"-se \emph{menos que nada}. Essa práxis"-teórica aponta o limite a
ser ultrapassado. É a neutralidade do marco simbólico da linguagem que
está em disputa e com ela a própria noção do que é \emph{ser negro}.

Por isso, desde o início, o destino da identidade em si mesma está
selado. O paradoxo em questão é que o próprio fato de não haver uma
identidade hipostasiada, na qual se possa fundamentar ontologicamente o
\emph{ser negro}, é o que torna possível a efetiva resistência negra a
partir da implosão da estruturação sociossimbólica.

A questão da linguagem, desse modo, determina uma forma de ser no mundo,
de estar aí em \emph{relação a}, fundamentando"-se a partir dos processos
sociais implicados no mundo concreto. Assim, se, por um lado, adotar a
linguagem do colonizador implica uma desestruturação da identidade, por
outro, é a partir dela que o negro/colonizado toma posição contrária e
se acerca dos seus limites.

Há duas posições antinômicas que \index{Fanon, Frantz}Fanon faz questão de evidenciar:

\begin{enumerate}[label=\alph*)]
\item
Aquela de superidentificação com os mecanismos colonialistas,
que adota e privilegia os aspectos dominantes da colônia, a branquitude,
a europeização, etc.; 
\item
Aquela que, \emph{negando}, busca um retorno a si e se redobra em defender suas origens. Ambas são patologias mistificadoras.
\end{enumerate}

São os processos implicados na aproximação com a linguagem do
colonizador que instauram a negação de si para acatar acriticamente as
formas da universalidade imposta. Há uma questão de fundo que ressoa; a
identificação pura e simples com a alteridade imposta do colonizador
leva à manutenção das relações de subordinação.

Assim, ao demonstrar a situação de assimilado do martinicano, \index{Fanon, Frantz}Fanon
deixa claro que o que regula seu processo psíquico é um desavim consigo
mesmo, uma negação de sua humanidade, por ver no \emph{Outro}
colonizador a capacidade de sua realização. O que está implicado nisso
são as condições de possibilidade nas quais o processo de colonização, e
subordinação, se torna um processo autorreferente de realização. O
processo aí é tão totalizador, quanto o comportamento da mulher negra
analisado por \index{Fanon, Frantz}Fanon, em relação ao branco europeu, que evidencia os
mecanismos de captura da subjetividade e de sua castração egóica que vê
no colonizador, com suas características fenotípicas, a possibilidade de
realização do próprio ego. O resultado disso é que o processo de
embranquecimento já está todo articulado por uma posição cuja antinomia
negro/branco está naturalizada e é aceita no registro simbólico.

Essa abstração real da raça\footnote{Tomamos este termo emprestado de
  \index{Marx, Karl}Marx, para quem o capital é uma abstração real -- empregamos aqui no
  mesmo sentido.} -- que ao mesmo tempo que fundamenta a relação social
funda sua forma categorial -- é um processo no qual a justificação
excludente se dá no plano sociossimbólico. Isso passará a governar os
destinos individuais guiando"-os para uma submissão frente àquilo que
aparece como o ``bom''. É essa estruturação da subjetividade colonizada
que importa a \index{Fanon, Frantz}Fanon.

Tendo isso em vista chega"-se à conclusão de que o processo entre acatar
essa condição ou fugir dela, em busca de um retorno originário, se
coloca como algo imediatamente interno ao processo, quer dizer: as duas
posições são coniventes com os termos erguidos pelo colonizador.

Assim, a análise empreendida dos romances \emph{Je suis Martiniquaise},
\emph{Nini} e o de \index{Maran, René}René Maran demonstra como no nível simbólico das
personagens os resultados da colonização já estão postos. É como se os
romances em sua possibilidade de desnudamento de um \emph{etos} fossem
mais impregnados de verdade do que a empiria da vida do \emph{aqui} e
\emph{agora}.

A incapacidade de Jean Veneuse -- personagem de \index{Maran, René}Maran comentado por
\index{Fanon, Frantz}Fanon -- de concretizar sua relação amorosa com uma europeia desnuda
como o processo de inferioridade circunstanciada por uma \emph{psique
abandônica} -- quer dizer, aquela cujo trauma de abandono na infância
impossibilita a realização de relacionamentos duradouros por uma
autocomplacência inferiorizada -- se efetiva a despeito da camisa de
força dada pela \emph{racialização}. A busca de um retorno à pátria,
substancial e orgânica, só revela a impotência desse neurótico se
realizar por se agarrar nas definições impostas pelo modo de controle
colonial.\footnote{Aliás, como já salientava \index{Bauman, Zygmunt}Bauman (\emph{op. cit.}, p. 35):
  ``O anseio por identidade vem do desejo de segurança, ele próprio um
  sentimento ambíguo.''} Por isso, ``Jean Veneuse não representa um
exemplo das relações negro/branco, mas o modo como um neurótico,
acidentalmente negro, se comporta''.\footnote{\versal{FANON}, \emph{idem}, p. 81.} A
posição de uma autoflagelação, de uma autopiedade, e de uma desconfiança
geral com todo o diferente, marca a postura desse Fiódor Karamazov negro
e bondoso.

A questão é muito simples: de pronto, tenta"-se justificar, pelo
componente racial, a presença do injustificado trauma de abandono na
infância. O paradoxo é que, sendo preto, não posso ser amado; e se for
amado, e corresponder a esse amor, nada me garante que não estou me
aproveitando dele -- ``como os pretos que adoram carne branca'' -- por
ser preto igual aos outros. Resultado, não posso porque quero e não
posso porque posso e serei igual aos outros. Com isso, uma patologia
ligada ao trauma infantil é ocultada pelas limitações impostas pela
racialização da própria subjetividade.

Independente de sua forma, o que \index{Fanon, Frantz}Fanon parece sugerir é que a lógica
interna do movimento da subjetividade dessas personagens eivadas de
preconceitos colonialistas se vê solapada quando passa de um extremo
para o extremo oposto e se funda numa unidade supostamente mais elevada:
o embranquecimento ou sua negação em nome de um retorno às origens nada
mais é do que os limites impostos pela condição de desumanidade absurda
na colônia.

Se a primeira opção dos oprimidos é tentar se livrar daqueles que os
oprimem, enquanto a segunda é deles se aproximar negando"-se a si mesmos,
ambas fracassam quando os oprimidos não percebem que a identidade de sua
posição está mediada pelo \emph{Outro}, de modo que para ultrapassar
essa condição é necessário transformar substancialmente o conteúdo dessa
própria posição.

Isso não significa querer se pôr no atual sistema de visibilidade
reivindicando uma representatividade limitante e limitada, mas
transformar radicalmente esse próprio sistema de visibilidade cuja raiz
está no sistema econômico"-social. É como se para o negro restasse não
apenas a negação de sua posição imposta por um sistema de linguagem que
o subordina -- negação que permanece em seus limites simbólicos --, como
ainda é necessário negar o próprio espaço simbólico.

A linguagem, com efeito, torna"-se uma via de mão dupla: por um lado, ela
lança o seu portador no mundo social, por outro, no caso dos
colonizados, ela subordina suas aspirações ao império da
\emph{metrópole}. Assim, no registro de visibilidade colonizador, a
suposta realização desse indivíduo limitado se dará quanto maior for sua
proximidade com os modos dos ``civilizados'', ou seja, quanto maior for
seu embranquecimento ou sua busca de uma fictícia negritude.

Com efeito, a \emph{ideia} que se faz do negro, enquanto uma categoria
cuja constituição é de subordinado -- de um assujeitado sem ser sujeito --
o reduz a caricatura e o coloca em termos limitados e limitantes
definidos pela própria linguagem. Por isso ``Compreende"-se {[}\ldots{}{]} que
a primeira reação do negro seja a de dizer \emph{não} àqueles que tentam
defini"-lo.''\footnote{\versal{FANON}, \emph{idem}, p. 48, grifo meu.} É dessa limitação
que surge uma espécie de antinomia colonizadora, uma espécie de
\emph{ou}, \emph{ou}: ou o negro se aproxima do civilizado, ou o
rejeita, estabelecendo uma fuga particularista.

O contra"-argumento de \index{Fanon, Frantz}Fanon, deixado de lado por grande parte daqueles
que dizem segui"-lo, é que os próprios mecanismos subordinadores abrem
espaço para a resistência na medida em que fundamentam uma espécie de
excedente, de um \emph{não"-lugar} para o negro, que acaba por engendrar
uma posição política.

Paradoxalmente, a referência discursiva posta na linguagem da metrópole
confirma o negro como o fundamento estabelecido sobre o qual opera a
linguagem do colonizador. Quando
o negro surge como estrutura patológica, como o demônio, o pecado, o mal
e o sexo abundante -- uma identidade fantasmagórica e alheia à psique
branca, -- é porque a relação da identidade branca já está implicada na
identidade negra. Esta via de mão dupla também incide no próprio
esmorecimento da identidade cultural do branco.

É como reação à dominação do
colonizador, reação fomentada por um \emph{não"-lugar} e dominação
efetivada pelo próprio negro, que a consciência se transforma em vontade
política ativa para afirmar sua identidade fantasmagórica. \index{Fanon, Frantz}Fanon revela
que assim que se alcança essa consciência, o indivíduo não só se
integrou ao universo do colonizador, como agora é capaz de implodi"-lo.
Quer dizer, a identidade fantasmagórica é um momento de integração que
parte para a desintegração da subjetividade colonizada. Em ambos os
casos, tanto a identidade branca quanto a negra forjam a mistificação de
um acesso direto à coisa, como se fosse possível superar a alienação
constitutiva do sujeito pela identificação com os processos culturais de
nossa identidade. Por isso, a morte do colonialismo deve ser ``real'', no
registro de operações sistêmicas; e simbólica, no registro linguístico
que estrutura tais operações.

Como é deliciosa a forma como \index{Fanon, Frantz}Fanon posiciona o sistema de estrutura
simbólica para demonstrar isso ao comentar o livro de \index{Maran, René}René Maran: ``só
sei de uma coisa!'' diz o crítico citando o literato: ``{[}\ldots{}{]} que o
preto é um homem igual aos outros, um homem como os outros, e que seu
coração, que só parece simples aos ignorantes, é tão complexo quanto o
do mais complexo dos europeus''.\footnote{\versal{FANON}, \emph{idem}, p. 71.}

Está demonstrado, portanto, como a impossibilidade de reconhecimento
impõe para esse \emph{aquém de um Outro} (o negro) a tarefa de desnudar
os limites que engendraram a impossibilidade de reconhecimento e, com
isso, a tarefa de se colocar para além da própria necessidade de
reconhecimento ao abolir as diferenças fictícias criadas socialmente
pela exclusão. Noutros termos: explode"-se o real absurdo ao desnudar
completamente o seu escândalo.

\chapter{Os significados da dialética}

Temos visto até agora que a posição crítica de \index{Fanon, Frantz}Fanon não constitui um
retorno mágico à identidade, mas a demonstração de seu descentramento
contra a presunção de universalidade baseada na noção de \emph{sujeito}.
O desnudamento do processo de subordinação imperante na linguagem
imprime um primeiro movimento que nega a substância social da qual essa
linguagem emerge, ao mesmo tempo que essa substância social se
transforma quando seus limites se tornam visíveis por essa
particularidade. Nesse processo, a identidade já absorveu e reestruturou
a concretude social.

É essa posição eminentemente dialética que permite a \index{Fanon, Frantz}Fanon escapar de
uma espécie de essencialismo do racismo e identificá"-lo à sombra dos
processos modernos cujo cerne é a economia. Assim, quando \index{Fanon, Frantz}Fanon fala de
estrutura, está falando dos processos econômico"-sociais que engendraram
não apenas a exclusão, como a concorrência. Por isso, ao rebater o
essencialista \index{Mannoni, Maud}Mannoni, expõe o seguinte:

\begin{quote}
Poderíamos retrucar que este desvio da agressividade do proletariado
branco na direção do proletariado negro é, fundamentalmente, uma
consequência da estrutura econômica da África do Sul.

Que é a África do Sul? Um caldeirão onde 2.530.300 brancos espancam
13.000.000 de negros. Se os brancos pobres odeiam os pretos não é, como
nos faz entender \index{Mannoni, Maud}Mannoni, porque ``o racismo é obra de pequenos
comerciantes e de pequenos colonos que deram duro durante muito tempo
sem sucesso''. Nada disso, é porque a estrutura da África do Sul é uma
estrutura racista [\ldots{}].\footnote{\versal{FANON}, \emph{idem}, p. 86.}
\end{quote}

Os processos analisados por \index{Fanon, Frantz}Fanon dizem respeito à lógica do fracasso de
colonização, que produz anomalias ligadas ao terreno histórico"-social de
onde emergem. Sua posição desnuda os processos subjetivos e objetivos
socialmente conduzidos pelas modernas forças de produção e reprodução
social. Não é à toa que \index{Fanon, Frantz}Fanon retruca as posições abstratas de seus
contemporâneos: ``Ao considerar abstratamente a estrutura de uma ou
outra exploração, mascara"-se o problema capital, fundamental, que é
repor o homem no seu lugar.''\footnote{\emph{Ibidem}.}

O feito de \index{Fanon, Frantz}Fanon é combinar o caráter constitutivo do negro em sua
atividade no mundo com o viés patológico da própria noção de negro;
quando ambos são pensados juntos, como uma característica recíproca,
então conseguimos captar a própria patologia que constitui a realidade
colonizada. ``A inferiorização é o correlato nativo da superiorização
europeia. Precisamos ter a coragem de dizer: \emph{é o racista que cria
o inferiorizado}.''\footnote{\versal{FANON}, \emph{id}., p. 90.} Noutros termos, é o
racista que cria a fantasmagoria do Negro.

Esses processos interobjetivos,\footnote{Este termo tomei emprestado de
  \index{Filho, Sílvio Rosa}Sílvio Rosa Filho.} por assim dizer, desestruturam todo o complexo
social dos colonizados, solapam suas estruturas de identidade e recriam,
à luz do processo, novas formas de sociabilidade. Uma vez desestruturada
a identidade, o choque não permite mais um retorno ao que foi
anteriormente: ``Uma ilha como Madagáscar, invadida de um dia para o
outro pelos `pioneiros da civilização', mesmo que esses pioneiros tenham
se comportado da melhor maneira possível, sofreu uma
desestruturação.''\footnote{\versal{FANON}, \emph{id}., p. 93.}

Tais relações, para \index{Fanon, Frantz}Fanon, obviamente, não incidem apenas na psique
daquele que se vê invadido por um Outro negador, mas na relação, entre a
consciência e o contexto social. É na medida em que se efetiva a
diferenciação do processo discriminatório determinado pela colonização,
que se impõe para mim a alteridade que tenho que assumir/desempenhar.
Fazer"-me branco é uma tentativa ilusória de obrigar o branco a
reconhecer a minha humanidade.

Para essa situação patológica, que dá movimento aos processos de
colonização tanto material quanto espiritual, \index{Fanon, Frantz}Fanon encontra uma via de
superação concreta: o complexo de inferioridade só surge numa sociedade
patologizada em que o racismo é estrutural; só com a mudança das
estruturas sociais é que se pode ultrapassar essa condição inumana. Não
cabem ilusões.

A busca é a de tornar o sofredor ``capaz de escolher a ação (ou a
passividade) a respeito da verdadeira origem do conflito, isto é, as
estruturas sociais''.\footnote{\versal{FANON}, \emph{idem}, p. 96.} Com isso, \index{Fanon, Frantz}Fanon
torna"-nos cientes de que a forma de ultrapassar o estado posto pelo modo
de sociabilidade colonizado é uma escolha que negue, concretamente, o
Todo abstrato fundado por essa mesma sociabilidade.

Com isso, não há esperanças de criar uma nova ordem orgânica que tenha
por função abolir a individualidade. É, pelo contrário, a afirmação do
indivíduo e a negatividade abstrata impressa por uma identidade
evanescente que faz com que o Todo concreto seja modificado em sua
raiz.\footnote{Nesse ponto se encontra a força da argumentação de \index{Fanon, Frantz}Fanon.
  A formação da identidade é um componente fundamental na experiência da
  consciência do negro, no entanto, enquanto componente relacional,
  embora a colonização tenha castrado essa capacidade de uma relação
  equitativa com o outro, a resposta se dá na sua forma de se fazer
  conhecer. Essa individualidade, por definição, está como algo além do
  jogo imposto pelo modo sociossimbólico opressor, e, como tal, irrompe
  com a inversão do racismo em racialização da própria identidade, que
  desnuda a relação de exploração e opressão, além da unilateralidade de
  sua própria posição.} Aí está colocada uma renúncia à esperança
nostálgica de um retorno à pátria perdida.

É por isso que estão distantes dos ensinamentos de \index{Fanon, Frantz}Fanon tanto aqueles
que defendem uma submissão voluntária e a aceitação pelo indivíduo dos
pressupostos colonialistas -- tidos como a totalidade concreta --, quanto aqueles que afirmam
inexoravelmente um retorno às raízes pré"-coloniais\ldots{}

\chapter{\emph{Weltanschauung}\footnote{\emph{Weltanschauung} cuja tradução literal significa ``visão de mundo'' diz respeito ao etos ou à forma de vida alcançada a partir da compreensão da necessidade e a liberdade que daí advém. É um conceito que encontra grande influência em \index{Hegel, G. W. Friedrich}Hegel e depois se desdobra nas escolas hegelianas chegando ao existencialismo e à fenomenologia.} do negro}

É a partir desse \emph{não"-lugar} que \index{Fanon, Frantz}Fanon busca demonstrar a situação
de completo expatriado do negro. Em \emph{A experiência vivida}, talvez
o mais desconcertante e difícil capítulo de seu livro, temos o complexo
desenvolvimento das relações intersubjetivas que fundamenta a psique
negra numa espécie de \emph{fenomenologia} existencial.

Esse marco conceitual nos permite abordar a situação de \emph{plena
abertura} vivida pelo negro. A ontologia incapacitada, ou incompleta,
pela dolorosa cisão enfrentada pelo negro é o que sustenta sua realidade
psíquica. A questão que podemos
colocar para \index{Fanon, Frantz}Fanon, que diz que ``Há, na \emph{Weltanschauung} de um
povo colonizado, uma impureza, uma tara que proíbe qualquer explicação
ontológica'',\footnote{\versal{FANON}, \emph{idem}, p. 103.} é se essa proibição não
desnuda toda a fragilidade da operação hegemônica colonial fundada no
caráter mistificador socialmente produzido contra o negro.

Parece"-me que a resposta é sim.
Os caminhos barrados à ontologia
colocam para o negro uma outra alteridade que lhe é contraposta e na
qual sua referência não está localizada nos limites da \emph{vivência}
em seu próprio corpo com relação ao outro: o domínio colonial suprime de
si o mundo do colonizado e, com isso, desestrutura também sua
identidade. Com o branco à espreita, o mundo estruturado do negro entra
em colapso.

Isso elucida dois acontecimentos que são inter"-relacionados no
pensamento de \index{Fanon, Frantz}Frantz Fanon: por um lado, toda identificação hegemônica
se revela mutável -- inclusive aquela do negro com seu sistema de
referência --; por outro, ter
isso às claras manifesta a consequência contingente de uma luta que se
dá no âmbito histórico"-social. Ora, é no próprio corpo que se marca a
diferença do negro com relação ao branco: pela negação de sua
subjetividade enquanto um Outro, o corpo negro é aquele que se
apresenta. Uma redução chocante que explica o porquê da natural
adjetivação de uma pessoa negra.\footnote{Quando um negro entra num
  recinto, acompanhado por um branco, o que se ouve é: ``havia um homem
  e um negro'', e assim sucessivamente, até chegar nos jornais que num
  acidente disseram: ``duas pessoas e um negro se acidentaram''; com
  esses exemplos, que o leitor brasileiro deve conhecer, fica evidente
  por que \index{Fanon, Frantz}Fanon diz que o negro é reduzido ao corpo e à apresentação.}

Os desdobramentos dessa diferença cairão em fantasmagorias psíquicas
sexualizadas explicadas de modo profundo por \index{Fanon, Frantz}Fanon, inclusive no que diz
respeito a sua redução à genitália. Aqui cabe, porém, demonstrar que o
campo de batalha entre os desejos secretos e as proibições simbólicas é
aquilo que objetificará o negro, mas também definirá a relação que o
branco tem com essas fantasmagorias. Como resultado, em sua formação
(\emph{Bildung}) o negro encontrará impossibilidades de se descobrir
como agente histórico e ativo. Ao negro em situação colonial está vedado
o componente relacional, justamente pelo caráter mistificador dessa
fantasmagoria colonialista, e com ele a possibilidade de reconhecimento
pelo Outro.

Essa espécie de fenomenologia do negro levada adiante por \index{Fanon, Frantz}Fanon desnuda
a relação de objetificação corporal que destroçará a subjetividade ao
abstrair todo o \emph{ser negro} a mera condição epidérmica. Tal
situação corrobora uma redescoberta dos limites que o indivíduo
estabelecia para si mesmo: ocupar um determinado lugar, ir ao encontro
de outro e ver o outro desaparecer -- como \index{Fanon, Frantz}Fanon deixa claro -- repõe toda
uma esfera de significações que permite ao negro se reconhecer enquanto
tal naquilo em que ele mesmo como um martinicano -- ou seria um
brasileiro? -- era incapaz de se reconhecer, isto é, em sua
\emph{condição de negro}.

O próprio indivíduo negro arrancado de sua certeza ao resistir a essa
objetificação racista é, no fundo de seu âmago, marcado por ela. A
questão terrificante é que quando o indivíduo negro se descobre por fim
negro já pesa em suas costas a melanina e com ela todas as referências
que o mundo dominante e branco lhe traz, quer dizer: preconceitos, taras
raciais, fetichismo\ldots{}

É natural que aqui esteja colocada uma questão importante para os
desnudamentos da dominação racial: como os negros subjetivam sua
condição?

\index{Fanon, Frantz}Fanon responde a essa questão de maneira genuína e original: ao querer
ser só um homem entre outros homens, me reduziram a cor da pele, ao
assumir minha cor, assumo os ancestrais escravizados e linchados, porém,
por essa situação permanecer vazia em sua estrutura simbólica, a
ultrapasso quando descubro que todas as formas de apreensão que tenho de
mim mesmo passam pelo Outro. Como negro apreendo a mim mesmo pelo
branco, embora sabendo que o branco nada sabe do meu Eu; é ele que me
nega e ao me negar me constitui. Vou adiante e descubro que sou neto de
escravos, assim como um branco foi neto de camponeses explorados e
oprimidos: ``Na América, os
pretos são mantidos à parte. Na América do Sul, chicoteiam nas ruas e
metralham os grevistas pretos. Na África Ocidental, o preto é um animal.
E aqui, bem perto de mim, ao meu lado, este colega de faculdade,
originário da Argélia, que me diz: `Enquanto pretenderem que o árabe é
um homem como nós, nenhuma solução será viável'".\footnote{\versal{FANON}, \emph{idem},
  p. 106.} Logo, todas as diferenças se revelam como diferenças
nenhumas. É a partir dessa condição de expatriado, que culmina num
\emph{não"-lugar reservado ao negro,} que se dá a possibilidade de uma
indiferença às diferenças, à medida que a descoberta sobre as diferenças
revela o componente da estrutura social que as fomenta. O negro é esse
ponto descentrado que sustenta o devir do não"-ser.

O problema se desdobra: como os negros subjetivam sua condição de
explorados e oprimidos?

Não basta apenas reconhecer"-se em sua condição, pois esse é o aspecto
mais simples porque imposto; é preciso compreender que, sendo impossível
livrar"-se de um \emph{complexo inato} e deixar de afirmar"-se como negro,
deve o negro fazer"-se \emph{conhecer} implodindo o próprio espaço de
construção simbólica estruturado pelas relações sociais de exploração
colonial. O paradoxo é que, tendo sido estruturado o corpo positivado do
negro como algo místico, não se pode fundamentá"-lo ontologicamente. E é
justamente isso que possibilita uma efetiva resistência. Sendo o negro
sobredeterminado pelo exterior, não é escravo da ideia,\footnote{Na visão
  de \index{Fanon, Frantz}Fanon, sustentada pela análise sartriana, são os judeus que são
  escravos das ideias que se fazem deles. O que ele quer dizer é que
  ambos, o negro e o judeu, em seu infortúnio sofrem do racismo, porém
  enquanto o judeu pode disfarçar sua origem, o negro a carrega nos tons
  da pele.} mas da sua própria aparição.

E aqui a bela poesia de \index{Cruz, Victoria Santa}Victoria Santa Cruz vale como ilustração:

\begin{verse}
\emph{De hoy en adelante no quiero\\
laciar mi cabello\\
No quiero\\
Y voy a reirme de aquellos,\\
que por evitar -- según ellos --\\
que por evitarnos algún sinsabor\\
Llaman a los negros gente de color\\
¡Y de que color!\\
\versal{NEGRO}\\
¡Y que lindo suena!\\
\versal{NEGRO}\\
¡Y que ritmo tiene!\\
\versal{NEGRO} \versal{NEGRO} \versal{NEGRO} \versal{NEGRO}\\
\versal{NEGRO} \versal{NEGRO} \versal{NEGRO} \versal{NEGRO}\\
\versal{NEGRO} \versal{NEGRO} \versal{NEGRO} \versal{NEGRO}\\
\versal{NEGRO} \versal{NEGRO} \versal{NEGRO}}\footnote{De hoje em diante não quero/
  alisar meu cabelo/
  Não quero/
  E vou rir daqueles/
  que para evitar -- segundo eles --/
  que para evitarmos algum dissabor/
  Chamam os negros de gente de cor/
  E de que cor?!/
  \versal{NEGRO}/
  E como soa lindo!/
  \versal{NEGRO}/
  E olha esse ritmo!/
  \versal{NEGRO} \versal{NEGRO} \versal{NEGRO} \versal{NEGRO}/
  \versal{NEGRO} \versal{NEGRO} \versal{NEGRO} \versal{NEGRO}/
  \versal{NEGRO} \versal{NEGRO} \versal{NEGRO} \versal{NEGRO}/
  \versal{NEGRO} \versal{NEGRO} \versal{NEGRO}.\\
  \emph{In}: \emph{Me gritaron negra}, \index{Cruz, Victoria Santa}Victoria Santa Cruz (tradução minha).}
  \end{verse}

É com isso, com esse aparecer que
a referência ao negro, o esforço colonial para conter e categorizar o
negro, mistificando"-o, produz formas de resistência como princípio ativo
contra a força opressora. O poder opressor gera a forma de resistência.
O princípio ativo contra a força opressora, a capacidade de mediar, não
apenas elabora as formas de estruturação do simbólico como ainda
transforma o próprio núcleo da identidade. Já que o negro não é
reconhecido é preciso que ele se faça conhecer.

A consciência do círculo infernal, que captura o negro reduzindo"-o, e
nadificando a sua existência, é o que possibilita escapar desse círculo.
\index{Fanon, Frantz}Fanon no seu caminho ao expor essa fenomenologia do negro demonstra,
porém, como alguns caminhos estão limitados a tornarem o negro um objeto
passivo. São eles:

\begin{enumerate}[label=\alph*)]
\item
A busca do conhecimento como forma política -- como sugere
\index{Fanon, Frantz}Fanon: \emph{A razão contra a irracionalidade racista produz náusea}.
Nesse ponto, é importante lembrar -- ainda que para nossos objetivos isso
não tenha tanta importância -- como \index{Fanon, Frantz}Fanon estabelece uma relação de
equidade no infortúnio entre o negro e o judeu.\footnote{O judeu e eu:
  não satisfeito em me racializar, por um acaso feliz eu me humanizava.
  Unia"-me ao judeu, meu irmão de infortúnio. Uma vergonha! À primeira
  vista, pode parecer surpreendente que a atitude do antissemita se
  assemelhe à do negrófobo. Foi meu professor de filosofia, de origem
  antilhana, quem um dia me chamou a atenção: ``Quando você ouvir falar
  mal dos judeus, preste bem atenção, estão falando de você''. E eu
  pensei que ele tinha universalmente razão, querendo com isso dizer que
  eu era responsável, de corpo e alma, pela sorte reservada a meu irmão.
  Depois compreendi que ele quis simplesmente dizer: um antissemita é
  seguramente um negrófobo. (\emph{In}: \versal{FANON}, \emph{idem}, p. 112, \emph{grifo meu}).}
Isso não se deve ao fato de que uma de suas fontes inspiradoras para
pensar a condição do negro seja \index{Sartre, Jean-Paul}Sartre em \emph{Réflexions sur la
question juive,} mas sim pela necessidade de demonstrar que o racismo se
produz de diversas formas. Mesmo nos dedicando ao saber, se a estrutura
sociossimbólica do capital em seu processo permanecer inalterada, os
horizontes de mudanças estarão fechados.
\item
Fracassado em sua busca da razão como forma de emancipação,
eis que surge o elemento da sensibilidade: ``O sacrifício tinha servido
de meio termo entre mim e a criação -- não encontrei mais as origens,
mas a \emph{Origem}. No entanto, era preciso desconfiar do ritmo, da
amizade Terra"-Mãe, deste casamento místico, carnal, do grupo com o
cosmos.''\footnote{\versal{FANON}, \emph{idem}, p. 115.} Imersa na poesia de um retorno
místico às origens, essa consciência que se viu objetificada pelo
colonialismo busca uma saída no elemento mítico.
\end{enumerate}

Ambas as posições fracassam, mas são constitutivas dessas consciências
como \emph{ilusões necessárias}. Felizmente, porém, a consciência segue
seu curso e, diante desse louvor de uma união mítica e sensível com a
mãe"-terra, desconfiada, repõe em curso a dúvida. Este é o movimento que
\index{Fanon, Frantz}Fanon faz deixando para trás as crenças que invadem hoje grandes setores
do movimento negro. Essa união mística é só misticismo:

\begin{quote}
Fiz caminhadas até os limites de minha essência; eles eram, sem dúvida
alguma, estreitos. Foi então que fiz a mais extraordinária das
descobertas, aliás, propriamente falando, uma redescoberta.

Revirei vertiginosamente a antiguidade negra. O que descobri me deixou
ofegante. No seu livro \emph{L'abolition de l'esclavage}, Schoelcher nos
trouxe argumentos peremptórios. Em seguida Frobenius, Westermann,
Delafosse, todos brancos, falaram em coro de Ségou, Djenné, cidades de
mais de cem mil habitantes. Falaram dos doutores negros (doutores em
teologia que iam a Meca discutir o Alcorão). Tudo isto exumado,
disposto, vísceras ao vento, permitiu"-me reencontrar uma categoria
histórica válida. O branco estava enganado, eu não era um primitivo, nem
tampouco um meio"-homem, eu pertencia a uma raça que há dois mil anos já
trabalhava o ouro e a prata.\footnote{\versal{FANON}, \emph{idem}, p. 118-9.}
\end{quote}

O que \index{Fanon, Frantz}Fanon aponta com grande lucidez é que ao contrário da loucura
mítica, a história comprova que os negros são também agentes da razão e
do intelecto e que esse aspecto místico"-religioso está limitado pelo
próprio pensamento colonial. \emph{Razão} e \emph{Sensibilidade} são
momentos constitutivos para apreender o que é \emph{ser negro} nos
limites impostos pelo colonialismo, porém ultrapassar essa condição é
fundamental. Nesse percurso da consciência, a identidade está sempre se
desestruturando e repondo seu movimento. Da tentativa de agarrar a razão
contra o irracionalismo até a tentativa de se agarrar a sensibilidade
poética que estrutura uma espécie de retorno às raízes culminando numa
redescoberta dessas próprias origens que não são aquelas míticas, mas
são as de um desenvolvimento histórico, o que está apontado por \index{Fanon, Frantz}Fanon é
o desdobramento da consciência do \emph{ser Negro} rumo a sua
emancipação efetiva.

\chapter{Sartre e a dialética espanada}

É nessa espécie de paciência frente aos fracassos que \index{Fanon, Frantz}Fanon arma seu
arsenal crítico disposto a demonstrar os problemas que surgem ao não se
demorar nesses sintomas impostos por uma realidade totalmente
patológica. O modo como expressa essa relação desafortunada de um
significante vazio, evanescente e contraditório, de uma subjetividade
destroçada, molda a visão de mundo que o negro forma a partir de sua
relação com o próprio mundo.

Nesse ponto ele está fazendo um exercício legado pelo pensamento
especulativo ou fenomenológico. Cada passo de sua formação são figuras
neuróticas da consciência que tentam se firmar naquilo que acreditam ser
a verdade de si mesmas. No entanto, as contradições e incertezas são o
resultado dessa experiência que passa de uma figura a outra sem poder se
firmar mediante as contradições emergentes de sua relação com o mundo
colonizado. Aí, tudo que é sólido se dissolve no ar.

Quando \index{Sartre, Jean-Paul}Sartre entra em cena, e aqui podemos pensá"-lo como mais uma
figura da consciência, ao tentar desbaratar o jogo utilizando o
pensamento especulativo para, por fim, tornar claras as limitações do
negro, o que faz é somente mais um exercício que dá com os burros
n'água. \index{Sartre, Jean-Paul}Sartre torna"-se uma figura da consciência de \index{Fanon, Frantz}Fanon. Comentarei
mais detalhadamente essa cena imperdível:

\begin{quote}
Mas a coisa pode ser mais séria ainda: o negro, nós o dissemos, cria
para si um racismo antirracista. Ele não deseja de modo algum dominar o
mundo: ele quer a abolição dos privilégios étnicos, quaisquer que sejam
eles; ele afirma sua solidariedade com os oprimidos de qualquer cor. De
repente a noção subjetiva, existencial, étnica da negritude ``passa'',
como diz \index{Hegel, G. W. Friedrich}Hegel, para aquela -- objetiva, positiva, exata -- do
proletariado. ``Para Césaire, diz Senghor, o ``branco'' simboliza o
capital, como o negro o trabalho\ldots{} É a luta do proletariado mundial que
canta através dos homens de pele negra de sua raça'' É mais fácil
dizer, menos fácil pensar. Não é por acaso que os mais ardentes vates da
negritude são, ao mesmo tempo, militantes marxistas. Mas isso não impede
que a noção de raça não se confunda com a noção de classe: aquela é
concreta e particular, esta universal e abstrata; uma vem do que Jaspers
chama de compreensão, e a outra, da intelecção; a primeira é o produto
de um sincretismo psicobiológico e a outra é uma construção metódica, a
partir da experiência. De fato, a negritude aparece como o tempo fraco
de uma progressão dialética: a afirmação teórica e prática da supremacia
do branco é a tese; a posição da negritude como valor antitético é o
momento da negatividade. Mas este momento negativo não é
autosuficiente, e os negros que o utilizam o sabem bem; sabem que ele
visa a preparação da síntese ou a realização do humano em uma sociedade
sem raças. Assim, a negritude existe para se destruir; é passagem e
ponto de chegada, meio e não fim último.\footnote{\versal{FANON}, 2008, p. 121,
  \emph{apud} Jean"-Paul Sartre. Orphée noir, prefácio à \emph{Anthologie de la
  poésie nègre et malgache}, p. \versal{XL} \emph{et seq}.}
\end{quote}

Se há uma posição relativa na ação dos negros tornada racializada no
próprio advento da modernidade, ela obedece ao percurso não determinado,
na verdade totalmente contingente, da realidade histórica. É nisso que a
passagem feita por \index{Sartre, Jean-Paul}Sartre de raça para classe se revela rápida demais
por ser uma posição esquemática que suprime de si as particularidades
constitutivas da compreensão de raça.

A noção algo esquemática de uma progressão dialética como finalidade
indiscutível é aquilo que espana a própria apreensão de dialética
sartriana. \index{Sartre, Jean-Paul}Sartre se trai. A força das análises de \index{Fanon, Frantz}Fanon reside no fato
de que os extremos (negro/branco) permanecem produzindo a cisão e o
único avanço obtido é a compreensão do interior dessa lacuna. Assim, o
próprio racismo antirracista, embora seja um momento necessário, ainda
atua no interior da limitação colonial.

Não é simplesmente identificar mecanicamente o branco com o capital --
ainda que seja uma verdade factível -- mas compreender que essa cisão é
uma abstração (metafísica) que dinamiza a realidade. O verdadeiro
significado está no vazio de seu conteúdo, cujo sentido é gerado na
medida em que o movimento se revela. Daí a necessidade dos exercícios
que \index{Fanon, Frantz}Fanon nos legou. Exercícios que parecem não ter tido
precedentes.\footnote{Mesmo \index{Mbembe, Achille}Mbembe não vai até as últimas consequências
  das lições deixadas por \index{Fanon, Frantz}Fanon.}

A \emph{identidade de opostos} nada tem em comum com a ideia de uma
resolução imposta que eleva a figura da consciência para um estágio
superior -- do tipo raça para classe --, pelo contrário: a luta se firma
no evanescer da experiência que formula uma nova negatividade encarnada
numa figura singular e, portanto, numa nova experiência que reescreve a
passada. E é por isso que \index{Fanon, Frantz}Fanon continua: ``{[}\ldots{}{]}
este hegeliano"-nato esqueceu de
que a consciência tem necessidade de se perder na noite do absoluto,
única condição para chegar à consciência de si.''

Perder"-se na noite do absoluto indica que cada
estágio é necessário e, ao mesmo
tempo, inútil em si mesmo. Só assim se pode chegar à consciência de si.

É aí que \index{Sartre, Jean-Paul}Sartre se trai de novo:

\begin{quote}
Pouco importa: a cada época, sua poesia; a cada época as circunstâncias
da história elegem uma nação, uma raça, uma classe para reacender a
chama, criando situações que só podem ser representadas ou superadas
pela poesia; ora o impulso poético coincide com o impulso
revolucionário, ora diverge. Saudemos, hoje, a oportunidade histórica
que permite aos negros dar com tal determinação o grande grito negro
que abalará os assentamentos do mundo.\footnote{\versal{FANON}, 2008, p. 121, \emph{apud}
  \index{Sartre, Jean-Paul}Sartre, \emph{idem}, p. \versal{XLIV}.}
\end{quote}

Nessa posição sintomática de um devir sem contingência, o espaço para a
liberdade é solapado. ``Foi só a história que produziu a poesia, não os
homens?''

Com razão \index{Fanon, Frantz}Fanon diz: ``Pronto, não foi eu quem criou um sentido para
mim, este sentido (segundo \index{Sartre, Jean-Paul}Sartre) já estava lá {[}\ldots{}{]}
esperando"-me'', e mais abaixo retruca: ``contra o devir histórico,
deveríamos opor a imprevisibilidade.''\footnote{\versal{FANON}, 2008, p. 121.}

Ora, o que \index{Sartre, Jean-Paul}Sartre exclui de sua formulação é que se, por um lado, nada é
sabido que não esteja na experiência, por outro, o devir na história é
marcado por uma finalidade cuja contingência lhe é constitutiva.
É factível a impossibilidade de
se apropriar do futuro tendo em vista que não se pode enquadrar a
tessitura da história.

A negatividade se alimenta da luta por essa apropriação como uma
tentativa impulsionada pela consciência. Por isso que \index{Fanon, Frantz}Fanon diz
\emph{não}!

De fato, \index{Sartre, Jean-Paul}Sartre está correto quando afirma a universalidade da classe em
contraposição à particularidade da raça, contudo, perde de vista que a
universalidade da classe é implicada pela particularidade, que não
apenas dinamiza essa universalidade, como lhe dá sentido. E esse sentido
é formulado pela própria condição do que é \emph{ser negro}, isto é, uma
condição proletária e em constante proletarização.

Se \index{Fanon, Frantz}Fanon afirma: ``Eu tinha necessidade de me perder absolutamente na
negritude. Talvez um dia, no seio desse romantismo
doloroso\ldots{}'',\footnote{\emph{Ibidem}.} é para demonstrar como as ilusões
necessárias formam e dão sentido ao estabelecimento da luta e
questionamento dos limites impostos pelo colonialismo. A abertura da
contínua atividade da consciência em sua busca de se apropriar do futuro
está aprisionada na retroversão, na qual só a exposição completa e
objetiva da experiência demonstra seus percalços e suas ilusões
necessárias para implodir as limitações coloniais.

É precisamente nesse passo que a questão da liberdade se efetiva; a
liberdade, com relação às determinações pressupostas, só pode ser
efetiva contra esse pano de fundo. Não se podem prever as consequências
das nossas ações tendo em vista que, se assim procedesse, a liberdade se
reduziria à necessidade. Perderíamos a retroatividade que constitui
nossa experiência e mantém a abertura para a contingência radical.

Na luta contra essa contingência, ergue"-se o único suporte possível para
a noção de sujeito; um devir capaz de sentido dado pela experiência
deste. Por isso, o negro, no exercício de sua liberdade, mantém de pé a
abertura ao futuro.

\begin{quote}
A dialética que introduz a necessidade de um ponto de apoio para a minha
liberdade expulsa"-me de mim próprio. Ela rompe minha posição
irrefletida. Sempre em termos de consciência, a consciência negra é
imanente a si própria. Não sou uma potencialidade de algo, sou
plenamente o que sou. Não tenho de recorrer ao universal. No meu peito
nenhuma probabilidade tem lugar. Minha consciência negra não se assume
como a falta de algo. Ela é. Ela é aderente a si própria.\footnote{\versal{FANON},
  \emph{idem}, p. 122.}
\end{quote}

Aqui \index{Fanon, Frantz}Fanon demarca uma posição a
postura dialética que encerra um trauma à consciência negra. O fato é
que diante da imposição colonial que presume uma universalidade
excludente, a consciência já não pode ficar impassível, ainda que tenha
razão. E não pode porque ela já é inteiramente, fora dos termos de
oposição binária (preto/branco), o estatuto do inexistente se apresenta
imediatamente. Essa re"-xistência da consciência negra, que imediatamente
\emph{é,} já é em si mesma a possibilidade do movimento dialético, o ser
um que existe como múltiplo. Tentarei explicar o porquê:

Essa individualidade da consciência entendida como negra, enquanto o
\emph{Um}, já está desdobrada em si mesma, quer dizer, é como se a
individualidade aqui estivesse elencada ao universal imediatamente sem,
contudo, dele necessitar. Logo, o Eu=Eu da consciência"-de"-si negra está
posto à prova de saída porque sua identidade jaz ligada ao todo
concreto. A imediação fundada pela impossibilidade de reconhecimento
torna a consciência imediatamente ato. Como nos fez entender \index{Fichte, Johann Gottlieb}Fichte lá
atrás, porém, entre o Eu=Eu da consciência há uma infinidade de
determinações\footnote{\versal{FICHTE}, J. \emph{A doutrina da ciência de 1794}.
  Tradução de Rubens Rodrigues Torres Filho. \emph{In}:
  \_\_\_\_\_\_. \emph{Os pensadores}. São Paulo: Abril Cultural, 1984, p. 35-176.}
tendo em vista que é impossível escapar das mediações sobrepostas, uma
vez que essa consciência é ser social.

A ação externa do mundo branco abala a calma organização desse Eu em seu
movimento. O que aparece como ordem e harmonia de si para consigo
torna"-se, através dessa ação exterior, uma transição de opostos, em que
cada qual se mostra como anulação de si mesmo. Essa ``anulação de si
mesmo'' pressupõe um corte radical imposto pela oposição de duas
tendências no interior de um mesmo plano simbólico (branco/negro).

Podemos intuir daí que há dois universais abstratos que nascem e
precisam morrer juntos: 1) a perda gerada a partir do movimento imposto
pela oposição (feita pelo branco); 2) o reconhecimento dessa própria
perda.

Desse ponto de vista, não é possível nenhum acordo; ambas as posições
são irredutíveis e, portanto, o conhecimento dessa resistência
antagônica é a condição de possibilidade da ação em si, isto é, de
implodir essa limitação simbólica.

Por isso, a posição irredutível defendida por \index{Fanon, Frantz}Fanon é aquela capaz de
reunir os cacos quebrados do que se tem por negritude e a partir dela
implodir o mundo onde essa negritude foi concebida como
diferença/exclusão. \index{Sartre, Jean-Paul}Sartre, que embora avançou radicalmente nas
contradições colônia/metrópole, deixou escapar que é a posição
irredutível dessa particularidade que abala os fundamentos simbólicos do
mundo branco e é nela que repousa o motor da luta de classes nas
sociedades colonizadas.

Passar tão logo às determinações universalmente abstratas de classe é
não se dar conta das determinações raciais e da sua fundamentação
determinada igualmente pela exploração do capital. Aquilo que é capaz de
implodir o modo de sociabilidade baseado na exploração e opressão é sua
especificidade, sua singularidade determinada e irredutível. A
universalidade da classe requer o seu negativo, isto é, a
particularidade que a compõe e a estrutura. Necessariamente, a
consciência de classe é dependente da particularidade e da
especificidade dos seus componentes, não o contrário. Para citar
ironicamente \index{Sartre, Jean-Paul}Sartre: ``a existência precede a essência''.\footnote{Nesse
  sentido a posição de \index{Fanon, Frantz}Fanon é mais sartreana que a do próprio \index{Sartre, Jean-Paul}Sartre,
  como vemos neste excerto: ``Quanto a nós, queremos constituir
  precisamente o reino humano como um conjunto de valores distintos do
  reino material. Mas a subjetividade que nós aí atingimos a título de
  verdade não é uma subjetividade rigorosamente individual, porque
  demonstramos que no cogito nós não descobrimos só a nós, mas também
  aos outros.'' (\versal{SARTRE}, J"-P. \emph{O existencialismo é um humanismo}.
  Tradução de Vergílio Ferreira. \emph{In}: \_\_\_\_\_\_. \emph{Os
  pensadores}. São Paulo: Abril Cultural, 1973, p. 21.)}

\chapter{A radicalidade do pensamento de Fanon}

De novo o misticismo às cegas, sem responsabilidade ou quase isso.
Depois de mais de sessenta anos desde que \index{Fanon, Frantz}Fanon nos legou sua obra,
estamos às voltas com velhos problemas: neorracismo\footnote{Aqui
  recorro à noção de \index{Balibar, Étienne}Étienne Balibar segundo o qual: ``O racismo,
  verdadeiro fenômeno social totalizador, se inscreve em práticas
  (formas de violência, desprezo, intolerância, humilhação, exploração),
  discursos e representações que são outros tantos desenvolvimentos
  intelectuais do fantasma da profilaxia ou da segregação (necessidade
  de purificar o corpo social, de preservar a identidade do ``eu'', do
  ``nós'', mediante a qualquer perspectiva de promiscuidade, de
  mestiçagem, de invasão), e que se articulam em torno de estigmas da
  alteridade (apelido, cor da pele, práticas religiosas).'' \versal{BALIBAR} \&
  \index{Wallerstein, Immanuel}\versal{WALLERSTEIN}. \emph{Race nation classe:} les identités ambigües.
  Paris: La Découverte, 1988, p. 31.} e busca do retorno às origens, o
que naturalmente são faces de uma mesma moeda.

Por meio de muita arrogância, pela primeira vez na história, operações
de polícia introjetada na psique dos potenciais descontentes se realizam
na busca de condenar qualquer voz dissidente ao estabelecido. De 1952 a
2018 o capital se transformou, se amoldou às circunstâncias. Cada crise
serviu para novo impulso. O negativo foi sua base de
sustentação.\footnote{Retirei essa ideia do importante livro de \index{Grespan, Jorge}Grespan:
  \versal{GRESPAN}, J. \emph{O negativo do capital:} o conceito de crise na
  política de Marx. São Paulo: Hucitec, 1998.}

E a voz cínica já se ergue: ``\index{Fanon, Frantz}Fanon nada tem mais a dizer, precisamos
nos ater às novas epistemologias''. Novas epistemologias? Defender a
filosofia banto sem o mundo banto? Trata"-se disso. Um escândalo, uma
regressão!

Ainda estamos aqui e ainda estamos vivos.

O mundo que acreditava ter esvaziado de significação seu entorno
descobre de repente o câncer em suas entranhas. Sob o signo da
catástrofe social, num horizonte francamente regressivo em que ``o tempo
do fim (da História) é antes de tudo um (novo) tempo de
guerra'',\footnote{\index{Arantes, Paulo E.}\versal{ARANTES}, \emph{op. cit.}, p. 63.} de repente, ouvem"-se
estalos de chicote nas costas de centenas de pretos na Líbia em plena
era informatizada.

Um exército de defensores do único mundo possível de prontidão se
apresenta com os seus comunicados midiáticos de uma vitória permanente.
O ópio do consumo paralisou a esquerda brasileira e as disputas
intestinais reduzem"-se à luta pela gestão da barbárie. A morte de
famílias inteiras naufragadas em uma balsa no Mediterrâneo, juntamente
com crianças sendo revistadas por soldados no Rio de Janeiro, apresentam
o coroamento da civilização.

Nunca houve tanta violência diária e, no entanto, nunca houve tanta
apatia. Com o retorno das desigualdades aos índices da era balzaquiana o
mundo torna"-se cada vez mais negro!

\begin{quote}
Transferência maciças de fortunas
para interesses privados, desapossamento de uma parte crescente das
riquezas que lutas anteriores tinham arrancado ao capital, pagamento
indefinido de dívida acumulada, a violência do capital aflige agora,
inclusive, a própria Europa, onde vem surgindo uma nova classe de homens
e de mulheres estruturalmente endividados {[}\ldots{}{]} Mais característica
ainda da potencial fusão do capitalismo e do animismo é a possibilidade,
muito distinta, de transformação dos seres humanos em coisas animadas,
em dados digitais e em códigos. Pela primeira vez na história humana, o
nome Negro deixa de remeter unicamente para a condição atribuída aos
genes de origem africana durante o primeiro capitalismo {[}\ldots{}{]} A este
novo carácter descartável e solúvel, à sua institucionalização enquanto
padrão de vida e à sua generalização ao mundo inteiro, chamamos o
\emph{devir"-negro do mundo}.\footnote{\index{Mbembe, Achille}\versal{MBEMBE}, \emph{op. cit.}, p. 18.}
\end{quote}

Se isso não é uma vantagem miraculosa -- como cinicamente meu ex"-mentor
sugeriu numa entrevista\footnote{Naturalmente trata"-se de uma ironia com
  \index{Santos, Frei David}Frei David Santos, fundador da Educafro que até mais ou menos 2005 era
  de esquerda, momento em que ainda muito jovem militava em suas
  fileiras; após ser absorvida pelo \versal{PT}, a Educafro abandonou sua postura
  de transformação social e tornou"-se \versal{ONG}. Segundo o Frei, ser negro
  agora é ter vantagem\ldots{} (``Concurso da prefeitura de \versal{SP} verifica cor da
      pele de cotistas aprovados'', \emph{Folha de São Paulo}, 26/06/2017.)}
-- pode ser o estopim de uma nova forma de sociabilidade. Quando um ou
outro estudante buscou refletir sobre \index{Fanon, Frantz}Fanon, então houve indícios de que
seu pensamento sobreviveu aos rebaixamentos, desvirtuações e
manipulações cínicas e nada ingênuas.

Quando uma parte do movimento negro renega \index{Fanon, Frantz}Fanon para aceitar de bom
grado a última nota conceitual elaborada nos porões do neoliberalismo,
isso só demonstra sua atualidade. E se esta discussão não é somente para
apresentá"-lo serve ao menos para lhe fazer, ainda que modestamente,
justiça. Por que \index{Fanon, Frantz}Fanon? Por que agora?\footnote{Estou me referindo à
  brilhante tese de \index{Faustino, Deivison Mendes}\versal{FAUSTINO}, D. M. \emph{Por que Fanon, por que
  agora?}: Frantz Fanon e os fanonismos no Brasil. 2015. Tese
  (Doutorado). Programa de Pós"-Graduação em Sociologia, Universidade
  Federal de São Carlos, São Carlos, \versal{SP}, 2015.} São questões que ressoam
no solo de um mundo que se ergue sob o signo da catástrofe da escravidão
moderna e num país cuja estrutura escravista está entranhada nas
instituições liberais e, pior, na formação psíquica dos indivíduos.
Nunca tivemos uma democracia racial, é fato, mas tivemos uma
``democracia racista''.

Quando \index{Fanon, Frantz}Fanon discute a linguagem e se depara com as patologias envoltas
da psique negra chega rapidamente à conclusão de que a patologia é da
própria sociedade colonizada. No campo da linguagem a ``racionalidade
universal'' demonstrou"-se justificadora da racialização da humanidade
para a manutenção e encobrimento da exploração capitalista. Da
biopolítica passou"-se à necropolítica; no Brasil nunca se tratou de
domínio dos corpos, mas sim das escolhas prementes de um estado de
exceção que escolhe quem pode morrer e ser invisibilizado.

Do mesmo modo, os místicos são só a cara da coroa de uma mesma moeda no
cofre do rentabilismo. Da Martinica às Ilhas Salomão, não há um modo de
sociabilidade que não esteja sob domínio do Império. Enquanto na Europa
grupos identitários voltam a reivindicar suas origens arianas, no Brasil
grupos débeis paulistas querem se separar do resto do país. Esses
sintomas, porém, não bastaram para demonstrar para alguns setores do
movimento negro a loucura patológica de reivindicar as origens\ldots{} ou
como disse \index{Fanon, Frantz}Fanon, a Origem\ldots{}

Depois de 60 anos de atraso em
relação ao formulado por \index{Fanon, Frantz}Fanon, poderia me perguntar: as figuras da
consciência não avançaram deste
lado do Atlântico? Permanecemos imobilizados? O desmoronamento do
``bloco socialista'' não deveria ser entendido como a inelutabilidade do
próprio processo de desmoronamento do sistema fundado pela escravidão
moderna? O retorno folclórico das caricaturas históricas não deu em
farsas, senão em comédias intragáveis.

A crise tornou"-se forma de governo.

Se a filosofia banta não conhece a miséria metafísica da Europa, a
miséria metafísica da Europa impôs seu mundo. É isso que \index{Fanon, Frantz}Fanon quer
demonstrar ao atacar as formas místicas e reacionárias que tentam
reviver aquilo que foi morto pela máquina. Se a existência dos bantos se
situa no plano do não"-ser é justamente porque sua sociedade é uma
sociedade fechada e ainda não tinha conhecido a violência da história.

Violência que se impôs a ferro nos calcanhares e fogo nos peitos. Não é
em vão que antes de comentar a filosofia banta, \index{Fanon, Frantz}Fanon cite um longo
trecho do desgraçado mundo do \emph{apartheid}. Num outro texto
desconhecido se lê: ``as raças que dividiam a humanidade de forma
irreversível sobrepõem"-se à igualdade dos cidadãos das cidades. A
realidade racial supera qualquer teoria do direito. Desse modo, a cada
raça cabe um lugar no mundo''.\footnote{\index{Santos, Gislene Aparecida dos}\versal{SANTOS},
  G. A. \emph{A invenção do ser negro}: um percurso das ideias que
  naturalizaram a inferioridade dos negros. São Paulo: Educ/Fapesp,
  2006, p. 53.} É por esse motivo que a universalidade pressuposta pelo
mundo colonial exclui de seu \emph{Todo} o negro. Com efeito, aquela
loucura de atingir uma universalidade mítica e imediata demarcando um
lugar próprio nessa cadeia, dada por formas de vida que desapareceram
com o choque colonial, sendo ingênua, é, contudo, amplamente
aproveitável por formas de mercado que em seu nicho se abre para o
afroempreendedorismo.

\index{Diop, Alioune}Alioune Diop, como representante
máximo desse tipo de posicionamento que tende ao universal sem mediação,
numa busca regressiva das origens, é ironizado: ``O preto se
universaliza, mas do Liceu Saint"-Louis, em Paris, um deles foi expulso:
teve a ousadia de ler Engels'';\footnote{\versal{FANON}, \emph{ibidem}.} e \index{Fanon, Frantz}Fanon
continua: ``Já adivinhamos \index{Diop, Alioune}Alioune Diop a perguntar"-se qual será a
posição do gênio negro no concerto universal. Ora, afirmamos que uma
verdadeira cultura não pode nascer nas condições atuais.''

E quais condições são essas? De 1952 a 2018 eles dirão: muita coisa
mudou. Nós diremos: muitas coisas mudaram, mas a exploração, e sua
consubstancial opressão, continua em escala ainda pior.

As novas coordenadas ideológicas efetivadas pelo ruir da modernidade a
partir dos anos 1970 são determinadas por dois pressupostos que
arrasaram quarteirões: por um lado, os direitos e valores tornaram"-se
historicamente particulares, não podem ascender à universalidade; por
outro, há a suspeita universalizada que destitui qualquer noção mínima
de corpo político que não aquela já estruturada pelo jogo eleitoral;
qualquer noção que esteja para além da ordem do dia é atacada como
ilusória e oportunista.

A loucura da busca da identidade hipostasiada só indica que o mundo do
trabalho ruiu.

A pergunta ``Por que \index{Fanon, Frantz}Fanon? Por
que agora?'' talvez, tenha nisso sua resposta. Já sabemos que tais
pressupostos são antagônicos à formulação de \index{Fanon, Frantz}Fanon.

O mecanismo fundado pela ideologia em tempos de capitalismo
financeirizado e altamente manipulatório não se baseia mais no
engajamento do indivíduo como sujeito capaz de alterar as coordenadas
pressupostas do esquema. Ironicamente é como se todos já estivessem
naquela universalidade do não"-ser banto.

O liberalismo em tempos de financeirização propõe uma espécie de direito
neutro que escapa da determinação social (a exemplo do que já impõe na
economia); estamos agora na esfera de um direito livre -- isto é, sem a
imposição da população -- que pode efetivar uma ordem política desejada
sem a necessidade de sujeitos políticos.\footnote{Podemos concluir que o
  componente jurídico que se seguiu ao golpe no Brasil se serve dessa
  noção.}

Logo, as soluções certas são reconhecidas pelo fato de que não precisam
ser escolhidas.\footnote{A esse respeito ver \index{Rancière, Jacques}\versal{RANCIÈRE}, J. \emph{O odio à
  democracia}. São Paulo: Boitempo, 2014.} Nada melhor que um técnico
para tirar as dúvidas; um governo dos mais capazes! Governar sem povo,
porque o próprio povo se tornou não apenas indiferente senão inútil para
o estabelecimento das vias do sistema, parece ser uma prerrogativa
acertada, pelo menos para a elite econômica. Aliás, tanto nos \versal{EUA} quanto
no Brasil essa ``verdade'' azeda o estômago.

Toda a questão das lutas é reduzida à esfera da visibilidade e da
representatividade, que tem seus lastros na própria forma de uma
democracia golpeada em época de um Eu"-empresa que impõe a concorrência
onde não há, ou não deveria haver. A identidade sem relação com o outro
é a bola da vez, enriqueça"-a e venda"-a como produto por meio de um
volumoso currículo de ações solidárias. Quer dizer, aceita"-se de antemão
a derrota para logo em seguida transformá"-la em triunfo mercadológico.

Ora, acima eu havia chamado a atenção para a noção de identificação;
agora, ela retorna em sua forma sintomática para refletirmos sobre os
dias atuais. Por um lado, o mercado aposta na identificação dos grupos;
não precisa existir democracia se nossa identificação for guiada por
líderes e técnicos capazes de fornecer o melhor para nós. Por outro, a
tentativa de questionar tais pressupostos, ainda que tenha razão, de
acordo com a ideologia dominante sempre acaba em assassinato, maior
opressão e desequilíbrio social, que pode pôr tudo a perder. Por fim, o
recado é claro: a transformação social internacional é uma utopia de
assassinos sedentos de poder.

A ideologia atual deixa evidente que as formas políticas capturadas pelo
mercado são só uma aliança oligárquica entre ciência e riqueza que exige
todo o poder. Os discursos que se voltam para os particularismos,
inclusive em toda a sua caricatura (quem não assistiu as propagandas
políticas de \index{Clinton, Hillary}Hillary Clinton?), retomam o velho princípio da filiação em
uma comunidade enraizada no sangue, na cor da pele, na
religião\footnote{Com esse diagnóstico da ideologia atual não fica
  difícil entender por que grandes setores da esquerda progressista
  nacional receberam de braços abertos um filme terrível e reacionário
  como \emph{Pantera Negra}. O filme apresenta claramente qual seria o
  projeto que poderia fomentar um desenvolvimento técnico e econômico
  sem precedentes -- segundo a visão imperialista, é claro: um Estado
  sem intervenção da democracia. Se o Negro até hoje viveu às margens da
  sociedade, oferta"-se a ele a adesão ao consenso eterno que repudia os
  conflitos antigos e dobra"-se às soluções dos especialistas que só
  podem discuti"-las com os representantes escolhidos pelos deuses que
  compõem a oligarquia.} e no respeito a todas, desde que elas não se
misturem\ldots{}

Atomizar as comunidades e indivíduos, apelar para a característica
particular, gerar identificação são as premissas básicas do controle
social exercido na \emph{era da emergência.} Posso ter contato com
outros grupos, mas sem estabelecer com eles relações, eis o pressuposto
posto do controle atual.

É como se estivéssemos na prisão do seriado de \emph{Orange is the New
Black},\footnote{Seriado famoso que teve a lucidez de demonstrar em toda
  sua força a incapacitante noção de identidades estanques e
  não"-relacionais no interior da prisão. Todas as formas de controle
  social e gestão da miséria ficam estampadas em sua narrativa sob um
  único atributo: defender as comunidades negra, latina, branca e
  religiosa sem deixar que elas se relacionem. O recado é bem claro:
  estamos todos numa prisão sendo conduzidos por gestores da miséria.
  Falar sobre esse seriado, contudo, equivaleria a um capítulo à parte
  que fugiria de nosso tema.} cujos gestores fossem nossos
governantes e cada um tivesse sua comunidade própria e não se
misturasse.

Naturalmente, numa sociedade forçosamente miscigenada como a nossa, tais
imposições do mercado imperialista entrariam em curto"-circuito.
Espalhar essas ideias por aqui
tem encontrado um terreno insólito cujo adubo é paradoxalmente fornecido
pela classe média letrada, que, dentro de seus confortáveis
apartamentos, tornam o \emph{tour} pela favela algo exótico.

Mas, a lucidez ainda brada:
``{[}\ldots{}{]} o problema negro não
se limita ao dos negros que vivem entre os brancos, mas sim ao dos
negros explorados, escravizados, humilhados por uma sociedade
capitalista, colonialista, apenas acidentalmente branca.''\footnote{\versal{FANON},
  \emph{op. cit.}, p. 170.}

Ao retirar o essencialismo e a cristalização categorial, \index{Fanon, Frantz}Fanon permitiu
pensarmos para além dos limites pressupostos pelo jogo. A radicalidade
de seu pensamento ecoa ainda hoje com a lucidez que golpeia o misticismo
e a obscuridade, que, infelizmente, grassa em grande parte do mundo
contemporâneo\ldots{}

\partepigraph{As situações de vantagem ou desvantagem de uma ou outra raça no sistema
capitalista de países específicos decorre de contextos históricos. Razão
pela qual, para extinguir os males advindos do capitalismo, não adianta
querer ``identitarizar'' o capitalismo. (\index{Seale, Bobby}\emph{Bobby Seale})}{}
\part{O MOVIMENTO NEGRO E O MISTICISMO}
\removeepigraph

\chapter{Os limites do problema}

Este pequeno ensaio não trata de uma pesquisa histórica, nem tampouco
sociológica. O esforço aqui é o de esboçar uma história do
desenvolvimento das ideias para, com isso,
desmistificar posições teóricas
que se tornaram força material e impregnaram as ações de grande parte do
\emph{Movimento Negro}.

Mas, repetindo uma pergunta feita lá atrás por \index{Gonzalez, Lélia}Lélia Gonzalez: será
possível falar do Movimento Negro?\footnote{\versal{GONZALEZ} \& \index{Hasenbalg, Carlos}\versal{HASENBALG}.
  \emph{Lugar de negro}. Rio de Janeiro: Marco Zero, 1982.}
Sabemos que o Movimento Negro é
só uma abstração para indicar lutas que se baseiam na compreensão da
estrutura racial do país, e nas formas ou de minimizar tais resultados
-- grande parte dos setores -- ou de superá"-la -- posições minoritárias.
Ele é múltiplo e mais dinâmico que nossas categorizações.

A posição hegemônica, aquela de minimização dos resultados catastróficos
de uma estrutura econômico"-social racializada, contribuiu para o
desenvolvimento das lutas em diversas frentes, obtendo, não neguemos,
alguns êxitos contra os resultados devastadores e assassinos do racismo
brasileiro. Daí que a história do \versal{MNU} (\emph{Movimento Negro Unificado})
é necessariamente o filtro por onde as ideias hegemônicas passaram.

A despeito dos avanços obtidos, principalmente no que se refere ao
debate sobre o racismo, entretanto, a violência e o assassinato das
pessoas negras aumentaram nas periferias. No exato momento em que
grandes setores do movimento negro foram cooptados para a máquina
burocrática do Estado, sob a égide do \versal{PT}, houve juntamente o aumento
exponencial do número de assassinatos de jovens no Brasil.\footnote{\emph{Cf}.
  \emph{Atlas da Violência} 2017.}

A Região Nordeste, abandonada
pelos olhares clínicos dos especialistas, vive hoje em estado de guerra
permanente. Na Região Norte, em um único presídio tivemos a morte de 56
presos em condições das quais os ``bárbaros'' se
envergonhariam.\footnote{``Matança em presídio de Manaus é uma das maiores
  desde Carandiru'', \emph{G1}, 02/01/2017.}

Os números de assassinatos compreendidos entre 2005 e 2015 tiveram um
aumento de 17,2\% entre indivíduos de 15 e 29 anos. Com 59 mil
assasinatos por ano, dos quais a maioria recai nas costas da população
negra. Vivemos uma catástrofe social difícil de encontrar parâmetros no
cenário mundial.

Essa dolorosa realidade solapou a ideia, veementemente combatida pelo
\versal{MNU}, de \emph{democracia racial}. Hoje, graças ao empenho de seus atores
políticos e à gritante realidade racialmente cindida, qualquer pessoa
com razoável coeficiente cognitivo não leva a sério essa noção.

Também os resultados da era lulista não foram tão bons para aqueles que
vivem nas periferias deste país. Com todo o ódio de classe jogado nas
costas da população negra, não me parece mais uma opção apostar todas as
fichas no jogo político posto nos limites da representatividade forjada
e hegemonizada pela oligarquia financeira. O paradoxo é que tampouco
pode"-se abandoná"-lo.

Se fracassamos, esta crítica é uma tentativa de fracassar melhor.
A representação política só se
torna legítima em condições de igualdade democrática na disputa, algo
nunca existente na maneira como o jogo eleitoral se processou desde a
refundação democrática, quando o componente econômico e controle da
mídia por oligopólios se tornaram decisivos.

No olhar retrospectivo se comprova que toda aquela tentativa de
hegemonia à esquerda recaiu em regressão: uma violência desmesurada
impregnada pelas formas de administração e controle dos corpos impresso
pela economia de mercado. Uma verdadeira calamidade social que se
refletiu numa necropolítica violenta e da qual o Estado de exceção é só
uma normalidade histórica e colonial.

Como diz \index{Mbembe, Achille}Achille Mbembe num diagnóstico preciso dos novos desdobramentos
de organização social sob a égide do capitalismo contemporâneo:

\begin{quote}
{[}\ldots{}{]} os novos processos de racialização visam marcar esses grupos
de populações, fixar do modo mais preciso possível os limites no seio
dos quais elas podem circular, determinar do modo mais exato possível os
espaços que elas podem ocupar, em suma, assegurar as circulações num
sentido que permita afastar as ameaças e assegurar a segurança geral.
Trata"-se de selecionar esses grupos de populações, de os marcar a um
tempo como ``espécies'', ``séries'' e como ``casos'', no seio de um cálculo
generalizado do risco.\footnote{Achille Mbembe em \emph{A universalidade
  de Frantz Fanon}.}
\end{quote}

A invasão das favelas com tanques do exército só confirma a tese.

Ora, se só no fim se compreende o começo, nossa aposta é precisamente
demonstrar os pontos de suturas e cordialidade com o \emph{status quo}
que foi levado adiante por grande parte da esquerda. Penso que a crítica
radical à hegemonia identitária, em sua limitação à esfera de
visibilidade representativa no interior do atual sistema, é mais que
central.

Não se pode esquecer, contudo, que a necessidade da identidade é na
verdade a resposta política a um sistema cuja universalidade é
excludente. Tenhamos calma e não caiamos no engodo de reduzir tudo à
\emph{universalidade} da classe. Para evitar isso, \index{Fanon, Frantz}Fanon e seu
pensamento especulativo, como ficou evidente linhas acima, nos trouxe
grandes lições.

A compreensão da necessidade do particular demonstra que é a própria
incompletude que fomenta a totalidade. Dizíamos acima que a existência
precede a essência: nada mudou.

O \emph{Outro} contraditório é o que nos constitui a partir do momento
em que expressa uma consciência negativa ao nosso Eu=Eu. Nossa
identidade, momento no qual a experiência fundamenta uma subjetividade,
depende do elemento externo negativo que necessita de uma outra
consciência.\footnote{Temos uma interpretação desse duelo entre
  consciências que foi sinalizado por \index{Cossetin, Vânia Lisa Fischer}Cossetin, quando diz que: ``A
  ênfase na corporeidade marca o encontro das consciências"-de"-si e tem
  como principal objetivo o desejo de reconhecimento cuja origem está
  num confronto que expressará a ascensão da consciência sobre a sua
  existência corporal objetiva e meramente natural. Por tal razão é que
  o estremecimento corpóreo será a experiência do modo e da sensação da
  consciência"-de"-si para si mesma, a experiência da diferença consigo
  mesma e também o meio imediato pelo qual ela se manifesta para outros,
  forma de sua objetividade. Para que o reconhecimento efetivamente
  ocorra, contudo, é necessário que ambas as consciências se deem conta
  de que seu agir e existência dependem uma da outra.'' (\emph{Cf}. \index{Cossetin, Vânia Lisa Fischer}\versal{COSSETIN},
  V. L. F. \emph{A dissonância do absoluto:} linguagem e conceito em
  Hegel. Ijuí: Editora Unijuí, 2012, p. 101). Tal interpretação destaca
  um movimento interessante a respeito da questão dos corpos, na luta
  entre Senhor e Escravo; no entanto, perde de vista a coisidade e o
  trabalho como coerção e, ao mesmo tempo, realização da consciência.
  Mais profunda nessa questão, a postura de \index{Butler, Judith}Judith Butler -- ao refletir
  sobre a questão dos corpos, sem perder de vista a categoria do
  trabalho -- permite"-lhe enxergar a questão da subordinação ilimitada
  empreendida pelo ato de trabalhar e pela falta do ato declamatório,
  justificando assim a fuga do estoicismo etc.} O \emph{Um} como o
negro se apresenta é imediatamente desdobrado em si mesmo. Ela (a
identidade) sempre é mutável e vazia, como a posição do sujeito, é
descentrada, disjuntiva em um ponto entre \emph{ser} e \emph{não"-ser}
que a sustenta e vai se sustentando por meio da experiência.

Por esse motivo, o hipotético"-leitor já deve ter se dado conta de que a
apreciação de \index{Fanon, Frantz}Fanon, que fizemos na primeira parte, serve como fio
condutor desta crítica. Sendo \index{Fanon, Frantz}Fanon nosso Vírgílio, não tenhamos medo de
adentrar as orlas daquilo que concebemos como \emph{Movimento Negro}
hegemônico.

Não farei aqui, como disse, uma história desse ``movimento''; me
limitarei somente a demonstrar como os componentes teóricos fermentados
por seus intelectuais nos guiaram até esta sinuca de bico em que, por um
lado, nos digladiamos pelas misérias ofertadas pelo poder, enquanto, por
outro, massacres perpetrados pelo Estado ocorrem impunemente e somos
incapazes de uma resposta efetiva.

\chapter{Contra o misticismo do trabalho libertador}

Um dos diagnósticos que já
podemos fazer de saída é que grande parte da tradição marxista ortodoxa
apostou numa vinculação total do componente racial à estrutura
produtiva.

O construto teleológico concebido como progresso, a noção de um
desenvolvimento histórico em linha ascendente e a aposta no trabalho
como suposta libertação incapacitaram uma crítica mais fecunda que
decifrasse a forma como o componente de \emph{desigualdade racial} fora
integrado ao modo de funcionamento do capital.

O engodo causado por esse tipo de noção é acreditar que há uma
progressão histórica necessária capaz de absorver as demandas produzidas
no seio da sociedade civil burguesa, quando, pelo contrário, essas
produções de demandas são o próprio movimento no interior dessa
sociedade na forma de seu excedente perpétuo comandado pelo capital.

Isso significa que a estrutura do capital no seu elemento, para citar
velhos fantasmas, \emph{desigual} e \emph{combinado}, produz anomalias
sociais que na verdade fornecem ao modo de produção e reprodução social
os componentes necessários para manter o grau de valorização do capital
intacto, sua taxa de lucro e, com isso, a realização do processo total
de circulação.

O processo pelo qual a produção e reprodução de mercadorias
historicamente determinadas pelo trabalho repõem seu contínuo movimento
é também aquele que fornece os componentes necessários para a realização
da circulação. A questão é que a
própria mercadoria estabelece a forma de relação social e não há por
detrás dela nenhum conjunto de relações senão o da própria mercadoria.

Historicamente, enquanto o negro era escravo, isto é, enquanto sua mão
de obra não se constituía como mercadoria, ele permanecia no interior de
uma relação de exploração total. A partir do momento em que o capital
pagou pela sua força de trabalho, então, o antigo escravo passou a ser
descartável.

Uma pequena mostra do trabalho de \index{Fernandes, Florestan}Florestan Fernandes diz um pouco desse
processo brutal em que a liberdade recém"-conquistada se confundiu com
uma invisibilidade social radical. ``Eliminado para os setores residuais
daquele sistema, o negro ficou à margem do processo, retirando dele
proveitos personalizados, secundários e ocasionais.''\footnote{\index{Fernandes, Florestan}\versal{FERNANDES},
  F. \emph{A integração do negro na sociedade de classes}: o legado da
  raça branca. São Paulo: Globo, 2008, p. 36.} Noutras palavras, a
possibilidade de integração do negro na sociedade foi solapada pelo
processo de concorrência com o branco.

O problema é que esse processo concorrencial no mercado de trabalho
jamais se esgotou, e a composição do próprio mercado fundamentou áreas
nas quais o artifício racial se tornou determinante. Não é difícil
perceber a discrepância racial até hoje existente em serviços como o
doméstico -- verdadeira herança escravagista -- e os setores da
construção civil, só para ficar em dois exemplos.

Voltando a balizar nossa crítica à história das ideias, é preciso
concluir daí que a famosa \emph{dialética do senhor e escravo}
hegeliana, mal interpretada por grande parte dessa tradição, foi
positivada e seus resultados foram vistos com olhar otimista. O que
quero dizer com isso é que houve sempre uma posição de otimismo em
relação aos resultados do desenvolvimento do trabalho por parte da
vulgata marxista.

Foi só com \index{Butler, Judith}Judith Butler que um outro modo de interpretação foi
construído a partir da noção de trabalho e sujeição.\footnote{\versal{BUTLER}, J.
  \emph{A vida psíquica do poder:} teorias da sujeição. Trad: Rogério
  Bettoni. Belo Horizonte: Autentica, 2017.} Evidenciando de saída os
problemas encontrados na noção de progressão histórica e o \emph{status}
de um sujeito que resume em si a universalidade, \index{Butler, Judith}Butler retoma a noção
de \emph{dominação e escravidão} de um ponto de vista distópico, que
muito tem a nos ensinar.\footnote{Não se pode esquecer, contudo, de
  Althusser, que pensa a noção de sujeição como interpelação, por
  exemplo, algo que será determinante para a crítica de \index{Butler, Judith}Butler.
  (\versal{ALTHUSSER}, L. \emph{Ideologia e aparelhos ideológicos de Estado}. \emph{In}: \index{Žižek, Slavoj}\versal{ŽIŽEK}, S. (Org.) \emph{Um mapa da ideologia}. Rio de Janeiro: Contraponto, 1996, p. 105-142).}

Em primeiro lugar, invertidos os
sinais da equação senhor/escravo, como resultado negativo não há uma
saída luminosa para o escravo no interior dessa dialética. ``O escravo
surge como corpo instrumental cujo trabalho provê as condições materiais
da existência do senhor, e cujos produtos materiais refletem tanto a
subordinação do escravo quanto a dominação do senhor.''\footnote{\index{Butler, Judith}\versal{BUTLER},
  \emph{op. cit.}, p. 43.}

Em segundo lugar, a própria noção de corpo instrumental tem um
significado importante para nós: o negro reduzido a cor da pele, e
desumanizado em sua humanidade, sofre no próprio corpo os resultados da
exploração imposta pelo senhor. Aliás, ele se reduz ao corpo (não nos
esqueçamos da profunda discussão de \index{Fanon, Frantz}Fanon sobre a genitália).

Num primeiro momento o escravo é posto como mero instrumento e,
portanto, reflexo do senhor. É como se na relação colonizada estivesse
preso à descorporeidade do branco. O branco é um desejo sem corpo que
obriga o negro a agir como seu corpo. Porém, o escravo sabe que não age
como uma extensão do corpo do senhor e por isso sabe que pode ser um
agente autônomo. No entanto, nessa lógica, o escravo ainda age como mero
reflexo, pois ainda se encontra no interior da estrutura de dominação
imposta pelo senhor.

É essa forma de mero reflexo que deve ser negada pelo escravo -- temos
aqui uma voz de \index{Fanon, Frantz}Fanon, que por outros caminhos chega ao mesmo resultado.
Nos pressupostos dados pelo senhor, foi concedida uma falsa autonomia; a
ação do escravo permanece presa no interior da lógica colonizadora.
Quanto mais ele concebe para si uma autonomia por meio do seu
trabalho,\footnote{Trabalho = \emph{tripalium}: Três paus, a que eram amarrados os
  animais bravios, alimentados no mínimo vital, até se tornarem mansos,
  domesticados.} mais escravo se torna. Com efeito, as formas de
exploração via trabalho são o próprio limite a ser superado, isto é, a
estrutura mesma da relação capital/trabalho deve ser implodida.

\index{Hegel, G. W. Friedrich}Hegel deixa claro que a liberdade conquistada pelo escravo nesse
processo não é verdadeira e, por isso, ele se volta para o interior de
si mesmo\ldots{}

Ao contrário dessa exposição, porém, o trabalho passou a ser o sumo bem,
e quem é capaz de trabalhar passou a ser glorificado. Logo, o vagabundo,
identificado como aquele que não encontra trabalho, torna"-se um cidadão
de segunda ordem, visto com desconfiança geral e diminuído em sua
humanidade.

Nos processos coloniais o negro foi aquele que, alijado dos processos
modernizantes da indústria e comércio, tornou"-se estigmatizado pela
teologia do trabalho. Podemos lembrar aqui as diversas formas como a
malandragem e a vadiagem se tornaram um componente epidérmico
estruturado a partir do racismo naturalizado e, como efeito, até hoje o
negro é visto como um vagabundo até que prove o contrário.\footnote{A
  análise que \index{Candido, Antonio}Candido empreende sobre \emph{Memórias de um sargento de
  milícias} é sem dúvida um dos caminhos mais lúcidos para se
  compreender como a malandragem efetiva um componente de integração que
  inclusive coloca em risco algumas das ``virtudes'' da metrópole
  (\versal{CANDIDO}, A. Dialética da Malandragem (caracterização das Memórias de
  um sargento de milícias), \emph{Revista do Instituto de Estudos
  Brasileiros}, nº 8, São Paulo, \versal{USP}, 1970, p. 67-89.}

Com um processo de integração racial totalmente reduzido ao processo de
valorização e produtividade do capital, a absorção da mão"-de"-obra negra
nunca foi um requisito necessário e, por isso, a cor da pele tornou"-se,
para as polícias, sinônimo de trabalhador ou vagabundo.

Numa sociedade estruturada em nome do trabalho, e que carrega como
emblema da sua bandeira o horroroso lema \emph{ordem e progresso}, esse
aspecto estrutural de sua organização caiu como chicote do \emph{capitão
do mato} no lombo do negro.

O \emph{etos} protestante do trabalho, desenvolvido por uma gama de
teóricos, sublinhou ao máximo ``a oposição entre trabalho e
não"-trabalho''.\footnote{Uma das críticas mais contundentes à teologia do
  trabalho aparece na obra de \index{Jappe, Anselm}Jappe. (\emph{In}: \versal{JAPPE}, A. \emph{As aventuras
  da mercadoria}: para uma nova crítica do valor. Lisboa: Antígona,
  2006).} A questão é que tal oposição não ficou somente entre
trabalhadores/capitalistas, mas também entre trabalhadores/não
empregados.

Igualmente, o componente racial como atributo absorvido pelo mercado de
trabalho fundamenta uma divisão no próprio seio da classe trabalhadora.
Assim, não são os negros que são divisionistas, senão o próprio sistema
que integra a massa trabalhadora. Podemos perceber como a questão do
negro adentra essa perspectiva por um prisma antagônico e alheio àquela
noção de dignidade pelo trabalho professada por grande parte da
esquerda.

Aliás, talvez um dos fatores comuns em ambos os espectros políticos é o
elogio acrítico ao trabalho. Tanto a esquerda quanto a direita
fetichizam a sujeição que atende pelo nome de trabalho.

A posição do negro, contudo, paradoxalmente se firmou fora daquela
unilateralidade da defesa de classe confundida com a defesa da labuta. A
questão aqui é um tanto mais complexa: a defesa de uma sociedade do
trabalho nega a sua oposição constitutiva: o capital. Separados os dois
polos reciprocamente antagônicos que fundamentam a realidade
contemporânea -- trabalho/capital -- chegou"-se numa superestimação de
uma sociedade do trabalho cuja verdade é o pórtico de
Auschwitz.\footnote{Lê"-se no pórtico: O trabalho dignifica o homem.}

Quando digo isso espero ressaltar toda a tragédia humana envolta na
noção de trabalho como dignidade do homem em seu artifício racializado
sem, no entanto, deixar de indicar a necessidade do trabalho para
sobrevivência comum enquanto estivermos sob o império da mercadoria.
Sabemos que a noção de raça, juntamente com a defesa religiosa do
trabalho durante a Segunda Guerra mundial, foi o componente central da
\emph{solução final}. Cumprir o dever do trabalho e defendê"-lo cegamente
são questões das quais os resultados jamais deveriam ser esquecidos.

\begin{quote}
Na questão da cooperação, não havia diferença entre as comunidades
altamente assimiladas da Europa Central e Ocidental e as massas falantes
do iídiche no Leste. Em Amsterdã assim como em Varsóvia, em Berlim como
em Budapeste, \emph{os funcionários judeus} mereciam toda confiança ao
compilar as listas de pessoas e de suas propriedades, ao reter o
dinheiro dos deportados para abater as despesas de sua deportação e
extermínio, ao controlar os apartamentos vazios, ao suprir forças
policiais para ajudar a prender os judeus e conduzi"-los aos trens, e
até, num último gesto, ao entregar os bens da comunidade judaica em
ordem para confisco final.\footnote{\index{Arendt, Hannah}\versal{ARENDT}, H. \emph{Eichmmam em
  Jerusalém}. Tradução de José Rubens Siqueira. São Paulo: Companhia das
  Letras, 1999, p. 134 (grifos meus).}
\end{quote}

Ora, então chegamos em dois resultados: primeiro, aquela noção de
absorção do componente racial pelo modo de produção capitalista se
revelou falha, uma vez que o modo de produção e reprodução do capital
absorveu o componente racial em sua estrutura, delegando locais e áreas
``especiais'' para os pretos. Segundo, a própria defesa acrítica de uma
sociedade do trabalho fez com que a crítica se mantivesse nos limites
impostos pelo próprio capital, considerando essa estruturação
racialmente desigual como algo sem muita importância.

Paradoxalmente, porém, isso não indica que o capital não possa absorver
o \emph{negro}; é a sua estrutura competitiva e em si mesma vazia (quer
dizer, o capital é uma fantasmagoria, uma abstração real) que promove a
disputa entre o todo social. ``A Ordem simbólica não é apenas sempre"-já
pressuposta como o âmbito único da existência social do sujeito: essa
própria Ordem só existe, só é reproduzida na medida em que os sujeitos
nela se reconhecem e, por repetidos gestos performativos, nela assumem
reiteradamente seus lugares.''\footnote{\index{Žižek, Slavoj}\versal{ŽIŽEK}, S. \emph{O sujeito
  incômodo:} o centro ausente da ontologia política. Tradução de Luigi
  Barichello. São Paulo: Boitempo, 2016, p. 281.}

Com isso posto, já podemos compreender as limitações com que temos que
lidar atualmente no interior do Movimento Negro. Se, por um lado,
abandonar a esfera da visibilidade é um suicídio que implica vidas e
formas de subsistência, por outro, tomar a desgraça por redenção, isto
é, adotar as limitações impostas pelo modo como o poder econômico se
estrutura com essa esfera de visibilidade, é se colocar ombro a ombro
com a exploração.

O que vou demonstrar doravante é que na maioria das vezes os teóricos do
Movimento Negro aqui no Brasil adotaram essa postura.

\chapter{A origem do mito e a construção de um epígono }

Apesar de tudo que foi exposto poder soar como novidade, essa impressão
é um engano. Há muito tempo se estuda a questão negra e a questão da
África de um ponto de vista que coloca em xeque aquilo que foi tido
tradicionalmente como ``África''. Não se trata de inventar a roda, mas
de retomar uma discussão para empreender uma nova força material que nos
leve a questionar o \emph{status quo}.

Foi \index{Mudimbe, Valentin-Yves}Mudimbe quem buscou descontruir aquilo que se convencionou chamar de
Africanismo, evidenciando como o problema de pensar a África já é um
problema ocidental, e o adjetivo \emph{africano} já é uma invenção para
controle dos corpos negros.\footnote{\versal{MUDIMBE}, V. Y. \emph{A invenção da
  África:} gnose, filosofia e a ordem do conhecimento. Luanda: Edições
  Pedago, 2013.}

Salienta"-se, contudo, que as posições, em geral, mais do que meras
diferenças epistemológicas, se revelam como diferenças políticas. Por
trás de uma posição teórica que pode soar ingênua há mecanismos
conscientes e inconscientes que podem se tornar força material de gestão
e controle, ou de quebra da gestão e do controle.

Os exemplos disso já estão desenhados na própria história recente do
continente africano. Muitos de seus países conseguiram emancipar"-se
apenas no século \versal{XX} e grande parte das teorias deram corpo às diversas
práticas. Práxis diversas que muito sangue derramaram e seguem abertas
em suas variações.

É desanuviando esses princípios que podemos nos acercar do instigante
debate sobre África sem cair naquele romantismo reacionário e hipócrita
que vem tomando grande parte do debate no cenário conservador atual. A
famosa onda conservadora atinge todos os meridianos ideológicos. Desconfiemos dos heróis, portanto.

Para entender a posição de \index{Nascimento, Abdias do}Abdias do Nascimento é preciso apreender seu
contexto histórico eivado de disputas teóricas que aconteciam na vida
política do continente africano.

Por um lado, existia um pensamento hegemônico que buscava resgatar as
supostas raízes africanas e consolidar um Estado africano, ou seja lá o
que for. Algo que pudesse reunir os negros do globo distanciados pela
diáspora violenta motivada pela invasão dos colonizadores. Como este
texto não deixa de ser expressão política, considero esse pensamento
muito próximo da direita e a partir da década de 70 totalmente integrado
às formas de gestão da barbárie capitalista. O triunfo desse pensamento
significou a ruína e assassinato dos \index{Negras, Panteras}Panteras Negras.

Por outro lado, existia um pensamento minoritário encarnado
principalmente nas figuras de \index{Fanon, Frantz}Frantz Fanon e Aime Cèsaire, que
reivindicavam uma nova universalidade a partir da destruição
sociossimbólica impressa pelo domínio colonial burguês. Isto é, a busca
de uma efetiva universalidade, que suprimisse a exploração do homem pelo
homem, já que o embrião e desenvolvimento capitalista impediu o
florescimento de uma comunidade livre e igual de pessoas e constitui uma
universalidade excludente mediada pela e para a mercadoria.

Naturalmente, Abdias, coerente com sua práxis política iniciada na
\emph{Frente Negra Brasileira} e desdobrada na \emph{Ação Integralista
Brasileira,} será adepto da primeira.\footnote{A Ação Integralista
  Brasileira era fascista, não tinha só tons fascistas como alguns
  ``abdistas'' fazem supor. Tinha um componente totalmente autoritário,
  e sua busca de integração passava pela uniformização social com base
  no sonho da Ordem e Progresso. Respondia por um nacionalismo
  chauvinista que de fato nunca saiu da teoria de Abdias, seja por uma
  África mística, seja por um Brasil cuja identidade negra funde guetos
  separados travestidos de quilombos (``A importância de \index{Nascimento, Abdias do}Abdias do
      Nascimento para a história do Brasil'', \emph{Brasil de Fato}, 10
  novembro 2014).}

Seja como for, analisar as múltiplas contradições daquilo que ficou
conhecido como \emph{Pan"-Africanismo}\footnote{É fato que também o que
  se chamou pan"-africanismo é heterogêneo e existem posições à esquerda
  e à direita do processo. Du Bois e Marcus Garvey são as figuras
  proeminentes que representam esse antagonismo político"-ideológico no
  seio do pan"-africanismo. O primeiro, de esquerda, o segundo, admirador
  do fascismo.} seria uma tarefa titânica e fugiria ao tema central
deste pequeno ensaio, que visa tão somente desmistificar posições
enraizadas no Movimento Negro atual. Mas cumpre dizer que no interior do
Pan"-Africanismo existiram e existem várias correntes que vão da extrema
esquerda à extrema direita.\footnote{Os epígonos de ambos os espectros
  políticos no interior das lutas pan"-africanistas são respectivamente
  Du Bois, à esquerda, e Garvey, à extrema direita.}

Se nem tudo que reluz é ouro, nem toda posição que se aparenta como
progressista de fato o é. Muitos movimentos de libertação nacional na
África em sua multiplicidade acabaram por consolidar e manter a
estrutura de exploração do capital. Quando isso encontrou estruturas
arcaicas de fetichismo religioso, ao contrário das previsões
``civilizatórias'', o que se viu foi uma radical exploração em nome
desse fetichismo.\footnote{Não se pode esquecer da terrível tradição
  milenar escravagista da Mauritânia, por exemplo, onde mais de cento e
  cinquenta mil pessoas são escravas.}

\index{Fanon, Frantz}Fanon, até o fim, combateu esse tipo de posição lutando ao lado da
\emph{Negritude}.\footnote{A Negritude foi um importante movimento de
  poetas e críticos que se encontraram na Sorbonne: \index{Cesaire, Aime@Césaire, Aimé}Aimé Césaire, René
  Depestre, Léopold Sédar Senghor são alguns dos principais nomes que
  depois se converterão em revolucionários nas guerras de independência.
  Este último tinha uma diferença radical com as posições de \index{Fanon, Frantz}Fanon, pois
  reivindicava posições estritamente identitaristas. Também o fato de
  ter surgido da Sorbonne não é algo pouco importante. É daí que surge a
  noção de uma África idealizada.} Hoje, se não dá mais para manter
ilusões, convém mostrar como elas foram engendradas. Como esse
romantismo conservador foi estruturado é a primeira forma de combate.

\chapter{Em busca da África perdida?}

Foi só com \index{Mudimbe, Valentin-Yves}Mudimbe que as posições identitárias começaram a ser
questionadas, para não falar da desconstrução da noção
\emph{pan"-africanista;} por isso, não é de se espantar que esse filósofo
seja pouco conhecido por aqui. Fazendo uma espécie de história das
ideias sobre a África, \index{Mudimbe, Valentin-Yves}Mudimbe desnuda o condão que funda uma noção
mística de África homogênea.

É em meados do século \versal{XIX} que, sobre o cavalo branco do iluminismo, cada
vez mais as ditas ciências humanas se voltam para a compreensão dos
africanos. A \emph{ethnè}\footnote{``Povo'' em grego.} é então
particularizada e funda uma nova doutrina: a \emph{etnologia}. Arma de
guerra a serviço das metrópoles, \index{Mudimbe, Valentin-Yves}Mudimbe nos faz lembrar que a etnologia
já era um saber branco ocidental que atribuía uma diferença
(inferioridade) aos demais povos. Já era, portanto, uma racialização que
impunha formas de tratamento e sociabilidade diferentes de acordo com os
aspectos fenotípicos.

Sobretudo no continente africano, a etnologia foi utilizada como um
discurso que buscava fundar uma alteridade africana particular e
homogênea. Se, por um lado, procurava descrever os modos e vivências dos
nativos, por outro, objetivava uma verdadeira política de domesticação
dos modos e costumes dos povos. No registro da dominação, o negro de
início é apresentado como um fenômeno diferente, como um ser estranho e
anódino, algo que carecia de explicação racional.

Esse exercício de verdadeira colonização era empreendido por duas
medidas reciprocamente complementares demonstradas por \index{Mudimbe, Valentin-Yves}Mudimbe:
a) por meio da análise das ``instituições'' nativas; b)
através da busca do homem ingênuo rousseauniano; o bom selvagem como uma
figura ideológica no interior do grande continente.

Por aí já é possível perceber como um discurso alienígena funda uma
noção de África que será levada adiante principalmente pelos negros da
diáspora.\footnote{Aqui marco essa distinção para evidenciar que grande
  parte da população do continente africano permaneceu aquém dessas
  categorizações.}

\begin{quote}
Poder"-se"-á pensar que esta nova configuração histórica significou, desde
as suas origens, a negação de dois mitos contraditórios: nomeadamente, a
``imagem hobbesiana de uma África pré"-europeia, onde não existia a noção
de Tempo; nem de Artes; nem de Escrita; uma África sem Sociedade; e,
pior ainda, marcada pela perpetuação do medo e pelo perigo de uma morte
violenta'': e ainda a ``imagem rousseauniana de uma era africana
dourada, plena de liberdade, igualdade e fraternidade''.\footnote{\index{Mudimbe, Valentin-Yves}\versal{MUDIMBE},
  \emph{op. cit.}, p. 15, \emph{apud} \versal{HODGKIN}, 1957, p. 174-5.}
\end{quote}

É desse modo que a velha antropologia é erguida para lidar com o
``primitivo''. Unindo as descrições das normas, as formas da consciência
e a tentativa de captar a projeção individual com o aporte das ciências
naturais, os estudiosos europeus buscaram compreender e designar a
estrutura cognitiva dos africanos, coisificando"-os.

São esses estudiosos que vão erguer o mito de estruturas pré"-lógicas no
homem africano e caracterizar o Ocidente como o local da fria razão, ao
passo que a África é o local dado às estruturas sensitivas e intuitivas.
Lugar onde seus indivíduos detêm uma estrutura pré"-lógica dominada pelas
formas de representação coletivas estritamente dependentes da
participação mística.

Ora, já vemos como alguns mitos vão sendo erguidos por, desculpe"-me o
leitor ter que frisar, \emph{intelectuais brancos pagos para categorizar
e definir}. Mitos que serão abraçados acriticamente inclusive por quase
todos os membros da \emph{Negritude}.

Assim, os estudos etnológicos empreendidos por uma antropologia
interessada terminam num etnocentrismo ideológico e conceitual como
forma de controle das populações do continente. A naturalização
etnográfica contribuiria para uma noção de \emph{raça}\footnote{Ao
  reduzir o corpo e o ser vivo a uma questão de aparência, de pele ou de
  cor, outorgando à pele e à cor o estatuto de uma ficção de cariz
  biológico, os mundos euro"-americanos em particular fizeram do Negro e
  da raça duas versões de uma única e mesma figura, a da loucura
  codificada (\index{Mbembe, Achille}\versal{MBEMBE}, \emph{op. cit.}, p. 11).} que seria coroada como uma
forma de natureza irredutível e particularizada.

É nessa posição antinômica -- entre uma capacidade de cognição lógica e
uma pré"-lógica -- que mesmo teóricos da envergadura de \index{Levi-Strauss@Lévi-Strauss, Claude}Lévi"-Strauss se
veem no interior de uma armadilha implicada em descrever o Outro sem
nele reconhecer"-se. No entanto, se, por um lado, os problemas levantados
pelo grandioso \index{Levi-Strauss@Lévi-Strauss, Claude}Lévi"-Strauss ficam às voltas com a antinomia produzida
pelo conhecimento antropológico, por outro, ele é o primeiro a
demonstrar a não existência de selvagens contrapostos aos
civilizados.\footnote{\versal{LÉVI"-STRAUSS}, C. \emph{O pensamento selvagem}.
  Tradução de Tania Pellegrini. Campinas: Papirus, 1989.}

Com arcabouço filosófico as noções do Outro e de si"-mesmo utilizadas por
\index{Levi-Strauss@Lévi-Strauss, Claude}Lévi"-Strauss já não são meras sombras de uma \emph{epistème} vazia, mas
estruturas conceituais que partem da relação sociossimbólica concreta.
Nesse sentido, as análises de \index{Levi-Strauss@Lévi-Strauss, Claude}Strauss são uma superação que retira da
Europa ``civilizada'' o monopólio da razão e cultura.

Sem \index{Levi-Strauss@Lévi-Strauss, Claude}Lévi"-Strauss, ficam expostos os pressupostos da adesão romântica e
hiperfetichista de uma África desenhada e narrada por uma antropologia
que tinha sobretudo a missão civilizatória de domesticação do
``primitivo''. \emph{A invenção da África} redefiniu o quadro teórico e
consequentemente a ação política no interior dos ignorados países
africanos. Comentá"-la sinteticamente aqui serve para abrir caminhos à
crítica do pensamento, sobretudo, diaspórico, que criou uma África
imaginária ora como local de realização de um mundo sem exploração -- à
esquerda -- ora como um local de soberania estatal guiado pelo
componente identitário, quando não monárquico (Wakanda?) -- à direita.

Uma África sensual e sensível, pré"-lógica e marcada pela intuição foi o
reino dos céus fornecido por uma literatura interessada no controle e na
fundamentação racial. Esse mito se manteve e, infelizmente, no
descompasso brasileiro em seu atraso em relação ao centro dos debates
vem ganhando força ultimamente.

É fácil prever que os desdobramentos dessas ideias são mais dialéticos
do que as considerações imanentes que se fazem sobre elas. Naturalmente,
as delimitações colonizadoras do que vem a ser \emph{África} quando
tomadas pelos colonizados foram, em raríssimos casos, subvertidas em seu
sentido.

Contudo, é agora, no momento em que escrevo, que um novo tipo de
questionamento começa a ser esboçado e ser reconhecido pela sua
capacidade e alcance universal num registro realmente
emancipatório (e aqui falo, sobretudo, de \index{Mbembe, Achille}Achille Mbembe).

\chapter{Uma ilusão necessária contra um mito perigoso}

Quando vozes feéricas erguem bandeira a favor da proibição dos
relacionamentos entre os povos, o cheiro de enxofre polui o ar e diante
da pupila fantasmas mal desencarnados voltam a dançar.

Ora, chocar"-se com esse tipo de posição não deveria ser monopólio apenas
daqueles que lutam por uma sociedade igualitária e livre, senão de todos
aqueles que conhecem minimamente um pouco de história. Sabemos onde isso
acabou.

Há no argumento um construto lógico que aproxima rapidamente a
inter"-relação negro/branco do genocídio. De fato, a violência radical do
processo de \emph{miscigenação} não pode nem deve ser
esquecida. O objetivo aqui é
eliminar a ideia de que os relacionamentos atuais sigam os mesmos termos
violentos do processo de colonização sem, contudo, cair na armadilha,
igualmente funesta, de ver nos processos inter"-raciais fonte de
reconciliação racial. Nem tanto ao mar, nem tanto à terra. O que irá
definir a superação do \emph{status quo} definitivamente não serão os
casamentos.

Se a própria noção de miscigenação carrega o estigma das raças e foi
utilizado pelas elites como uma forma de ocultar a violenta
discriminação por meio do mito da ``democracia racial'', também esse
mito retirou da crítica a capacidade de pensar as possibilidades de
superação efetiva do quadro proposto pelo colonialismo.\footnote{A
  própria noção de ``mulato'' advém de um meio termo entre negro e
  branco. Como fica exposto na carta do racista \index{Lacerda, João Batista}João Lacerda: nem tão
  ``inferior'' quanto o negro, nem superior ao branco\ldots{} É preciso
  lembrar como tais princípios estavam expostos no processo de
  republicanização do Brasil, acompanhando inclusive o lema da bandeira.
  Em seu ``magnânimo'' artigo dedicado ao Marechal Hermes da Fonseca,
  \index{Lacerda, João Batista}João Batista Lacerda, médico e cientista de ``grata estirpe'', assim
  finalizava: ``a importação -- sim como objeto -- em uma vasta escala,
  da raça negra ao Brasil, exerceu influência nefasta sobre o progresso
  deste país: ela retardou por muito tempo seu desenvolvimento material,
  e tornou difícil o emprego de suas imensas riquezas naturais''.
  \index{Lacerda, João Batista}\versal{LACERDA}, João Batista. \emph{Sobre os mestiços no Brasil}. Primeiro
  Congresso Universal das Raças. Londres. 26. 9 de julho de 1911. Logo
  se vê como a fundação e a tentativa de justificar a ``mestiçagem'' é
  tardia e veio para tentar aplacar um processo inexorável, tentando
  justificá"-lo à sombra da ciência positivista. O artigo completo
  demonstra de forma factível todo o racismo envolto nas análises
  teóricas que justificavam práticas políticas voltadas em sua maioria
  para a separação entre raças. Nosso ``valente'' doutor -- espero que
  tenhamos humor para entender ironias -- foi lá para tentar demonstrar
  como o aspecto da mistura poderia ser bom a partir da aniquilação do
  componente negro da sociedade. Felizmente, faltou"-lhe experimentação
  histórica.}

Resta, contudo, claro que o processo de miscigenação foi acomodado pelos
princípios positivistas que se baseavam na noção de inferioridade
biológica e na aposta de uma limpeza étnica dos genes num longo prazo.
Camuflou"-se assim a violência sexual contra as mulheres negras e
retirou"-se do horizonte a concreta contribuição dos negros para o
desenvolvimento da colônia.

Contudo, o processo cientificista positivista naufragou e a miscigenação
resultou no maior contingente populacional negro fora da África. Já são
54\% de negros no Brasil. População que vagueia pelas ruas sofrendo a
violência diária da polícia, que tem o ``trabalho'' de eliminar um dos
maiores \emph{exércitos de reserva de negros}\footnote{Por
  \emph{exército de reserva negro} entenda"-se um dos maiores
  contingentes populacionais negros desempregados e precarizados do
  mundo.} do mundo.

Aquilo que anima a crítica de \index{Nascimento, Abdias do}Abdias do Nascimento é sua contraposição
radical e produtiva à noção de ``democracia racial''. A tensão
propriamente existente entre a miscigenação e o construto freyriano de
uma lânguida mistura entre as raças que culminaria num paraíso racial
são os termos que serão, com razão, demolidos pelas análises do autor de
\emph{Genocídio do negro brasileiro}.\footnote{\versal{NASCIMENTO}, A. \emph{O
  genocídio do negro brasileiro:} o processo de um racismo mascarado.
  Rio de Janeiro: Paz e terra, 1978.} A questão básica a ser vista é
que não se pode opor simplesmente os dois extremos,
\emph{miscigenação}/\emph{identidade}, e postular uma interação entre
eles.

A vida tanto objetiva quanto
subjetiva oscila entre uma negatividade radical perturbando o equilíbrio
socialmente existente e impondo uma nova ordem sociocultural que busca
estabilizar a situação. Quero dizer com isso que se, por um lado,
\index{Freyre, Gilberto}Gilberto Freyre se prende em unilateralidade -- por ter uma visão
conservadora do processo e se apegar demasiadamente rápido à
universalidade violenta e excludente, por outro, contrapor esses termos
através de um apelo à manutenção da identidade, ainda que seja muito
mais coerente e eficaz para a transformação efetiva da ordem excludente,
reflui em unilateralidade se não ultrapassa as limitações da própria
individualidade limitada ao interior da ordem estabelecida.

Por isso, há dois pontos interconectados que precisamos determinar: 

\begin{enumerate}
\item
A autoformação do ser, enquanto ser social, não reside apenas na adaptação
a uma forma cultural pré"-estabelecida, seja ela no continente africano
ou nas colônias; essa formação ocorre quando se resiste aos próprios
limites impostos por ela; 
\item
A própria noção de raça incide na cor por
meio de uma espécie de necessidade ontológica vazia do \emph{negro}: não
basta falar ou confundir os dois termos, o negro e a constituição da
ficção raça, é a identidade dos dois termos que indica uma contradição
radical que dinamiza os processos sociais tanto na colônia, como no
continente africano, como ainda na própria Europa.
\end{enumerate}

Usando um refinado arsenal crítico, Abdias, porém, não conseguirá sair
dessa dicotomia, que, embora traduza elementos produtivos e fecundos,
algo como uma abertura crítica radical, o restringe à elevação
fetichista da nacionalidade, da identidade e da preservação cultural
como cerne da práxis política.

Com saudável iconoclastia, \index{Nascimento, Abdias do}Nascimento desmonta o mito da
\emph{miscigenação} conservadora assentando sua análise no chão
histórico. Como demonstra, sobre as mulheres negras sobejou o peso desse
fetiche, porque foram elas as que mais sofreram os abusos sexuais na
colônia.

É interessante notar que desde a década de 1960 até hoje houve um
aumento de mais de 200\% nos casamentos inter"-raciais, que passaram de
8\% na década de 1960 para 31\% em 2010;\footnote{A pesquisa foi
  realizada para um pós"-doutorado de Lia Vainer Schucman na \versal{USP} em 2017,
  com apoio da Fundação de Amparo à Pesquisa do Estado de São Paulo
  (\versal{FAPESP}), colaboração de Felipe Fachim e supervisão de Belinda
  Mandelbaum, coordenadora do Laboratório de Estudos da Família do
  Instituto de Psicologia (\versal{IP}) da \versal{USP}.} contudo, a estratificação
social e um sexismo subjacente a esses casamentos permanecem sendo a
toada das uniões. Sexismo porque a maioria dos casamentos são entre
homens negros e mulheres brancas, algo como um troféu ou vingança tão
bem explicitado por \index{Fanon, Frantz}Fanon.\footnote{\versal{FANON}, \emph{op. cit.}} Estratificação
social, porque eles ocorrem somente entre membros das classes populares,
indicando o racismo na estrutura de classes.

Paradoxalmente a miscigenação atual é algo interno ao proletariado.

Como esta análise visa desnudar o que está presente na ordem do
discurso, podemos concluir que o choque da maioria das pessoas frente à
noção de antimiscigenação advém, sobretudo, da universalidade ideológica
que a miscigenação diz promover. Acentuando a ideologia por detrás dessa
noção, \index{Nascimento, Abdias do}Abdias do Nascimento coloca sobre o teto a violência implicada no
mito da ``democracia racial''.

\begin{quote}
Postula o mito que a sobrevivência dos traços da cultura africana na
sociedade brasileira teria sido o resultado de relações relaxadas e
amigáveis entre senhores e escravos. Canções, danças, comidas,
religiões, linguagem, de origem africana, presentes como elemento
integral da cultura brasileira, seriam outros tantos comprovantes da
ausência de preconceito e discriminação racial dos brasileiros
``brancos''.\footnote{\versal{NASCIMENTO}, \emph{op. cit.}, p. 55.}
\end{quote}

Quando se desnudam as relações promíscuas e violentas que são ocultadas
pelo termo \emph{miscigenação} um curto"-circuito parece ocorrer. A
suposta universalidade parece conter a cor \emph{branca} e funcionar com
a total subordinação do negro como um negativo indesejado. Assim, o
negro fica fora do circuito fechado dessa universalidade, sendo aceito
somente enquanto perda de si mesmo, enquanto negação de sua identidade.

Isso nos leva ao centro de nossos questionamentos: a identidade surge
como resultado de um paradoxo entre o presente (conscientemente
refletido) e um passado (memória).
A rememoração é o componente
fecundo que por meio do desvio pelo passado constitui nossa própria
experiência do presente. Com efeito, a história -- o cadáver -- é
disputada na formação da própria subjetividade. \index{Nascimento, Abdias do}Abdias do Nascimento
segue essa trilha agarrando"-se ao resgate de uma individualidade
despedaçada e suprimida.

Afirmo que a posição de \index{Nascimento, Abdias do}Abdias do Nascimento é uma disputa firme e
valente pelos cadáveres que sucumbiram no Oceano, mas é preciso
ressaltar que a interação entre passado e presente precisa ser mais que
``interações e relações'', precisa de interpenetração e se tornar uma
autorreferência capaz de abrir o horizonte para aquilo que era até então
posto como impossível.

O que quero dizer é que se, por um lado, de fato a noção de miscigenação
foi a borracha que visava apagar os crimes ocorridos no interior da
colônia, cujos resultados estão impregnados no nosso cotidiano, por
outro, essa demonstração não precisa incidir no retorno abrupto de uma
identidade \emph{não relacional}. A identidade é um fenômeno relacional
que advém da interação entre conjuntos diferentes de atividades no mundo
circundante. Ela é sempre aberta, historicamente determinada e algo
passageiro.

\asterisc

Também as formas de controle sobre a cultura e religiosidade africana
serão marcas da intransigente crítica de \index{Nascimento, Abdias do}Abdias do Nascimento. A
religião tornada caso de polícia, a explosão de terreiros, as prisões
arbitrárias compõem o enredo surdo de uma desfaçatez que se une à
ideologia alcunhada de ``democracia racial''. A perseguição que destituía
o negro de seus recursos simbólicos torna"-se razão de Estado que o
impede de se achegar à compreensão de si mesmo.

O controle radical e violento exercido sobre as formas de expressão
religiosa -- para se ter uma ideia, os terreiros na Bahia só precisaram
deixar de ser registrados na polícia em 1976\footnote{\versal{NASCIMENTO}, \emph{op.
  cit.}, p. 105.} -- foram uma forma de submeter o negro para que não
pudesse resistir ao modo de vida imposto pelas formas de exploração e
opressão colonizadoras. Essas máculas vieram acompanhadas da noção de
sincretismo.

Tal noção oculta não só uma violência que desestrutura a própria matriz
simbólica do negro, como oculta também um processo de resistência que
vai sendo aos poucos elaborado a fim de manter vivas as tradições
religiosas. Agarrado aos elementos constitutivos de sua religiosidade, o
negro tenta manter sua estrutura, que, como resistência, provoca a
suspeita do poder metropolitano, atingindo até mesmo a era republicana.

Esse passo dado por \index{Nascimento, Abdias do}Abdias do Nascimento, embora tenha elementos que
incidem numa reação ao choque cultural, ergue também uma disputa por
essa cultura. É preciso, no entanto, salientar que o processo de
autorreferência mantida pela religião destacada do seu lugar originário
designa o momento em que a atividade religiosa -- e com ela a identidade
-- não circula mais em torno do local que a produziu, mas gera seu
próprio ``rito''. A consciência negra passa a pôr a si mesma como algo
\emph{outro} do que fora, isto é, a produzir"-se sob nova condição.

É essa impossibilidade teórica de \index{Nascimento, Abdias do}Abdias do Nascimento, causada
sobretudo porque parte das limitações concernentes à estrutura ligada à
raça como elemento hipostasiado, que o faz cometer algumas
idiossincrasias conceituais e o faz regredir perigosamente o escopo de
abertura que sua própria crítica havia possibilitado.

Noutros termos, quando se aferra ao resgate puro de uma cultura
impossibilitada, porque historicamente destroçada, nosso crítico fecha a
abertura crítica, com implicações políticas seríssimas, que sua própria
análise propicia. Em todo seu caminho, o vemos debater"-se com esses
limites e mesmo afirmar que somente por uma revolução as limitações da
raça poderiam ser solapadas. No entanto, ele não sabe se essa revolução
é racial ou social, embora chegue aos mesmos resultados de \index{Marx, Karl}Marx no
\emph{Manifesto do Partido Comunista}:

\begin{quote}
O ponto de partida da classe dirigente branca foi a venda e compra de
africanos, suas mulheres e seus filhos; depois venderam; o sangue
africano em suas guerras coloniais; e o suor e a força africanos foram
vendidos, primeiramente na indústria do açúcar, no cultivo do cacau, do
fumo, do café, da borracha, na criação do gado. {[}\ldots{}{]} ``Venderam'' os
espíritos africanos na pia do batismo católico assim como, através da
indústria turística, comerciam o negro como folclore, como ritmos,
danças e canções. A honra da mulher africana foi negociada na
prostituição e no estupro. Nada é sagrado para a civilização ocidental
branca e cristã. Teria de chegar a vez da venda dos próprios deuses. De
fato, os orixás estão sendo objeto de recentes e lucrativas
transações.\footnote{\versal{NASCIMENTO}, \emph{op. cit}., p. 119.}
\end{quote}

Este curto"-circuito ocorre porque \index{Nascimento, Abdias do}Nascimento tem uma noção identitária
de identidade. Para ele sem identidade resta a alienação. Abdias não vê
que a alienação é constitutiva da identidade como processo de
organização vazia, contínua e permeada de colisões que a estruturam e a
reestruturam. Nesse sentido, por vias outras comete o mesmo erro
daqueles que veem a alienação como um empecilho para chegar ao ser
\emph{em"-si}. Por isso, ele quer dar corpo à descorporizada \emph{classe
dirigente}, quer dar cor ao sistema de visibilidade que funciona sob o
modo de exploração.

Isso marca a diferença radical entre \index{Nascimento, Abdias do}Abdias do Nascimento e \index{Fanon, Frantz}Frantz
Fanon. Este último sabia que a substancialidade de qualquer significante
perene que possa fundamentar uma plenitude do Eu está barrada pela
própria forma como a consciência se põe no mundo. Ou melhor, é lançada
no mundo.

Se a ilusão necessária de \index{Nascimento, Abdias do}Abdias do Nascimento vai até a possibilidade
de desnudar o tema tabu \emph{democracia racial}, a partir do momento em
que busca o império da identidade como não relacional, sua abordagem
torna"-se \emph{reativa,} culminando na adoção utópica -- ou seria
distópica? -- de um Estado africano \emph{por vir}.

É assim que sua \emph{análise estética,} reduzida sempre ao conteúdo
explícito, torna"-se equivalente à censura de \index{Platão}Platão aos poetas. Se a
forma artística tem autonomia quanto ao contexto e é uma tentativa de
unir elementos dispersos e heterogêneos no mundo e configurá"-los pela
sua união, num todo artístico criado, então ela responde por esse mundo
demonstrando"-o em toda a sua hipocrisia, paixão, vício e
virtude.\footnote{\index{Lukács, György}\versal{LUKÁCS}, G. \emph{A teoria do romance}. São Paulo:
  Editora 34, 2010.}

A autonomia da arte implica sua liberdade de trazer à luz aquilo que é
produzido à sombra. Faz isso na reunião de elementos heterogêneos,
dispersos e que indicam que não há sentido dado para a vida senão aquele
produzido na ação humana. Por isso, quando Abdias condena os artistas
negros por não se expressarem enquanto tais, perde de vista não só as
possibilidades que se evidenciam nas formas impressas por esses
artistas, como também recai num proselitismo rebaixado.

A obra de arte responde por seu mundo, se esse mundo é racista a obra de
arte conterá tais elementos, ou subvertendo"-os, ou demonstrando"-os, ou
ainda fazendo ambos. Impedida de chegar em tais conclusões, dado o ponto
de partida identitário, a crítica de \index{Nascimento, Abdias do}Nascimento a \index{Assis, Machado de}Machado de Assis é
medíocre. Para ele, Machado de Assis é simplesmente um aculturado que
privilegiou a classe média branca em suas obras.

Reduzindo a obra de \index{Assis, Machado de}Machado à superfície aparente -- classe social e
epiderme --, não consegue compreender as tensões e críticas radicais
contidas na fina ironia de \index{Assis, Machado de}Machado de Assis. Nem sequer se dá conta da
formação racista, hipócrita e violenta contida em seus personagens
aparentemente brandos e eruditos. Quer dizer, \index{Nascimento, Abdias do}Abdias do Nascimento não
se dá conta da crítica mordaz machadiana, que desnuda as relações
raciais cujo cinismo sangrento assenta as raízes de uma elite plutocrata
e racista que se mantém no poder até hoje.\footnote{A esse respeito o
  clássico ensaio de \index{Schwarz, Roberto}Schwarz ensina muito (\versal{SCHWARZ}, R. \emph{Um mestre
  na periferia do capitalismo}. São Paulo: Editora 34, 2000).}

Há, sem dúvida, diferença entre, por um lado, a busca da identidade
(\emph{relacional}) como fundamento de uma subjetividade atuante nos
processos sociais e, por outro, o identitarismo (identidade
hipostasiada) como elevação de características originárias que não podem
``se misturar''.

O discurso de \index{Nascimento, Abdias do}Abdias do Nascimento se funda nessa tensão, que, ao
afirmar a identidade e desnudar os mitos empreendidos pela ``democracia
racial'', se vê atolado na busca de uma raiz inexistente. Como resultado,
fundamenta um discurso ideológico cujo conteúdo ganha força material
pelo seu vazio de significado e torna"-se facilmente cooptável pelas
forças hegemônicas.

São tais forças que com sua
ideologia atestam a luta entre os conteúdos particulares promovendo a
universalidade ideológica capaz de estabelecer uma organização social.
Essa universalidade está sempre em disputa, daí a importância da
afirmação da identidade desde que se veja nela um elemento evanescente,
isto é, formado negativamente pelo embate com o \emph{Outro} negativo.
Como vimos, não é essa identidade que está em cena no desenvolvimento
teórico de \index{Nascimento, Abdias do}Nascimento.

Como parte de um pressuposto local e de uma identidade estanque, Abdias
ignora as particularidades tanto dos descendentes de africanos da
diáspora quanto daqueles negros em continente africano. A
particularidade de um negro brasileiro fatalmente se contrapõe à
particularidade de um negro em Angola. Na relação de ambos o elemento
epidérmico se desfaz para se projetar o ambiente socioeconômico e
cultural que forma suas individualidades. \index{Fanon, Frantz}Fanon sempre ilustrou a
diferença entre um negro da Martinica e outro de qualquer colônia, isto
porque, contrariamente a \index{Nascimento, Abdias do}Abdias do Nascimento, era anti"-essencialista.

\chapter{Uma crítica necessária}

É interessante notar como as análises empreendidas e o apontamento sobre
a violência colonial feitos por Abdias são verdadeiros no que se refere
ao desmascaramento da ideologia da ``democracia racial'', enquanto os
pressupostos por trás de suas conclusões, todavia, jogam no campo da
integração como um componente afirmativo.

A integração do negro se baseia, desse modo, na prioridade da
preservação da cultura, da tradição, dos costumes e do desenvolvimento
identitário de sua particularidade.\footnote{O hipotético"-leitor atento
  já percebeu que esta é a bula papal do integralismo\ldots{}} Não é a
exploração constituída pelo sistema, que arrancou os negros do
continente africano, a chave da dominação e da exploração da população
negra, segundo \index{Nascimento, Abdias do}Nascimento, mas, sim, o ``roubo'' de sua identidade pelo
embranquecimento social.

A questão sintomática da posição de \index{Nascimento, Abdias do}Abdias do Nascimento é que a
proclamação da diferença se limita ao quadro referencial posto. Ora, se
muitas vezes ``o desejo da diferença emerge precisamente dos lugares
onde se vive mais intensamente a experiência de exclusão'',\footnote{\index{Mbembe, Achille}\versal{MBEMBE},
  \emph{op. cit.}, p. 304.} a possibilidade de ultrapassá"-la, enquanto gesto de
poder, necessita estar intrínseca a um projeto mais vasto.

É natural que para aqueles que passaram pelo processo de dominação
colonial e supressão de sua alteridade, a proclamação da diferença seja
central, e daí o apelo da identidade. Entretanto, ela só pode
ultrapassar seus limites quando se põe no sentido de endossar um mundo
livre do peso da raça, que é, sobretudo, uma construção colonialista.

A raça, ao contrário das previsões otimistas do marxismo vulgar, foi
tomada como componente estrutural no mercado de trabalho e não absorvida
equitativamente por este. Abdias também sabia disso, todavia, restrito à
metafísica da raça (metafísica, no sentido aqui atribuído, como um
componente fetichista que impõe uma dinâmica social) busca reafirmá"-la
ao procurar integrar o componente racial como possibilidade de sanar a
desigualdade.

E nesse ponto se encontra todo o construto que fecunda suas análises:
\index{Nascimento, Abdias do}Nascimento torna o conceito de raças \emph{a"-histórico}.

Quando faz o debate histórico, parte de uma unilateralidade latente cujo
mérito foi o de desmontar a noção de ``democracia racial''. Quando parte
para o debate econômico, para se livrar rapidamente da noção de luta de
classes, entrega"-se ao interior dos modos de operacionalização social
efetivados pelo processo de produção e reprodução do capital sem
questionar sua sociabilidade baseada na organização e divisão entre as
raças, as classes e as nacionalidades.

Não sendo ingênuo suficiente para negar as divergências e complexas
diversidades dos grupos africanos que foram arrancados de suas terras,
\index{Nascimento, Abdias do}Nascimento, para justificar seu pan"-africanismo, necessita de um
componente que possa unificá"-los sem comprometer sua argumentação:
\emph{a religião}.

As chamadas ``culturas irmãs'', quer dizer, um grupo imenso e
heterogêneo, são reduzidas à insígnia da oralidade e politeísmo. Tal
como os primeiros etnólogos fizeram, parte"-se aqui de um mito formulado
pela ``ciência''. É, por fim, o Candomblé o sinal de unidade -- mesmo
havendo dicotomias entre os Yorubas, os Ewe do Benin e do Congo, etc.

A argumentação torna"-se toda etnóloga e utiliza os elementos místicos
como fundadores de uma nova forma de subverter o \emph{status quo}. É
óbvio que as religiões de matriz africana sofreram o policiamento
grotesco enviesado por uma política racista. Defender a liberdade de
culto é uma premissa básica para uma construção democrática.

Também a conclamação por um retorno aos vínculos orgânicos não é o
problema em \index{Nascimento, Abdias do}Abdias do Nascimento, senão sua formulação entre esse
suposto retorno e a busca de uma mobilização tecnológica capaz de fundar
uma ideologia corporativista estetizada: a busca de um Estado que põe
para si mesmo o elemento de modernização radical ligado à defesa
orgânica de vínculos étnicos.

\index{Nascimento, Abdias do}Abdias do Nascimento, ao desnudar o racismo institucionalizado na
sociedade brasileira, não vai além das premissas básicas. Não despe as
relações de poder baseadas nas formas do desenvolvimento econômico que
fundamentam as instituições. E embora questione a construção ideológica
do racismo como uma forma necropolítica, que divide os cidadãos naqueles
que podem ser mortos e os que não podem, reivindica a participação no
projeto em que o capitalismo se torne mais humano e não
racista.\footnote{\index{Mbembe, Achille}\versal{MBEMBE}, A. \emph{Necropolítica}: biopoder, soberania,
  estado de exceção, política da morte. São Paulo: \versal{N}-1 Edições, 2017.}

Parando a meio caminho, identifica a dominação como algo do branco
europeu e em sua retórica não identifica o modo de sociabilidade
capitalista como o cerne a ser combatido, senão a luta pela afirmação
cultural como \emph{leitmotiv} da ``transformação'' social.

Essencializando um inimigo por meio da epiderme, o poder derivado das
forças econômicas em jogo desaparece e o capitalismo não é mais o
problema, mas a ferramenta que se usa para a ``manutenção racial
branca''.\footnote{\versal{NASCIMENTO}, A. \emph{Quilombismo}. São Paulo: Editora
  Vozes, 1980, p. 16.} Basta portanto tomar esse poder. Em sua
argumentação culturalista e etnóloga, o que impede a integração do negro
na vida coletiva é o não reconhecimento dos vínculos culturais e
ideológicos entre os afrodescendentes da diáspora e os africanos.

Ora, tudo isso não quer dizer que Abdias não faça a crítica ao
capitalismo. O problema é que ele oculta que o espírito foi transformado
em capital no próprio continente africano e uma vez ocorrido o processo
em que o termo \emph{nativo} foi empregado para aqueles que estão
enterrados em seu local de nascimento, não há retorno. O processo é
irreversível.

Sabotando a roda violenta da história, \index{Nascimento, Abdias do}Abdias do Nascimento crê na
possibilidade de salvaguardar as origens de uma África existente somente
nos delírios de etnólogos, como possibilidade de redimir inclusive o
opressor:

\begin{quote}
A restituição aos africanos daquilo que era antes unicamente seu, neste
momento histórico de crise aguda do capitalismo, apresenta
necessariamente implicações de relevante função ecumênica. Pois uma vez
mais a redenção do oprimido em sua plena consciência histórica, torna"-se
em instrumento de libertação do opressor encurralado nas prisões a que
foi conduzido pela ilusão da conquista.\footnote{\emph{Idem}, p. 42.}
\end{quote}

Para justificar suas posições,
parte de uma premissa que traçará a rota de suas argumentações até o
fim: \emph{as culturas africanas}. Sem dizer o que são tais culturas ou
demonstrar a multiplicidade de culturas existente no continente
africano, o líder do \versal{MNU} apenas dirá que elas estão fundamentadas na
organização social coletiva, criatividade, redistribuição e propriedade
de forma equitativa, princípios meramente liberais e atualmente
bem"-vindos. A propriedade não
deixa de existir nas supostas ``culturas africanas'', muito embora
propriedade seja um substantivo criado, mantido e defendido a ferro e
fogo pelos colonizadores.

\begin{quote}
As culturas africanas são aquilo que as massas criam e produzem: por
isso elas são flexíveis e criativas, assim como bastante seguras de si
mesmas, a ponto de interagir espontaneamente como outras culturas,
aceitando e incorporando valores científicos e/ou progressistas que por
ventura possam funcionar de modo significativo para o homem, a mulher e
a sociedade africana.\footnote{\emph{Idem}, p. 46.}
\end{quote}

Seja o que for, o entendimento de culturas africanas é genérico e não
leva em conta a multiplicidade da África. A África para Abdias é um
país, ou melhor, uma pátria. Seu projeto de unidade pan"-africana é,
sobretudo, nacionalista;\footnote{A esse respeito é preciso assinalar que
  o Pan"-Africanismo foi heterogêneo em suas posições políticas. A
  começar pelos seus principais idealizadores Marcus Garvey à direita do
  processo e William Edward Burghardt Du Bois à esquerda.} visa
edificar o \emph{ser nacional}:

\begin{quote}
Na estrutura da presente fase da ``ajuda técnica'' as formas avançadas
de tecnologia do capitalismo industrial, além de não cooperar na
construção, em verdade instigam e promovem a penetração do capital
monopolístico internacional e a alienação do autoconhecimento nacional.
Esta ``ajuda'' tecnológica e científica estará apta a tomar os rumos da
libertação somente quando os valores capitalistas que regem e regulam
seus mecanismos não forem utilizados para deter o desenvolvimento da
consciência dos povos e da independência nacional.\footnote{\emph{Idem}, p. 73.}
\end{quote}


Com tais posições, \index{Nascimento, Abdias do}Abdias do Nascimento não era só contra o pensamento
de \index{Marx, Karl}Marx, ele era na verdade antimarxista.\footnote{Isso pode iluminar, a
  meu ver, suas escolhas e posições políticas durante o regime militar.
  \index{Nascimento, Abdias do}Abdias do Nascimento preferiu o \versal{PDT} ao \versal{PT} -- por este na época ser
  muito classista. Do mesmo modo, nunca se arrependeu de ter sido um
  integralista e foi até o fim coerente com sua posição juvenil,
  amadurecendo"-a ao longo dos anos e encontrando no pan"-africanismo,
  nacionalista e identitário, os resultados conceituais de que
  necessitava.} Radicalmente contrário ao pensamento dialético que
impõe, além de outras coisas, a necessidade de se mesurar o choque de
culturas e o desenvolvimento histórico destas em suas trocas; para ele
os intelectuais que partissem das análises de \index{Marx, Karl}Marx fracassariam ao
tentar compreender o desenvolvimento das raças. Hipostasiada a noção de
raça em sua safra ``conceitual'', é óbvio que \index{Nascimento, Abdias do}Nascimento vê em \index{Marx, Karl}Marx um
forte oponente ao seu misticismo retórico.

Segundo ele, foi ``\index{Marx, Karl}Marx [quem] substituiu a categoria humana dos
africanos pela categoria econômica''\footnote{\emph{Ibidem}.} e não o
capitalismo. E triunfante conclui: ``não aceitamos que uma pura mágica
conceitual possa apagar a realidade terrível da opressão dos brancos
europeus contra todo continente e sua raça negra''\footnote{\emph{Ibidem}.}
apaga"-se com isso o componente socioeconômico e no seu lugar \index{Nascimento, Abdias do}Abdias do
Nascimento ergue o princípio cultural essencialista. Não é o capital o
problema, mas o branco europeu.

Ora, sabemos que o anseio por uma vida autêntica em comunidade não pode
ser reduzido ao significante de anseios totalitários; há no caráter
utópico e não ideológico dessa posição algo que deve ser afirmado. A
dificuldade nessa posição é como tais anseios serão articulados e
funcionalizados.

Se se partir de um ponto especifico ante a exploração totalizadora do
capital (o domínio do capital financeiro, a ``influência judaica'', a
``epiderme'' dos indivíduos que compõem as elites, a ``influência dos
estrangeiros'' no desmonte da nacionalidade, etc.,) fatalmente se
deixará de engajar"-se numa transformação estrutural para atuar no
interior da limitação posta. É aquilo que \index{Bernardo, João}João Bernardo chamou de
revolta na/pela ordem.\footnote{\emph{Cf}. \versal{BERNARDO}, J. \emph{Labirintos do
  fascismo:} na encruzilhada da ordem e da revolta, 2015.}

Ora, é exatamente por isso que a ilusão necessária proposta por \index{Nascimento, Abdias do}Abdias
do Nascimento, em seu desnudamento do racismo estrutural brasileiro,
encontrou seus limites que precisam mais que nunca ser ultrapassados e
criticados.

Por outro lado, não se pode jogar o bebê junto com a água, pois as
tecnologias de reificação do negro conduziram a processos históricos que
o dilaceraram em sua humanidade. Por isso, a noção que \index{Nascimento, Abdias do}Abdias Nascimento
tem sobre as políticas de reparação se aproxima daquela necessidade,
vislumbrada por \index{Mbembe, Achille}Mbembe, de recuperação social dos laços que foram
quebrados e de instauração de uma alteridade recíproca, sem a qual a
possibilidade de uma consciência comum do mundo estaria vedada.

\chapter{Ao pé do muro}

Somos convidados pelas circunstâncias históricas a desafiar os limites
socialmente impostos. A cada vinte e três minutos um jovem negro é
assassinado neste país.\footnote{Mapa da Violência da Faculdade
  Latino"-Americana de Ciências Sociais (Flacso).} No momento em que as
balas do Estado perfuraram o corpo frágil de \index{Franco, Marielle}Marielle\footnote{\index{Franco, Marielle}Marielle,
  a quinta vereadora com mais votos no Rio de Janeiro, era negra,
  lésbica, socialista e ativista dos direitos humanos; foi assassinada
  logo após ser nomeada para monitorar a intervenção federal no Rio de
  Janeiro e denunciar a violência policial.} acabou"-se qualquer ilusão
com as limitações da representatividade. Nossa resiliência enfim se
esgotou.

Se ``o processo histórico foi, para grande parte da nossa humanidade, um
processo de habituação à morte do outro'',\footnote{\index{Mbembe, Achille}\versal{MBEMBE}, \emph{op. cit.}, p.
  305.} teremos que reinventar o próprio sentido de comum, superando as
lesões sem deixar cicatrizes, por meio da partilha de nosso destino.

O ato soberano de definir, num \emph{horizonte decrescente de
expectativas},\footnote{\index{Arantes, Paulo E.}\versal{ARANTES}, \emph{op. cit.}} quem morre ou quem vive se
realiza com a morte de centenas de negros. A necropolítica realizada
diuturnamente, num país que jamais abandonou sua posição de periferia,
carrega em seu jardim regado a sangue o sonho de modernização e
progresso.\footnote{\index{Mbembe, Achille}\versal{MBEMBE}, \emph{op. cit.}}

É a situação que ganha através dos saberes divergentes que ela suscita,
diz \index{Stengers, Isabelle}Isabelle Stengers,\footnote{\versal{STENGERS}, I. \emph{No tempo das
  catástrofes} -- resistir à barbárie que se aproxima. São Paulo: Cosac
  Naify, 2015.} e, se ela estiver correta, a situação atual solapou
nossos referenciais teóricos; nossos saberes até aqui foram coniventes e
buscaram atuar nas limitações sistêmicas. Ignoramos até ontem o aumento
do assassinato da juventude das periferias; essa ignorância cobrou seu
preço e, infelizmente, com mais sangue.

A política reduzida à antipolítica, isto é, uma forma de guerra
permanente na qual o elemento democrático é esvaziado em nome das
``limitações'' econômicas enviesadas pela busca do lucro \emph{ad
infinitum}, impõe ao grosso da população a morte como resolução dos
possíveis \emph{conflitos vindouros}.

É preciso revisitar nosso entendimento sobre a noção de soberania e
como, no Brasil, essa noção, longe de buscar a autonomia dos indivíduos,
foi somente a instrumentalização dos corpos em prol da valorização do
capital imposta por uma elite econômica composta por Bentinhos -- ou
melhor, por piratas em busca de lucro. O Estado aqui foi sempre
\emph{Estado de} \emph{exceção}.\footnote{Isto
  é liberal na aparência, conservador e racista na realidade.}

Se, contudo, \index{Mbembe, Achille}Achille Mbembe estiver certo e a noção de negro estiver
sendo apartada da condição epidérmica, doravante o elemento
revolucionário é o negro. É para ele que os imperativos
contra"-insurgentes e as guerras de ocupação são perpetrados. Do Rio de
Janeiro à Palestina, da Turquia a Afrin, passando pela Catalunha e pelos
desabrigados de Detroit, essa condição parece se perpetuar. A
intervenção federal que cerca a favela, relembrando os velhos guetos
poloneses, é somente mais um indício dessa ``tendencial universalização
da condição negra.''\footnote{\versal{MBEMBE}, 2014, p. 16.}

Se o negro se tornou sobretudo uma condição de experimentar a si mesmo
como forma de vida imposta pela ``gestão dos destroços do
presente'',\footnote{\index{Arantes, Paulo E.}\versal{ARANTES}, \emph{op. cit.}, p. 91.} ou ele se torna
radicalmente anticapitalista ou não será nada.

Isto, porque ``a raça foi a sombra sempre presente sobre o pensamento e
a prática das políticas do Ocidente, especialmente quando se trata de
imaginar a desumanidade de povos estrangeiros -- ou
dominá"-los''.\footnote{\index{Mbembe, Achille}\versal{MBEMBE}, 2016, p. 128.} Excepcionalmente numa
posição em que se supere essa forma de política condicionada e
condicionante pela morte, poderemos exercer a indiferença pelas
diferenças num horizonte de revolução social.

Em uma época na qual a forma de aparição do negro foi redefinida em
função da configuração geral das hostilidades do Império do mercado, a
mais lamentável confusão diz respeito à ``regressão identitária''. Nesse
sentido, a perda da ilusão concernente à desalienação radical é a pedra
de toque de uma nova interpretação.

Já sabemos haver uma alienação radical, que é constitutiva da nossa
própria ordem simbólica. Ordem essa que impõe que nossa verdade esteja
fora de nós mesmos; uma linguagem que descentra nossa identidade. Se
somos descentrados, a única centralidade possível é um gesto de verdade
que nos torne sujeitos na busca por um destino comum. Um sonho podado
pela realidade que nos atravessa, impondo a necessidade de
ultrapassá"-la. Portanto, não temos ilusões.

Nem a identidade, nem a universalidade, nem a classe constituirão um
novo horizonte. Mas a relação entre esses limites. Não se trata da busca
de um paraíso perdido, mas da construção de possibilidades do que era
impossível a partir da compreensão de nossas condições de possibilidade.

É nesse momento que rememorando nosso terrível passado podemos redefinir
as coordenadas do presente:

\begin{quote}
Qualquer relato histórico do surgimento do terror moderno precisa tratar
da escravidão, que pode ser considerada uma das primeiras instâncias da
experimentação biopolítica. Em muitos aspectos, a própria estrutura do
sistema de colonização e suas consequências manifesta a figura
emblemática e paradoxal do estado de exceção. Aqui, essa figura é
paradoxal por duas razões. Em primeiro lugar, no contexto da
colonização, figura"-se a natureza humana do escravo como uma sombra
personificada. De fato, a condição de escravo resulta de uma tripla
perda: perda de um ``lar'', perda de direitos sobre seu corpo e perda de
status político. Essa perda tripla equivale a dominação absoluta,
alienação ao nascer e morte social (expulsão da humanidade de modo
geral). Para nos certificarmos, como estrutura político"-jurídica, a
fazenda é o espaço em que o escravo pertence a um mestre.\footnote{\index{Mbembe, Achille}\versal{MBEMBE},
  2016, p. 131.}
\end{quote}

Noutros termos, estivemos enquanto colônia até agora na vanguarda dos
processos de terror e controle social empregados pelo Estado. Sabemos
disso, sentimos em nossa pele diariamente. Nossos mortos já somam muitos
milhões. Doravante se trata do esforço por uma verdadeira democracia que
não se sustente mais na economia predatória do capital, da qual emanam
as formas de gestão da barbárie.

É necessário empurrar o ídolo ladeira abaixo.

As formas de gestão do capital em sua fase manipulatória\footnote{\index{Alves, Giovanni}\versal{ALVES},
  G. \emph{Trabalho e subjetividade:} o espírito do toyotismo na era do
  capitalismo manipulatório. São Paulo: Boitempo, 2011.} não são apenas
monopólios das grandes transnacionais. Elas invadem o espaço geográfico
das cidades; determinam o ritmo empenhado ao lucro pelo fetiche;
congregam vidas em torno do seu objetivo; objetivam a vida dos
indivíduos reduzindo"-os ao currículo e invadem sua subjetividade
dilacerada pelo modo de operação da linguagem.

A política moderna vinculada à esfera das demandas econômicas tornou"-se
uma política de resultados determinada pelo nível de investimentos e o
esperado retorno. Enquanto isso, ela exerce o ato soberano sobre quem
vive e quem morre com intuito de conter as insurgências. A polícia é o
braço assassino dessa política que paradoxalmente torna a condição negra
algo universal.

Não há mais política, somente polícia. Os espaços de discussão
``democrática'' reduzem"-se à discussão sobre a manutenção do lucro e a
redução dos danos que incidem no corte abrupto dos direitos da maioria
da população. Todos os partidos apegados à gramática e ao \emph{modus
operandi} da situação a"-histórica do capital -- ``já não há mais futuro''
- tornaram"-se partidos da ordem. Grande parte do movimento negro
dominado por dinheiro de fundações torna"-se partícipe dessas
prerrogativas.

Ainda que os partidos de esquerda sejam vítimas do ato soberano imposto
pela oligarquia, só podem superar a sua condição se superarem sua
limitação, ou seja, ultrapassarem as limitações de um órgão burocrático
guiado pela gestão das demandas do capital e organização interna.

O capital nunca cessou sua guerra aos opositores e se a mão invisível do
mercado é leve com seu riso de raposa nos países centrais, nas antigas
colônias, ao sul do globo, rege sua população com mão de ferro,
eliminando à bala todos os que contestam sua hegemonia. Sob o disfarce
de guerra ao tráfico se elimina, com pólvora e chumbo, um contingente
populacional preto e supérfluo anestesiado por uma violência estatal
constante que sustenta e se sustenta pelo narcotráfico e contrabando de
armas promovidos pelo próprio Estado.

No horizonte de desgraças, se ``como instrumento de trabalho, o escravo
tinha um preço e como propriedade tinha um valor'',\footnote{\index{Mbembe, Achille}\versal{MBEMBE},
  2016, p. 131.} na era das expectativas decrescentes ele torna"-se um
excedente que precisa ser eliminado. O mundo espectral de horrores e
crueldade fundamentado pela tecnologia da morte já pode quantificar do
espaço aéreo os gastos que se podem ter com o míssil ao assassinar de
uma vez cinquenta ou mais possíveis insurgentes. A condição negra chegou
à Palestina e sua neocolonização imposta por Israel. A condição negra se
expressa no extermínio das forças revolucionárias de Afrin sob um
silêncio implacável dos países civilizados. A condição negra se expressa
na expulsão de famílias dos \versal{EUA} pela etnologização da política. A
condição negra se expressa no assassinato de crianças em favelas pelas
balas ``perdidas'' da polícia militar.

Implacavelmente, os gestores da miséria avançam como um colosso
indestrutível cedendo aos seus lacaios benesses e espalhando \emph{think
tanks} para jovens ovelhas que queiram se ajoelhar ao seu império. Ao
mesmo tempo, mantêm, inseparável de sua política, o ódio fomentado pela
divisão racial e nos arcaísmos de sua estrutura impõem a luta intestina
pela sobrevivência no mercado de trabalho. Esse é o papel que a esquerda
durante muito tempo representou: convencer os milhões de miseráveis que
o sistema ainda é viável.

Contra essa situação de miséria atual somente uma organização negra --
entendida aqui como condição universalizada -- capaz de confrontar a
gramática e estabelecer desde já alternativas que se coloquem além das
formas estatais/capitalistas de controle poderá ser efetiva. A união de
todos os explorados, dos párias e da plebe, e não a unidade
programática, é a única forma de responder à altura os desafios que a
ofensiva do capitalismo impõe.

Se o fantasma das gerações passadas pesa na consciência dos vivos, não
há motivo para nostalgia das formas carcomidas por duzentos anos de
capital. O desbloqueio do novo é tarefa dos que se voltam contra a
universalidade do capital e que compreendem que só com sua superação
será possível uma vida digna. Nós negros sabemos o que essa
universalidade nos impôs e continua impondo\ldots{}

Já passou o tempo de choramingar e encontrar prazer numa melancolia
eterna que, nostálgica, se lembra dos velhos tempos democráticos. Foi no
lombo do negro que a república se perpetuou, elidindo de si a
participação do negro, assassinando"-o friamente pela fome ou pela arma.
Foi no lombo do negro que o
significante vazio de democracia se tornou uma ponte ideológica para
que, sob os olhos civilizados da Europa, países fossem invadidos e
crianças fossem assassinadas.

É sob o capitalismo que os direitos humanos servem para apresentar um
homem que já não tem mais nada e olha desesperado o seu filho cruzar a
fronteira sozinho. É em nome disso que moradores são cercados pelo
exército na maior favela do mundo, enquanto a burguesia nacional nos
seus jornais se orgulha de dar essa lição ao ``mundo civilizado''. É em
nome dessa política que uma companheira foi assassinada na segunda maior
cidade do país. E estão procurando quem foi. Sabemos quem foi. Todos
sabem!

Disputar esse significante vazio é tarefa urgente: de que democracia
estamos falando? Precisamos de um pastor ou deixaremos nosso destino nas
mãos de oligarquias? São esses limites ou seremos capazes de
implodi"-los? A democracia radical -- livre dos modos de existência do
capital -- será negra ou não será, o \emph{devir"-negro do mundo} precisa
ser subvertido. A transformação será obra daqueles que efetivamente
construíram este país.

Está na hora de novamente reivindicarmos a violência fanoniana: como uma
prática de ressimbolização social advinda da ruptura radical com a
estrutura sociossimbólica atual. É através da violência escolhida e não
mais sofrida que como colonizados poderemos obter uma reviravolta sobre
nós mesmos e nosso destino. Revolução é sofrimento.

Só assim nos libertaremos do colonialismo psíquico e objetivo que
engendrou neste país forças fundamentalmente necropolíticas animadas por
um instinto genocida e disseminadas pelo modo de sociabilidade
competitivo e classista do capital. Só assim deixaremos de produzir
mártires e só assim deixaremos de nos ocupar com os tons de nossa
epiderme. Só assim estaremos aptos a viver nossas vidas com todas as
nossas potencialidades. Só assim criaremos um mundo em que caibam todos
os outros.

\part{APÊNDICE}

\chapter*{Contra o retorno às raízes:\\ identidade e identitarismo no centro do debate}
\addcontentsline{toc}{chapter}{Contra o retorno às raízes: identidade e identitarismo no centro do debate
\bigskip}

A exaltação da identidade como algo
fixo, absoluto, algo dado,
pré"-existente, e não relativo é a pura expressão da forma de valorização
do capital como fim em si mesmo, que precisa assegurar para alguns
indivíduos uma colônia ainda viável de exploração.

Temos assistido no mundo, nos últimos anos, a um tsunami conservador
que, mediante a falta de perspectivas sólidas e alternativas concretas à
esquerda, criou uma miscelânea sincrética de sabedoria oriental com
filosofice barata, visando promover uma espécie de autoajuda para
``rebeldes''.

No Brasil, porém, cujo recorte constitutivo"-social, historicamente, é
africano e indígena -- mas também, europeu -- essa \emph{new wave} se
apropriou de temas caros a essas matrizes culturais criando uma espécie
de romantismo conservador adaptável ao mercado, que propõe uma espécie
de retorno às origens e se marca pelo resgate de um local místico
inexistente que fora ultrajado.

Este pequeno ensaio é não apenas para anunciar o óbvio -- o lugar
místico jamais existiu --, como também, para demonstrar como tais
caracteres já estão domesticados por aquilo que dizem contestar. Essa
posição deve ser bem definida, pois visa marcar uma contraposição ao
identitarismo que grassa no \emph{movimento negro} nos últimos anos.

Sendo seu autor militante -- ainda que desconhecido e sem almejar alguma
posição de liderança ou de falso guru -- desse movimento, acredita que o
debate pode ser feito para se recuperar a matriz contestatória dele.
Além disso, o autor tem razões profundas para crer que o \emph{movimento
negro}, em sua heterogeneidade, é uma das únicas forças capazes de
indicar uma superação efetiva das relações baseadas no capital nestes
tristes trópicos.

Uma hipótese básica da ofensiva neoliberal está em retomar os
pressupostos ideológicos da fundamentação estanque do indivíduo.

O Eu=Eu na sua perversa reatualização ante a nova conjuntura estrutural
do capital, a partir da década de 1970, encontra agora sua realização.
Não há mais um Outro que não seja o \emph{si} da consciência, e se acaso
houver um Outro -- como sempre há -- deve ser reconhecido como imagem e
semelhança do Eu. Usar as mesmas roupas, cabelos, etc. -- a despeito do
espectro político.

Sendo importante a discussão sobre identidades, ela, no entanto, se
estabelece atualmente em dois ramos dissidentes: um, cujo universal está
posto e do qual as determinações particulares são harmônicas, visto que
estão subsumidas ao universal; e outro, cuja luta pela manutenção da
singularidade é central, pois, supostamente, o universal tornou"-se
inviabilizado.

Contrariando as palavras de \index{Fanon, Frantz}Frantz Fanon, segundo o qual o ``indivíduo
deve tender ao universalismo inerente à condição humana'',\footnote{\versal{FANON},
  F. \emph{Pele negra, máscaras brancas}. Salvador: \versal{EDUFBA}, 2008, p. 32.}
essa segunda posição, tão limitada quanto a primeira, ergue"-se num
subjetivismo irracionalista apostando na particularidade e na luta por
territorialização dos espaços facilmente manipuláveis.

Sob o pretexto de que as regras da antiga lógica, da classificação, do
raciocínio e da definição não convêm às novas descobertas em diversos
campos do conhecimento, passou"-se a dissertar arbitrariamente ao sabor
da sentimentalidade e das intuições. Como, por isso, não se pôde ir além
da reflexão abstrata e das relações entre grupos, obedece"-se a processos
habituais de linguagem e criação conceitual numa literatura que tem por
sentido somente enriquecer a si mesma, tal como os antigos metafísicos
tentavam desvendar o sexo angelical e desenvolviam um jargão próprio aos
iniciados.

Foi, dessa maneira, que alguns conceitos foram hipostasiados e pararam
de comunicar qualquer sentido que não seja aquele identificado por
grupos de afetos e interesses comuns. A especialização das áreas e a
moderna divisão do trabalho intelectual nas academias decerto têm nisso
o seu quinhão.

Sendo as Ideias, entretanto, algo capaz de ganhar força material, a
confusão de conceitos de última hora ligados à conjuntura específica --
que, quanto mais específica e privada for, melhor -- converteu"-se numa
salada indigesta. Nada novo no horizonte, porém. Invadida pelo
``jornalismo'', a ``filosofia'' se tornou a senhora dos comentários
especializados de última hora no \emph{feed de notícias} do Facebook.

Esta sensaboria consiste essencialmente em fundamentar o pensamento e a
reflexão no sentimento imediato, na suposta ``conjuntura'', no
entusiasmo e na amizade. Se apega ao mais íntimo e ao mais pessoal,
abandonando o rigor de ver cada momento, em síntese, determinar o
concreto. Mediante tais resultados, abraça"-se o místico da vez e, com a
ruptura de qualquer estrutura \emph{simbólica} coerente, abraça"-se
aquilo que mais se parece com o meu"-Eu.

Esse remédio atual dispensa o esforço da cognição e inteligência. Joga
no campo dos sentimentalismos. O sentimento ingênuo e militante se
limita à verdade publicamente reconhecida com confiante convicção de que
está no lado certo do caminho certo. A esta atitude, porém, a
diversidade de opiniões se opõe, e num átimo tudo não passa de
posicionamento político e questão de a qual grupo eu pertenço.

Esse fosso pantanoso historicamente produzido não é, contudo, inédito.
Também não é inédito o elogio da estupidez, da ignorância como
ferramenta política, do logro -- em todo caso fascista --, da
desconfiança contra os intelectuais e o saber científico. A eloquência
com que a vulgaridade se pavoneia é assustadora.~ Claro está que há
razões fundamentadas socialmente para que chegássemos a este passo da
estupidez ser reverenciada em escala mundial.

O pensamento especulativo, todavia, tem ainda na filosofia sua guarida.
E se os adeptos de máximas ou de \emph{teorias da última hora} lhe
torcem o nariz é porque a reflexão exige que a preguiça e a
desonestidade intelectual sejam afastadas. A reflexão é contrária ao
obreirismo desesperado e às fugas na militância virtual ou real que
engendram subcelebridades como gurus dos novos tempos.

Isso também é política. Seria talvez a hora de resgatar a tarefa
necessária e vital da reflexão sobre a atividade científica e da teoria
como forma de ação. Assim, se atualmente os holofotes se voltam para
essa velha e caduca senhora chamada filosofia, talvez, seja porque ela
tenha muito a dizer numa época em que o fracasso da \emph{ficção
simbólica} induz os indivíduos a se apegarem a vários \emph{simulacros
imaginários}, retornando reacionariamente para uma postura religiosa e
emotiva.\footnote{\emph{Cf}. \index{Žižek, Slavoj}\versal{ŽIŽEK}, S. \emph{O
  Sujeito Incômodo}: o centro ausente da ontologia política. São Paulo:
  Boitempo, 2016.}

Como dizia \index{Fanon, Frantz}Frantz Fanon: ``Para nós, aquele que adora o preto é tão
`doente' quanto aquele que o execra. Inversamente, o negro que quer
embranquecer a raça é tão infeliz quanto aquele que prega o ódio ao
branco.''\footnote{\emph{Ibidem}, p. 35.}

Sabemos que a filosofia tem como tarefa reposicionar os problemas, não
resolvê"-los. Cabe a ela investigar o desenvolvimento dos conceitos e não
os produzir. Ao filisteu não nos toca convencê"-lo dessa tarefa, mas
àqueles que ainda acreditam que o pensamento e a reflexão são capazes de
suplantar o subjetivismo romântico e obscurantista que atualmente
circunda todos os meridianos ideológicos.

Também a crítica e a ciência são filhas de seu tempo, e o tempo traz
velhos problemas à sombra de novos contextos. As teorias da identidade
ressurgidas, sobretudo, a partir dos anos 1970, cujo apogeu se dá nos
anos 1990, invocam o colapso da modernização com o processo final de
globalização; o processo final de colonização capitalista das mentes e
corações sai das formas organizacionais fordistas e passa para as formas
do toyotismo manipulatório. Desapareceu com esse colapso a \emph{ficção
simbólica} com suas normas proibitivas que lança uma nova forma de
\emph{ideal imaginário superegóico}. Noutras palavras, na total ausência
de um Deus, outros deuses surgem -- sucesso social, busca pelo corpo
perfeito, identidade estanque --, e o corpo passa a ser avaliado como
empresa\ldots{}

Não é à toa que formas reacionárias com suas ferozes figuras do supereu
tenham ressurgido. Inversamente proporcional, as noções à direita
pan"-africanistas, pan"-eslavistas, pan"-europeias ascendem nessa tendência
da subjetividade pós"-moderna permissiva. Os simulacros imaginários de
uma recuperação do elo perdido e sagrado contribuem para uma noção de
pureza, em todo caso inexistente, mas que ganha, pelo seu caráter
narcisista, centenas de milhares de adeptos. Narciso gosta do que é
espelho, dizia o cantor popular.

Quando um dito grupo pan"-africanista impediu -- como a extrema direita
-- diversos outros participantes de hastearem sua bandeira contra a
escravidão na Líbia em nome da pureza e dos tons de pele daqueles que
participavam, já estava aceso o alarme de incêndio dessas práticas, cujo
simulacro imaginário reacionário era a
verdade.\footnote{Disponível em:
  \textless{}\emph{https://bit.ly/2FG0nfF}\textgreater{}.}

É interessante notar como os revisionistas querem impor e reduzir sob a
égide de ``pan"-africanismo'' à direita todos as manifestações e
organizações do povo preto, por meio de diversos malabarismos
conceituais. É interessante notar igualmente que aquilo que se
convencionou chamar de ``pan"-africanismo'' foi um modelo fundado em solo
inglês. Seus adeptos em solo tupiniquim silenciam sobre seus problemas,
adaptando suas ideias sem ao menos criticá"-las. Só para se ter ideia,
Muammar Al Kadhafi era um
pan"-africanista no interior da Unidade Africana que buscava, com seus
pares especuladores, criar um Banco Central para imprimir uma moeda
continental.

Sinalizar isso, se nada revela sobre o problema em si, traz a
necessidade de se refletir sobre o desdobramento da própria noção do que
é ser negro. Quando \index{Fanon, Frantz}Fanon diz que o negro não é um homem (\emph{ibidem}), já
deixa muito evidente a condição de não"-sujeito imposta ao negro. É aqui
que precisamos refletir sobre a incompletude do sujeito moderno e a
indeterminação \emph{pária} do negro como um \emph{não"-sujeito}.
Lembrando"-se ainda que a incompletude do sujeito moderno não é um
``privilégio'', senão, nas palavras do filósofo, uma ``descida aos
verdadeiros Infernos'', é preciso refletir como as promessas da
revolução burguesa foram todas abortadas.

A própria noção de negro é uma identidade criada a partir de um universo
de exclusão do qual o homem negro precisa se retirar. \index{Fanon, Frantz}Fanon, como bom
hegeliano que era, sabia que a luta do \emph{homem de cor} era se
liberar de si próprio. Isso não quer dizer um retorno mítico às raízes
ou a um continente que jamais existiu senão em referência direta aos
colonizadores. Silenciar sobre as posições francamente reacionárias de
um Garvey não me parece de bom tom. Criar um Estado militarizado em
nenhuma época ou lugar foi algo emancipatório.

Contrariamente, é interessante notar como na concepção de \index{Fanon, Frantz}Fanon a
universalidade efetiva é o poder do negativo, que traz para sua verdade
todas as particularidades, submetendo"-as e destruindo"-as no processo. A
contingência é então a verdade desse processo e isso detém um
significado fundamental: o universal é a experiência mediadora da
separação entre objetividade e subjetividade, ao passo que a totalidade
é a experiência da incompletude dessa individualidade.

Tal concepção só pode estar intrincada nas teias do pensamento
especulativo do qual \index{Fanon, Frantz}Fanon era legítimo conhecedor. A palavra de ordem é
assim a destruição revolucionária da própria noção de raça: \emph{Eu não
sou seu negro} é a tendência emancipatória no interior do pensamento de
\index{Fanon, Frantz}Fanon. Nenhuma dessas panaceias identitárias consegue engolir -- ainda
mais se tivermos como ponto de referência os estudos ``radicais'' afro
pós"-colonialistas que aportaram por aqui -- a ideia fundamental de \index{Fanon, Frantz}Fanon
do caráter inevitável da violência no processo concreto de
descolonização. É claro que isso não se dará com a consolidação de
nenhum Estado, senão com a própria destruição do sistema que criou a
escravidão moderna.

Então, como ficará claro adiante, utopia não é lutar pelo fim do
capital, mas acreditar que ele possa propiciar um horizonte de
expectativas crescentes que englobem aqueles que ele mesmo relegou como
párias -- e aqui cabem as mulheres, \versal{LGBT}s, negros, indígenas, etc.

Hoje setores preguiçosos e mal"-intencionados no interior do movimento
negro -- que é, em todo caso, felizmente, heterogêneo -- buscam uma
inversão desse prognóstico: para eles o capitalismo, que se estruturou
pelo racismo e exclusão de grandes contingentes populacionais, um dia
acordará de bom humor e deixará de ser racista e excludente.

Para desmistificar essa posição é interessante notar como o capitalismo
brasileiro em sua relação de subordinação ao capital global acabou por
determinar uma superexploração do trabalho, que caiu, ao longo dos anos,
sobretudo, nas costas da população negra. O capital não só criou a noção
contemporânea de raças, inexistentes por exemplo durante a Idade Média,
como na moderna divisão do mercado de trabalho internacional e nacional
regulou os espaços supostamente ``adequados'' para cada tipo de raça.
\index{Fernandes, Florestan}Florestan Fernandes em seu clássico \emph{A integração do negro na
sociedade de classes} demonstra que ``o que há de essencial, para a
análise do negro na ordem econômica e social emergente, é que eles foram
excluídos, como categoria social das tendências modernas de expansão do
capitalismo''.\footnote{\index{Fernandes, Florestan}\versal{FERNANDES}, F. \emph{A integração do negro na
  sociedade de classes}. São Paulo: Globo, 2008, p. 72.}

Ora, é justamente esse desenvolvimento excludente que vai não só
fomentar a estruturação das áreas periféricas como ainda interferir na
designação de lugares para trabalhadores negros na estrutura produtiva.
Assim, a questão da exclusão racial é absorvida pela estrutura produtiva
do capital, que a utiliza como parâmetro na contratação de mão"-de"-obra.

É por isso que algumas tendências teóricas chegarão à lúcida conclusão
de que no Brasil \emph{o} \emph{racismo é estrutural}. Isso significa
dizer que a ordem político"-econômica desenvolvida por aqui tem seus
lastros contraditórios na passagem da escravidão para o trabalho livre.
Dado o histórico dessa passagem, o legado da escravidão é o que
constituirá a forma de exploração do trabalho pelo capital que será
inserida em toda a estrutura econômico"-social.

Assim, a manutenção das relações sociais no interior do capital se deu
por aqui em sua correspondência direta com o escravismo, legando uma
diferenciação na classe trabalhadora, que fica evidente, sobretudo, por
meio dos salários. Segundo estudo divulgado pelo Departamento
Intersindical de Estatística e Estudos Socioeconômicos (Dieese), em
2013, o negro brasileiro recebia em média salários 36,11\% menores que o
branco.

A coisa é ainda pior para a mulher negra, que ganha um terço do salário
dos brancos e metade do salário de um homem negro. Não obstante, na
maioria das vezes as mulheres negras ocupam o setor de serviços
domésticos cuja porcentagem é de 19,2\% comparada a 10,6\% de mulheres
não negras. Tais dados demonstram que a estruturação do mercado de
trabalho detém um componente de racialização, legado da escravidão, que
em tempos de precarização fatalmente incidirá negativamente sobre essa
população.

Lutar por tais demandas, isto é, pela equiparação dos salários, é
obviamente uma demanda classista. Dada a estruturação racializada do
mercado de trabalho brasileiro, tal demanda aponta para além de si
própria ao desnudar as relações promíscuas e indecentes do sistema
liberal. Não apenas isso, mas se levarmos em consideração que a
população negra no Brasil é de 53,6\% veremos então que ela compõe a
quase totalidade da classe trabalhadora.

Sendo assim, se as questões negras em seu pertencimento e necessidade
foram deixadas de lado, historicamente, pela esquerda nacional, o que
pode ser posto nas costas do marxismo tradicional, que, em sua
formulação histórica, foi estritamente dogmático, abandonando a própria
perspectiva da análise crítica. Por outro lado, algumas ideias ``fora do
lugar'' que adentraram aqui por meio dos estudos culturalistas tiveram o
seu quinhão que atualmente causa grande confusão, deixando margem para o
obscurantismo. É a partir disso que temos que refletir sobre a noção de
identidade.

Atualmente o conceito de
identidade não recai simplesmente numa ânsia de rigor meramente
filológico. Sua força atual se afirma porque se mostra como uma
necessidade concreta devido a uma série de fenômenos sociais
problemáticos que se assentam sobretudo porque, abarcando muito além do
processo produtivo e se desenvolvendo para além dele, o processo de
\emph{valorização do capital} tendeu a realizar"-se superando o próprio
movimento e suprimindo cada vez mais o espaço"-tempo determinado pela
produção e reprodução da mercadoria.

Isso explica duas coisas:
a) \emph{a chamada crise
do valor}, pois, (\versal{I}) o processo produtivo altera pouco a
afluência da realização do capital, já que passa a ocorrer um
valor"-sem"-valor -- isto é, a busca da realização do lucro numa relação
em que se joga dinheiro para obter mais"-dinheiro (\versal{D"-D}'); (\versal{II})
ocorre a quebra e queda geral dos níveis de reprodução da força de
trabalho, que, por sua vez, cada vez mais, fica precarizada, e
(\versal{III}) instala"-se um estado de crise permanente como forma de
governo; b) \emph{o modo como esse processo estrangula o próprio
``processo''}, quer dizer, como as tendências mais harmônicas do
processo de produção, na reprodução do capital como fim em si mesmo, são
solapadas, tornando patente que a violência é desencadeada em várias
frentes, quer seja pela
\emph{despossessão},\footnote{Essa
  noção é de \index{Harvey, David}David Harvey.} quer seja pela superexploração do trabalho.

Daí que é preciso observar as
diferenças entre a noção de identidade, como um ponto de verdade que
possibilita o desenvolvimento contraditório da subjetividade sempre em
devir, e das práticas identitárias num sentido particularista, não
relacional e estanque em si mesmas, que vêm assegurar a possibilidade de
inclusão nesse processo, visto que o terreno próprio das lutas fordistas
desapareceu. A noção identitária marcada como fim em si já é,
paradoxalmente, o fim da subjetividade. Ela já é a identificação da
própria identidade com os modos de gestão do capital (\emph{poder}).

A exaltação da identidade como algo fixo, absoluto, algo dado,
pré"-existente, e não relativo, é a pura expressão da forma de
valorização do capital como fim em si mesmo, que precisa assegurar para
alguns indivíduos uma colônia ainda viável de exploração. É esse
fenômeno que busca uma identidade estanque, ideal e não relativa, um
Eu=Eu, como forma inconsciente de realização de valorização do capital,
que chamo de identitarismo.

Por outro lado, implodida a esperança com qualquer afirmação
representativa no interior da ordem constituída, vislumbra"-se uma
possibilidade emancipatória concernente à identidade. Sabemos que desde
Antígona é a posição da identidade intransigente que põe abaixo todo o
edifício universal dando a este uma nova forma. Por isso, a superação só
pode ser efetivada naquilo que as palavras de \index{Fanon, Frantz}Fanon ilustram bem:

\begin{quote}
Eis na verdade o que se passa: como percebo que o preto é o símbolo do
pecado, começo a odiá"-lo. Porém constato que sou negro. Para escapar ao
conflito, duas soluções. Ou peço aos outros que não prestem atenção à
minha cor, ou, ao contrário, quero que eles a percebam. Tento, então,
valorizar o que é ruim -- visto que, irrefletidamente, admiti que o
negro é a cor do Mal. Para pôr um termo a esta situação neurótica, na
qual sou obrigado a escolher uma solução insana, conflitante, alimentada
por fantasmagorias, antagônica, desumana enfim, -- só tenho uma solução:
passar por cima deste drama absurdo que os outros montaram ao redor de
mim, afastar estes dois termos que são igualmente inaceitáveis e,
através de uma particularidade humana, tender ao universal.\footnote{\versal{FANON},
  \emph{ibidem}, p. 166.}
\end{quote}

\index{Fanon, Frantz}Fanon chama a atenção para a possibilidade de solapar a universalidade
que constitui essa particularidade como algo excludente. Noutros termos,
não há possibilidades reais de superação das tendências racistas do
capital no jogo que ele próprio impôs. Por outro lado, a particularidade
do negro tem em si a potencialidade de suplantar essa condição não
aceitando os termos postos a partir da reivindicação de sua própria
particularidade.

Donde se destacam duas conclusões importantes, e aqui encerro: 1) utopia
é acreditar que o capital -- que se fundamentou por meio da escravidão
moderna -- deixará de ser racista; 2) a identidade como componente de
reivindicação em seu caminho ao universal irá se dissolver enquanto tal
a partir do momento em que sua posição se tornar universal. Desse modo,
o evanescimento da identidade faz parte desse processo, não há uma
individualidade incorruptível porque ela é social e historicamente
determinada. Todos, portanto, independentemente de cor e religião, são
bem"-vindos à luta pela emancipação e podem hastear sua bandeira. A
solidariedade comum e pelo comum continua sendo o caminho.
